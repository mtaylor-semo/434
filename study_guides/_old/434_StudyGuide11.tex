%!TEX TS-program = lualatex
%!TEX encoding = UTF-8 Unicode

\documentclass[nofonts, letterpaper]{tufte-handout}

%\geometry{showframe} % display margins for debugging page layout

\usepackage{graphicx} % allow embedded images
  \setkeys{Gin}{width=\linewidth,totalheight=\textheight,keepaspectratio}
  \graphicspath{{img/}} % set of paths to search for images
  
\usepackage{fontspec}
  \setmainfont[Ligatures={Common,TeX},Numbers={Proportional}]{Linux Libertine O}
  \setsansfont{Linux Biolinum O}
\usepackage{microtype}
\usepackage{enumitem}
\usepackage{multicol} % multiple column layout facilities
%\usepackage{hyperref}
%\usepackage{fancyvrb} % extended verbatim environments
%  \fvset{fontsize=\normalsize}% default font size for fancy-verbatim environments

% Change the header to shift the title to the left side of the page. 
% Replaced \quad with \hfill.  See \plaintitle in tufte-common.def
{\fancyhead[RE,RO]{\scshape{\newlinetospace{\plaintitle}}\hfill\thepage}}

\makeatletter
% Paragraph indentation and separation for normal text
\renewcommand{\@tufte@reset@par}{%
  \setlength{\RaggedRightParindent}{1.0pc}%
  \setlength{\JustifyingParindent}{1.0pc}%
  \setlength{\parindent}{1pc}%
  \setlength{\parskip}{0pt}%
}
\@tufte@reset@par

% Paragraph indentation and separation for marginal text
\renewcommand{\@tufte@margin@par}{%
  \setlength{\RaggedRightParindent}{0pt}%
  \setlength{\JustifyingParindent}{0.5pc}%
  \setlength{\parindent}{0.5pc}%
  \setlength{\parskip}{0pt}%
}

\makeatother

\title{Study Guide 11}
\author{Estuaries and Productivity}

\date{} % without \date command, current date is supplied

\begin{document}

\maketitle	% this prints the handout title, author, and date

%\printclassoptions

\section{Vocabulary}
\marginnote{\textbf{Study:} pgs. 375--408.}
\vspace{-1\baselineskip}
\begin{multicols}{2}
estuary \\
coastal plain estuary \\
bar-built estuary \\
tectonic estuary \\
fjord \\
euryhaline \\
stenohaline \\
detritus \\
outwelling hypothesis
\end{multicols}

\section{Concepts}

\begin{enumerate}
\item
  Compare and contrast the four major types of estuaries. Consider how
  morphology and formation of the estuary determines the relative
  importance of the different environmental factors that affect
  community structure (e.g., salinity, water flow, substrate, etc.).
\item
  Diagram the salinity gradient in a ``typical'' estuary during high
  tide and low riverine input, and low tide and high riverine input.
  Indicate the region in which organisms must have the greatest
  tolerance for salinity variation.
\item
  Does temperature show a similar or different gradient compared to what
  you diagrammed for the previous question? Explain why.
\item
  Explain the relationship between wind and GPP in a bar-built estuary.
  Would you expect the relationship to differ in a coastal-plain
  estuary? Explain why or why not.
\item
  Explain how sediments are size-distributed in an estuary. Do the same
  for POM.
\item
  Diagram the distribution of freshwater, brackish, euryhaline and
  stenohaline species in an estuary. Explain why these four groups of
  organisms show this distributional pattern.
\item
  What are the sources of primary productivity in a salt marsh estuary?
  How is NPP exported out of the estuary?
\item
  Explain the outwelling hypothesis. What types of estuaries best fit
  this hypothesis? What types of estuaries don't fit the outwelling
  hypothesis? (Hint: think of American vs European estuaries, but don't
  use those terms.)
\item
  Be familiar with the meaning of detritus that we are using for this
  class.
\item
  Explain the role of aerobic bacteria in estuarine food webs.
\item
  What is the single physical process that most influences community
  structure in a bar-built estuary? A coastal plain estuary? A fjord? A
  tectonic estuary? Explain why for each.
\item
  Explain phytoplankton blooms in an estuary. How are they similar to
  and different from coastal spring phytoplankton blooms?
\item
  In what ways can light intensity limit GPP in a shallow estuary (e.g.,
  bar-built).
\end{enumerate}

\end{document}