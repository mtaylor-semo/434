%!TEX TS-program = lualatex
%!TEX encoding = UTF-8 Unicode

\documentclass[nofonts, letterpaper]{tufte-handout}

%\geometry{showframe} % display margins for debugging page layout

\usepackage{graphicx} % allow embedded images
  \setkeys{Gin}{width=\linewidth,totalheight=\textheight,keepaspectratio}
  \graphicspath{{img/}} % set of paths to search for images
  
\usepackage{fontspec}
  \setmainfont[Ligatures={Common,TeX},Numbers={Proportional}]{Linux Libertine O}
  \setsansfont{Linux Biolinum O}
\usepackage{microtype}
\usepackage{enumitem}
\usepackage{multicol} % multiple column layout facilities
%\usepackage{hyperref}
%\usepackage{fancyvrb} % extended verbatim environments
%  \fvset{fontsize=\normalsize}% default font size for fancy-verbatim environments

% Change the header to shift the title to the left side of the page. 
% Replaced \quad with \hfill.  See \plaintitle in tufte-common.def
{\fancyhead[RE,RO]{\scshape{\newlinetospace{\plaintitle}}\hfill\thepage}}

\makeatletter
% Paragraph indentation and separation for normal text
\renewcommand{\@tufte@reset@par}{%
  \setlength{\RaggedRightParindent}{1.0pc}%
  \setlength{\JustifyingParindent}{1.0pc}%
  \setlength{\parindent}{1pc}%
  \setlength{\parskip}{0pt}%
}
\@tufte@reset@par

% Paragraph indentation and separation for marginal text
\renewcommand{\@tufte@margin@par}{%
  \setlength{\RaggedRightParindent}{0pt}%
  \setlength{\JustifyingParindent}{0.5pc}%
  \setlength{\parindent}{0.5pc}%
  \setlength{\parskip}{0pt}%
}

\makeatother

\title{Study Guide 01}
\author{Oceanography}

\date{} % without \date command, current date is supplied

\begin{document}

\maketitle	% this prints the handout title, author, and date

%\printclassoptions

\section{Using the Study Guides}
The\marginnote{The pages to study listed in the study guides are required reading and fair game for exams.\\ \textbf{Study:} pgs. 1--18\\ \textbf{Review:} pgs. 19--25; 46--59.} study guides will help you learn the material.  Each study guide contains vocabulary to learn and a series of questions based on the lectures and the assigned reading from the textbook.  The guides may also contain information to supplement the lecture.  Read the study guides in advance of lecture to get familiar with the day's topic. Bring the study guide to class to see the vocabulary and questions in context of the lecture discussion.  This will help you recall the information during your regular study. I will not collect your answers to study guide questions.

\section{Vocabulary}
The vocabulary lists terms from each lecture that you learn and be able to apply in a broader context.  I will use the terms in lectures and on exams. If you do not know the terms, you may not understand the lecture or be able to answer an exam question. I expect you to use the proper vocabulary in your answers to questions on exams and assignments.  Use the terms as you learn the material.  I may not cover all terms in class or do so only in passing.  I expect that you will learn them by reading your textbook and using the terms in the context of the course material.

%\vspace{-1\baselineskip}
\begin{multicols}{2}
Arctic Ocean \\
Atlantic Ocean \\
Indian Ocean \\
Pacific Ocean \\
Southern Ocean \\
abyssal plain \\
continental shelf \\
continental slope \\
oceanic ridge \\
trench \\
oxygen minimum zone \\
carbonate buffering system
\end{multicols}

\subsection*{General ecological terms you should be able to apply,}
\begin{multicols}{2}
autotroph (-ic) \\
heterotroph (-ic) \\
trophic level \\
trophic structure \\
herbivore \\
carnivore \\
omnivore \\
decomposers \\
biogeochemical cycles \\
fundamental niche \\
realized niche \\
species richness \\
species diversity \\
ecological succession \\
competition \\
competitive exclusion \\
predation (and predator) \\
prey \\
grazer \\
parasite
\end{multicols}


\section{Concepts}

You should \emph{write} clear and concise answers to each question in this section.  The questions are not necessarily independent.  Think broadly across lectures to see ``the big picture.''  Study guide questions may be used as a basis for short answer or essay exam questions. I may also write exam and assignment questions that do not appear on the study guides. These are guides, not exhaustive test banks.

\begin{enumerate}
	\item Draw a representative graph of dissolved oxygen concentration as a function of depth.  What is the oxygen minimum zone? What factors cause the oxygen minimum zone?

	\item Explain the role of bicarbonate and carbon dioxide as a buffer that maintains the alkaline state of the oceans.  What could happen if CO$_2$ levels in the atmosphere become to high?  What types of marine organisms would be most affected and why?

	\item Name and identify the five oceans.  Draw a typical ocean basin profile.

	\item Describe the basic properties of seawater.

\end{enumerate}


\end{document}