%!TEX TS-program = lualatex
%!TEX encoding = UTF-8 Unicode

\documentclass[nofonts, letterpaper]{tufte-handout}

%\geometry{showframe} % display margins for debugging page layout

\usepackage{graphicx} % allow embedded images
  \setkeys{Gin}{width=\linewidth,totalheight=\textheight,keepaspectratio}
  \graphicspath{{img/}} % set of paths to search for images
  
\usepackage{fontspec}
  \setmainfont[Ligatures={Common,TeX},Numbers={Proportional}]{Linux Libertine O}
  \setsansfont{Linux Biolinum O}
\usepackage{microtype}
\usepackage{enumitem}
\usepackage{multicol} % multiple column layout facilities
%\usepackage{hyperref}
%\usepackage{fancyvrb} % extended verbatim environments
%  \fvset{fontsize=\normalsize}% default font size for fancy-verbatim environments

% Change the header to shift the title to the left side of the page. 
% Replaced \quad with \hfill.  See \plaintitle in tufte-common.def
{\fancyhead[RE,RO]{\scshape{\newlinetospace{\plaintitle}}\hfill\thepage}}

\makeatletter
% Paragraph indentation and separation for normal text
\renewcommand{\@tufte@reset@par}{%
  \setlength{\RaggedRightParindent}{1.0pc}%
  \setlength{\JustifyingParindent}{1.0pc}%
  \setlength{\parindent}{1pc}%
  \setlength{\parskip}{0pt}%
}
\@tufte@reset@par

% Paragraph indentation and separation for marginal text
\renewcommand{\@tufte@margin@par}{%
  \setlength{\RaggedRightParindent}{0pt}%
  \setlength{\JustifyingParindent}{0.5pc}%
  \setlength{\parindent}{0.5pc}%
  \setlength{\parskip}{0pt}%
}

\makeatother

\title{Study Guide 08}
\author{Deep-Sea Communities}

\date{} % without \date command, current date is supplied

\begin{document}

\maketitle	% this prints the handout title, author, and date

%\printclassoptions

\section{Vocabulary}
\marginnote{\textbf{Study:} pgs. 466--477. Review r- and K-life history strategies from any reliable source. Review the small amount of information on intermediate disturbance on page 73. Review the handout from class on the three hypotheses proposed to explain deep-sea benthic diversity.}
\vspace{-1\baselineskip}
\begin{multicols}{2}
r-type life history strategy \\
K-type life history strategy \\
depth gradient \\
latitudinal gradient \\
intermediate disturbance \\
patch dynamics \\
spatial heterogeneity \\
sediment size
\end{multicols}

\section{Concepts}

\begin{enumerate}
\item
  How does deep sea biomass correlate with the nutritional quality of
  marine snow.
\item
  How does density correlate with depth in deep-sea benthic
  (bottom-living) communities?
\item
  We discussed three important hypotheses that were proposed between
  1968 and 1979 to explain high species diversity in deep-sea benthic
  communities. Briefly explain each hypothesis, and then explain why
  each does not adequately explain deep-sea benthic diversity.
\item
  We discussed three important that were proposed between 1968 and 1978
  to explain high species diversity in deep-sea benthic communities.
  However, none of those hypotheses adequately explain deep-sea benthic
  diversity. Following diversity studies in the 1980s and 1990s, a
  hypothesis was developed that does seem to explain diversity. Describe
  the hypothesis and the interaction of the three factors that have been
  proposed to explain deep-sea benthic diversity.
\item
  Diversity gradients in deep-sea benthic communities have been observed
  both for depth and for latitude. What hypothesis or hypotheses have
  been proposed to explain these gradients?
\item
  How does the mesopelagic nekton community contribute to the transfer
  of NPP from the epipelagic to the mesopelagic (and deeper)?
 \end{enumerate}


\end{document}