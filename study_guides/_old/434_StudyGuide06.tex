%!TEX TS-program = lualatex
%!TEX encoding = UTF-8 Unicode

\documentclass[nofonts, letterpaper]{tufte-handout}

%\geometry{showframe} % display margins for debugging page layout

\usepackage{graphicx} % allow embedded images
  \setkeys{Gin}{width=\linewidth,totalheight=\textheight,keepaspectratio}
  \graphicspath{{img/}} % set of paths to search for images
  
\usepackage{fontspec}
  \setmainfont[Ligatures={Common,TeX},Numbers={Proportional}]{Linux Libertine O}
  \setsansfont{Linux Biolinum O}
\usepackage{microtype}
\usepackage{enumitem}
\usepackage{multicol} % multiple column layout facilities
%\usepackage{hyperref}
%\usepackage{fancyvrb} % extended verbatim environments
%  \fvset{fontsize=\normalsize}% default font size for fancy-verbatim environments

% Change the header to shift the title to the left side of the page. 
% Replaced \quad with \hfill.  See \plaintitle in tufte-common.def
{\fancyhead[RE,RO]{\scshape{\newlinetospace{\plaintitle}}\hfill\thepage}}

\makeatletter
% Paragraph indentation and separation for normal text
\renewcommand{\@tufte@reset@par}{%
  \setlength{\RaggedRightParindent}{1.0pc}%
  \setlength{\JustifyingParindent}{1.0pc}%
  \setlength{\parindent}{1pc}%
  \setlength{\parskip}{0pt}%
}
\@tufte@reset@par

% Paragraph indentation and separation for marginal text
\renewcommand{\@tufte@margin@par}{%
  \setlength{\RaggedRightParindent}{0pt}%
  \setlength{\JustifyingParindent}{0.5pc}%
  \setlength{\parindent}{0.5pc}%
  \setlength{\parskip}{0pt}%
}

\makeatother

\title{Study Guide 06}
\author{Nekton (and some plankton)}

\date{} % without \date command, current date is supplied

\begin{document}

\maketitle	% this prints the handout title, author, and date

%\printclassoptions

\section{Vocabulary}
\marginnote{\textbf{Study:} pgs. 109--117, 258--261. The text does not have much on adaptations to the epipelagic so study your notes carefully.}
\vspace{-1\baselineskip}
\begin{multicols}{2}
nekton \\
epipelagic \\
holoepipelagic \\
meroepipelagic
\end{multicols}

\section{Concepts}

\begin{enumerate}
\item
  Which groups of organisms are nektonic?
\item
  Marine birds such as gulls, puffins, petrels and albatrosses are
  considered terrestrial organisms but they are an important component
  of the trophic ecology of the epipelagic environment. Explain the
  trophic role of marine birds in the epipelagic zone.
\item
  Compare and contrast the adaptations of planktonic and nektonic
  organisms that allow them to feed and survive in the epipelagic zone.
  You should consider water density, water clarity, light availability,
  the large area, low diversity, and the lack of structure.
\item
  Compare and contrast the epipelagic food webs for polar, cold
  temperate and tropical waters. Include what you know about planktonic
  trophic ecology, especially for the tropical epipelagic food webs.
\item
  Describe the different ways in which nektonic organisms are
  camouflaged.
\item
  Describe how the keel on many nektonic fishes increases their
  camouflage.
\item
  Describe the many ways in which plankton and nekton maintain neutral
  buoyancy.
\item
  The dense water medium often selects naturally for streamlined body
  shapes. Yet, not all nektonic organisms are efficiently streamlined.
  Why are small filter-feeding fishes such as sardines and herrings, as
  well as large nekton like whale sharks and baleen whales, not
  particularly streamlined?
\item
  How does a streamlined body shape in fast-swimming fishes increase
  foraging efficiency?
\item
  Why is body coloration as a form of camouflage so important in the
  epipelagic environment?
 \end{enumerate}


\end{document}