%!TEX TS-program = lualatex
%!TEX encoding = UTF-8 Unicode

\documentclass[nofonts, letterpaper]{tufte-handout}

%\geometry{showframe} % display margins for debugging page layout

\usepackage{graphicx} % allow embedded images
  \setkeys{Gin}{width=\linewidth,totalheight=\textheight,keepaspectratio}
  \graphicspath{{img/}} % set of paths to search for images
  
\usepackage{fontspec}
  \setmainfont[Ligatures={Common,TeX},Numbers={Proportional}]{Linux Libertine O}
  \setsansfont{Linux Biolinum O}
\usepackage{microtype}
\usepackage{enumitem}
\usepackage{multicol} % multiple column layout facilities
%\usepackage{hyperref}
%\usepackage{fancyvrb} % extended verbatim environments
%  \fvset{fontsize=\normalsize}% default font size for fancy-verbatim environments

% Change the header to shift the title to the left side of the page. 
% Replaced \quad with \hfill.  See \plaintitle in tufte-common.def
{\fancyhead[RE,RO]{\scshape{\newlinetospace{\plaintitle}}\hfill\thepage}}

\makeatletter
% Paragraph indentation and separation for normal text
\renewcommand{\@tufte@reset@par}{%
  \setlength{\RaggedRightParindent}{1.0pc}%
  \setlength{\JustifyingParindent}{1.0pc}%
  \setlength{\parindent}{1pc}%
  \setlength{\parskip}{0pt}%
}
\@tufte@reset@par

% Paragraph indentation and separation for marginal text
\renewcommand{\@tufte@margin@par}{%
  \setlength{\RaggedRightParindent}{0pt}%
  \setlength{\JustifyingParindent}{0.5pc}%
  \setlength{\parindent}{0.5pc}%
  \setlength{\parskip}{0pt}%
}

\makeatother

\title{Study Guides 02 and 03}
\author{Oceanography and Larval Ecology}

\date{} % without \date command, current date is supplied

\begin{document}

\maketitle	% this prints the handout title, author, and date

%\printclassoptions

\section{Vocabulary}\marginnote{\textbf{Study:} pgs. 24--38, 42--77; 141--145, 155--159 (larval ecology).}
\vspace{-1\baselineskip}
\begin{multicols}{2}
gyre \\
plate tectonics \\
sea floor spreading \\
subduction \\
thermocline \\
neritic \\
oceanic \\
pelagic \\
epipelagic \\
mesopelagic \\
bathypelagic \\
abyssopelagic \\
hadopelagic \\
photic \\
disphotic \\
aphotic \\
littoral \\
intertidal \\
sublittoral \\
benthic \\
bathyal \\
abyssal \\
hadal \\
facilitation model \\
inhibition model \\
tolerance model \\
sessile \\
filter feeder \\
larvae (singular: larva) \\
planktotrophic larvae \\
lecithotrophic larvae \\
nonpelagic larvae 
\end{multicols}

\section{Concepts}

\begin{enumerate}
\item
  Assume you are on a ship at the equator and you sample water
  temperature from the surface to the sea floor. Graph the temperature
  as a function of water depth (just draw a typical profile). Label the
  three layers and the thermocline. Explain how the water temperature
  changes in each layer. How would you expect this temperature profile
  to change if you sampled water temperatures in the Southern Ocean?
\item
  What is the great ocean conveyor? Roughly where does the surface water
  sink? What causes the water to sink? How does the conveyor affect
  deep-sea communities?
\item
  What are the four major biogeographic realms in the oceans? What
  physical factor(s) determine these zones?
\item
  If you were to look at a band of tropical water (surface temperature
  of 25°C and higher) across an ocean basin, the band would be wider
  along the west side of the basin and narrower along the east side of
  the basin. What explains this phenomenon?
\item
  Explain why succession in the marine environment, if it occurs at all,
  is very different from the classical or facilitation model of
  terrestrial succession? What models have been developed to explain
  community succession in the marine environment? Which model do you
  think seems most likely to occur in the marine environment? Justify
  your answer.
\item
  When comparing terrestrial and marine ecosystems, more phyla and
  classes are found in marine ecosystems, but more genera and species
  are found in terrestrial ecosystems. Describe the abiotic and biotic
  factors that potentially explain this discrepancy between the two
  ecosystems.
\item
  Compare and contrast the three types of larval development. Consider
  larval dispersal ability, larval mortality, clutch size (number of
  offspring produced), species distribution, and population genetic
  structure of the species. Relate population genetic structure to the
  potential for speciation to occur (this will take some thought).
\item
  The density of water creates a new type of community not found in
  terrestrial ecosystems. What is this community? What types of
  adaptations evolved in other communities to take advantage of this new
  community? (Hint: think trophic). Provide several examples of the
  adaptations.
\item
  Discuss the types of cues that planktonic larvae may use to determine
  whether they are settling into the proper habitat.
\item
  Know the basic zones of the ocean. They will form the framework of our
  community discussions.
\item
  Are the photic, disphotic, and aphotic zones of the ocean synonyms for
  other names of the oceans zones? For example, is the photic zone
  synonymous with the epipelagic zone? Are some of the zones
  functiinally equivalent? Explain.
  \end{enumerate}


\end{document}