%!TEX TS-program = lualatex
%!TEX encoding = UTF-8 Unicode

\documentclass[nofonts, letterpaper]{tufte-handout}

%\geometry{showframe} % display margins for debugging page layout

\usepackage{graphicx} % allow embedded images
  \setkeys{Gin}{width=\linewidth,totalheight=\textheight,keepaspectratio}
  \graphicspath{{img/}} % set of paths to search for images
  
\usepackage{fontspec}
  \setmainfont[Ligatures={Common,TeX},Numbers={Proportional}]{Linux Libertine O}
  \setsansfont{Linux Biolinum O}
\usepackage{microtype}
\usepackage{enumitem}
\usepackage{multicol} % multiple column layout facilities
%\usepackage{hyperref}
%\usepackage{fancyvrb} % extended verbatim environments
%  \fvset{fontsize=\normalsize}% default font size for fancy-verbatim environments

% Change the header to shift the title to the left side of the page. 
% Replaced \quad with \hfill.  See \plaintitle in tufte-common.def
{\fancyhead[RE,RO]{\scshape{\newlinetospace{\plaintitle}}\hfill\thepage}}

\makeatletter
% Paragraph indentation and separation for normal text
\renewcommand{\@tufte@reset@par}{%
  \setlength{\RaggedRightParindent}{1.0pc}%
  \setlength{\JustifyingParindent}{1.0pc}%
  \setlength{\parindent}{1pc}%
  \setlength{\parskip}{0pt}%
}
\@tufte@reset@par

% Paragraph indentation and separation for marginal text
\renewcommand{\@tufte@margin@par}{%
  \setlength{\RaggedRightParindent}{0pt}%
  \setlength{\JustifyingParindent}{0.5pc}%
  \setlength{\parindent}{0.5pc}%
  \setlength{\parskip}{0pt}%
}

\makeatother

\title{Study Guide 04}
\author{Plankton and Primary Production}

\date{} % without \date command, current date is supplied

\begin{document}

\maketitle	% this prints the handout title, author, and date

%\printclassoptions

\section{Vocabulary}
\marginnote{\textbf{Study:} pgs. 167--173 (plankton communities); 258--268 (primary productivity).}
\vspace{-1\baselineskip}
\begin{multicols}{2}
plankton \\
phytoplankton \\
zooplankton \\
mixoplankton \\
holoplankton \\
meroplankton \\
net plankton \\
nanoplankton \\
diatoms \\
Bacillariophyceae \\
dinoflagellates \\
Dinophyceae \\
primary production \\
gross primary production \\
net primary production \\
compensation depth \\
compensation intensity \\
PAR
\end{multicols}

\section{Concepts}

\begin{enumerate}
\item
  You should know the basic nutrient acronyms: DIC, DOC, DIP, DOP, DIN,
  DON, DOM, etc.
\item
  What is plankton? Describe the many ways in which planktonic organisms
  can be categorized.
\item
  What is primary production? What is photosynthesis? Are they the same
  thing? Explain why or why not?
\item
  Explain the difference between gross and net primary production.
\item
  Explain the differences and similarities of compensation depth and
  compensation intensity.
\item
  Explain the factors that influence photosynthetically available
  radiation (PAR) at depth in the water column (for example, at 50 or
  75m of depth). Consider both the atmospheric conditions and oceanic
  conditions.
\item
  Considering (for now) only light intensity, why is primary
  productivity higher at roughly 20m depth than at the surface?
 \end{enumerate}


\end{document}