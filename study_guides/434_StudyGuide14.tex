%!TEX TS-program = lualatex
%!TEX encoding = UTF-8 Unicode

\documentclass[letterpaper]{tufte-handout}

%\geometry{showframe} % display margins for debugging page layout

\usepackage{fontspec}
\def\mainfont{Linux Libertine O}
\setmainfont[Ligatures={Common,TeX}, Contextuals={NoAlternate}, BoldFont={* Bold}, ItalicFont={* Italic}, Numbers={OldStyle}]{\mainfont}
\setsansfont[Scale=MatchLowercase, Numbers={OldStyle}]{Linux Biolinum O} 
\setmonofont{Linux Libertine Mono O}
\usepackage{microtype}

\usepackage{graphicx} % allow embedded images
  \setkeys{Gin}{width=\linewidth,totalheight=\textheight,keepaspectratio}
  \graphicspath{{img/}} % set of paths to search for images
\usepackage{amsmath}  % extended mathematics
\usepackage{booktabs} % book-quality tables
\usepackage{units}    % non-stacked fractions and better unit spacing
\usepackage{multicol} % multiple column layout facilities
%\usepackage{fancyvrb} % extended verbatim environments
%  \fvset{fontsize=\normalsize}% default font size for fancy-verbatim environments

\usepackage{enumitem}
\usepackage{mhchem}

\makeatletter
% Paragraph indentation and separation for normal text
\renewcommand{\@tufte@reset@par}{%
  \setlength{\RaggedRightParindent}{1.0pc}%
  \setlength{\JustifyingParindent}{1.0pc}%
  \setlength{\parindent}{1pc}%
  \setlength{\parskip}{0pt}%
}
\@tufte@reset@par

% Paragraph indentation and separation for marginal text
\renewcommand{\@tufte@margin@par}{%
  \setlength{\RaggedRightParindent}{0pt}%
  \setlength{\JustifyingParindent}{0.5pc}%
  \setlength{\parindent}{0.5pc}%
  \setlength{\parskip}{0pt}%
}
\makeatother

% Set up the spacing using fontspec features
   \renewcommand\allcapsspacing[1]{{\addfontfeatures{LetterSpace=15}#1}}
   \renewcommand\smallcapsspacing[1]{{\addfontfeatures{LetterSpace=10}#1}}

\title{{\scshape bi} 434 Study Guide 14}
\author{Ocean acidification}
\date{} % without \date command, current date is supplied

\begin{document}

\maketitle	% this prints the handout title, author, and date

%\printclassoptions
%section*{Kelp forest ecosystems}

We\marginnote{\textbf{Read:} Chapter 19, pages 428--430.} covered ocean acidification in greater depth than your text because I think this topic is important.

\section*{Vocabulary}
\begin{multicols}{2}
ocean acidification \\
pH \\
bicarbonate buffering system \\
\ce{HCO3^{-}} (bicarbonate) \\
\ce{CO3^{2-}} (carbonate) \\
aragonite \\
$\Omega$ \\
\end{multicols}
\section*{Concepts}

\begin{enumerate}

\item Recognize and write equilibrium equation for the bicarbonate buffering system. Explain how it works to buffer pH.

\item Explain how addition of \ce{CO2} to the atmosphere increases ocean acidity.

\item In the bicarbonate buffering equation, identify the course of {H+} ions (protons) most responsible for ocean acidification.
 
\item What value of $\Omega$ indicates the water is saturated with aragonite? What is the optimum value of $\Omega$ for marine organisms that produce calcium carbonate structures?

\item Identify the source of carbonate for the carbonate buffering equation.

\item How have aragonite levels changed in the ocean since the 1700s. Explain how this change relates to pH change in the oceans.

\item Tie together all of the above questions.\marginnote{See also Fig.~19.3B, page 431 for a diagrammatic representation of the process.} Explain how addition of atmospheric \ce{CO2} is causing the loss of aragonite in the oceans.

\item Explain how ocean acidification may cause a phase shift for coral reefs.



\end{enumerate}

\end{document}