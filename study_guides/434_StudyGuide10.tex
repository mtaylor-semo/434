%!TEX TS-program = lualatex
%!TEX encoding = UTF-8 Unicode

\documentclass[letterpaper]{tufte-handout}

%\geometry{showframe} % display margins for debugging page layout

\usepackage{fontspec}
\def\mainfont{Linux Libertine O}
\setmainfont[Ligatures={Common,TeX}, Contextuals={NoAlternate}, BoldFont={* Bold}, ItalicFont={* Italic}, Numbers={OldStyle}]{\mainfont}
\setsansfont[Scale=MatchLowercase, Numbers={OldStyle}]{Linux Biolinum O} 
\setmonofont{Linux Libertine Mono O}
\usepackage{microtype}

\usepackage{graphicx} % allow embedded images
  \setkeys{Gin}{width=\linewidth,totalheight=\textheight,keepaspectratio}
  \graphicspath{{img/}} % set of paths to search for images
\usepackage{amsmath}  % extended mathematics
\usepackage{booktabs} % book-quality tables
\usepackage{units}    % non-stacked fractions and better unit spacing
\usepackage{multicol} % multiple column layout facilities
%\usepackage{fancyvrb} % extended verbatim environments
%  \fvset{fontsize=\normalsize}% default font size for fancy-verbatim environments

\usepackage{enumitem}
%\usepackage{mhchem}

\makeatletter
% Paragraph indentation and separation for normal text
\renewcommand{\@tufte@reset@par}{%
  \setlength{\RaggedRightParindent}{1.0pc}%
  \setlength{\JustifyingParindent}{1.0pc}%
  \setlength{\parindent}{1pc}%
  \setlength{\parskip}{0pt}%
}
\@tufte@reset@par

% Paragraph indentation and separation for marginal text
\renewcommand{\@tufte@margin@par}{%
  \setlength{\RaggedRightParindent}{0pt}%
  \setlength{\JustifyingParindent}{0.5pc}%
  \setlength{\parindent}{0.5pc}%
  \setlength{\parskip}{0pt}%
}
\makeatother

% Set up the spacing using fontspec features
   \renewcommand\allcapsspacing[1]{{\addfontfeatures{LetterSpace=15}#1}}
   \renewcommand\smallcapsspacing[1]{{\addfontfeatures{LetterSpace=10}#1}}

\title{{\scshape bi} 434 Study Guide 10}
\author{Seagrass ecosystems}
\date{} % without \date command, current date is supplied

\begin{document}

\maketitle	% this prints the handout title, author, and date

%\printclassoptions
%section*{Kelp forest ecosystems}

We\marginnote{\textbf{Read:} Chapter 12. The figures and tables from the text that I use for lecture are given in the slides. You should at least read the text associated with those figures and tables.} covered seagrass ecosystems, emphasizing the mutualistic mesograzer model.

\section*{Vocabulary}

\vspace{-1\baselineskip}
\begin{multicols}{2}
seagrass \\
epiphytes \\
mesograzers \\
direct effects \\
indirect effects \\
mutualistic mesograzer \\ \hspace*{1em}hypothesis \\
alternate stable states
\end{multicols}

\section*{Concepts}

\begin{enumerate}

\item What are seagrasses?\marginnote{Read page 271.} From a conservation perspective, why are they important? From a biological perspective (ecosystem function, etc.), why are they important?

\item About how much gross primary productivity\marginnote{Your text does not cover this directly but I did.} comes from seagrass beds? What are the major groups of primary producers in seagrass beds?
	
\item Explain the interactions between seagrasses and other epiphytes. Is it good to have both in the ecosystem? Explain.	

\item
  What are mesograzers? How do they regulate seagrass ecosystem function? 
  
\item Explain the mutualistic mesograzer hypothesis.\marginnote{See pages 280--284.} What does ``mutualisitic'' mean with regard to this hypothesis? Where are direct interactions important in this hypothesis? Where are indirect interactions important?

\item
  Compare and contrast the type of cascade associated with the mutualistic mesograzer hypothesis to a regular trophic cascade (density-mediated indirect interactions; e.g., sea otters and urchins) and to a \textsc{tmii} cascade. How are they similar? How are they different? For all three cascade types, identify the direct and indirect interactions. Explain why they are considered direct or indirect.

\item
  Is the mutualistic mesograzer hypothesis a bottom-up or top-down hypotheses?\marginnote{Fig. 12.5 and page 280.} Explain. Is the other type of process (the one you did not choose) also present in seagrass ecosystems? Explain.
  
\item Describe the many types of interactions illustrated by Fig. 12.6\marginnote{See pages 280--282.} Describe how seagrass abundance increases or decrease depending on whether the high level shown (not grayed out) increases or decreases. State whether the changes happen through direct or indirect interactions.

\item Do alternate stable states exist in seagrass ecosystems?  If so, describe the differences and explain what causes the change between states.

\end{enumerate}

\end{document}