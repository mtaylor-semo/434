%!TEX TS-program = lualatex
%!TEX encoding = UTF-8 Unicode

\documentclass[letterpaper]{tufte-handout}

%\geometry{showframe} % display margins for debugging page layout

\usepackage{fontspec}
\def\mainfont{Linux Libertine O}
\setmainfont[Ligatures={Common,TeX}, Contextuals={NoAlternate}, BoldFont={* Bold}, ItalicFont={* Italic}, Numbers={OldStyle}]{\mainfont}
\setsansfont[Scale=MatchLowercase, Numbers={OldStyle}]{Linux Biolinum O} 
\setmonofont{Linux Libertine O}
\usepackage{microtype}

\usepackage{graphicx} % allow embedded images
  \setkeys{Gin}{width=\linewidth,totalheight=\textheight,keepaspectratio}
  \graphicspath{{/Users/goby/Documents/teach/434/lectures/}} % set of paths to search for images
\usepackage{amsmath}  % extended mathematics
\usepackage{booktabs} % book-quality tables
\usepackage{units}    % non-stacked fractions and better unit spacing
\usepackage{multicol} % multiple column layout facilities
%\usepackage{fancyvrb} % extended verbatim environments
%  \fvset{fontsize=\normalsize}% default font size for fancy-verbatim environments

\usepackage{enumitem}
\usepackage{mhchem}

\makeatletter
% Paragraph indentation and separation for normal text
\renewcommand{\@tufte@reset@par}{%
  \setlength{\RaggedRightParindent}{1.0pc}%
  \setlength{\JustifyingParindent}{1.0pc}%
  \setlength{\parindent}{1pc}%
  \setlength{\parskip}{0pt}%
}
\@tufte@reset@par

% Paragraph indentation and separation for marginal text
\renewcommand{\@tufte@margin@par}{%
  \setlength{\RaggedRightParindent}{0pt}%
  \setlength{\JustifyingParindent}{0.5pc}%
  \setlength{\parindent}{0.5pc}%
  \setlength{\parskip}{0pt}%
}
\makeatother

% Set up the spacing using fontspec features
   \renewcommand\allcapsspacing[1]{{\addfontfeatures{LetterSpace=15}#1}}
   \renewcommand\smallcapsspacing[1]{{\addfontfeatures{LetterSpace=10}#1}}

\newcommand\lecturefile{434_lecture03_instructor}

\title{{\scshape bi} 434/634 Study Guide 03}

\date{} % without \date command, current date is supplied

\begin{document}

\maketitle	% this prints the handout title, author, and date

%\printclassoptions
\section*{Ecosystem-based Conservation and Management}

We\marginnote{\textbf{Read:} Chapter 21. Any figures and tables from the text that I use for lecture are numbered in the slides. You should at least read the text associated with those figures and tables.} covered the requirements and considerations for taking an ecosystem-based approach to marine conservation and management.

\section*{Vocabulary}

\vspace{-1\baselineskip}
%\begin{multicols}{2}
\vspace*{\baselineskip}
No new vocabulary was introduced. \\
%\end{multicols}

\section*{Concepts}

\begin{enumerate}

	\item Ecosystem-based approaches to conservation and management requires
	
	\begin{enumerate}
		\item goals encompass all ecosystem services,
		\item the spatial scale is based on natural boundaries across multiple ecosystems, 

	\begin{marginfigure}
		\includegraphics[page=5]{\lecturefile}
	\end{marginfigure}
		
		\item all sectors of human use are integrated,
		\item cumulative effects across sectors are estimated, and
		\item strategies adapt over time to account for uncertainty.
	\end{enumerate}

	Write a one paragraph summary of each item that explains the meaning and why it is important for conservation and management. Do not just think it in your head\dots write it out.
		
\end{enumerate}

\end{document}