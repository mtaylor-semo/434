%!TEX TS-program = lualatex
%!TEX encoding = UTF-8 Unicode

\documentclass[letterpaper]{tufte-handout}

%\geometry{showframe} % display margins for debugging page layout

\usepackage{fontspec}
\def\mainfont{Linux Libertine O}
\setmainfont[Ligatures={Common,TeX}, Contextuals={NoAlternate}, BoldFont={* Bold}, ItalicFont={* Italic}, Numbers={OldStyle}]{\mainfont}
\setsansfont[Scale=MatchLowercase, Numbers={OldStyle}]{Linux Biolinum O} 
\setmonofont{Linux Libertine Mono O}
\usepackage{microtype}

\usepackage{graphicx} % allow embedded images
  \setkeys{Gin}{width=\linewidth,totalheight=\textheight,keepaspectratio}
  \graphicspath{{img/}} % set of paths to search for images
\usepackage{amsmath}  % extended mathematics
\usepackage{booktabs} % book-quality tables
\usepackage{units}    % non-stacked fractions and better unit spacing
\usepackage{multicol} % multiple column layout facilities
%\usepackage{fancyvrb} % extended verbatim environments
%  \fvset{fontsize=\normalsize}% default font size for fancy-verbatim environments

\usepackage{enumitem}
%\usepackage{mhchem}

\makeatletter
% Paragraph indentation and separation for normal text
\renewcommand{\@tufte@reset@par}{%
  \setlength{\RaggedRightParindent}{1.0pc}%
  \setlength{\JustifyingParindent}{1.0pc}%
  \setlength{\parindent}{1pc}%
  \setlength{\parskip}{0pt}%
}
\@tufte@reset@par

% Paragraph indentation and separation for marginal text
\renewcommand{\@tufte@margin@par}{%
  \setlength{\RaggedRightParindent}{0pt}%
  \setlength{\JustifyingParindent}{0.5pc}%
  \setlength{\parindent}{0.5pc}%
  \setlength{\parskip}{0pt}%
}
\makeatother

% Set up the spacing using fontspec features
   \renewcommand\allcapsspacing[1]{{\addfontfeatures{LetterSpace=15}#1}}
   \renewcommand\smallcapsspacing[1]{{\addfontfeatures{LetterSpace=10}#1}}

\title{{\scshape bi} 434 Study Guide 11}
\author{Hydrothermal vent ecosystems}
\date{} % without \date command, current date is supplied

\begin{document}

\maketitle	% this prints the handout title, author, and date

%\printclassoptions
%section*{Kelp forest ecosystems}

We\marginnote{\textbf{Read:} Chapter 17. The figures and tables from the text that I use for lecture are given in the slides. I may give pop quizzes based on the assigned reading.} covered hydrothermal vent ecosystems, emphasizing connectivity among vent systems.

\section*{Vocabulary}
\begin{multicols}{2}
hydrothermal vents\\
oceanic ridges\\
chemosynthesis\\
symbiosis\\
foundation species \\
population connectivity\\
\end{multicols}
\section*{Concepts}

\begin{enumerate}

\item What are hydrothermal vents?\marginnote{Vent systems are very dynamic. They form sporadically along ocean ridges but may last for only 200--300 years.} What is the source of energy for this ecosystem?  How does this energy enter into the organisms living in this ecosystem?

\item Do vent systems on different ocean ridge systems have the same or different foundation species? What about vent systems within the same ridge system?

\item Is zonation present in a hydrothermal vent ecosystem? If so, explain the pattern and the cause of the zonation gradient. If not, explain the processes that prevent the formation of zonation.

\item How does larval transport explain maintenance of vent systems within ridges?

\item Does larval transport occur among different ocean ridges? Explain how water currents facilitate or prohibit transport among ocean ridge systems. 

\item Does larval transport occur among all vent systems along entire ocean ridge systems? Explain how water currents facilitate or prohibit transport within ocean ridge systems. 

\item Do hydrothermal vents display evidence of panmixis? Explain why or why not.

\item Describe the pattern of genetic differentiation among three vent systems across the East Pacific Rise ridge system. How does larval transport explain this pattern?

\item Why are vent systems considered metapulations? How does metapopulation dynamics relate to larval transport among vent systems?\marginnote{Carefully study pages 393--396. I may not cover all of this in lecture.}

\end{enumerate}

\end{document}