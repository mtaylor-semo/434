%!TEX TS-program = lualatex
%!TEX encoding = UTF-8 Unicode

\documentclass[letterpaper]{tufte-handout}

%\geometry{showframe} % display margins for debugging page layout

\usepackage{fontspec}
\def\mainfont{Linux Libertine O}
\setmainfont[Ligatures={Common,TeX}, Contextuals={NoAlternate}, BoldFont={* Bold}, ItalicFont={* Italic}, Numbers={OldStyle}]{\mainfont}
\setsansfont[Scale=MatchLowercase, Numbers={OldStyle}]{Linux Biolinum O} 
\setmonofont{Linux Libertine Mono O}
\usepackage{microtype}

\usepackage{graphicx} % allow embedded images
  \setkeys{Gin}{width=\linewidth,totalheight=\textheight,keepaspectratio}
  \graphicspath{{img/}} % set of paths to search for images
\usepackage{amsmath}  % extended mathematics
\usepackage{booktabs} % book-quality tables
\usepackage{units}    % non-stacked fractions and better unit spacing
\usepackage{multicol} % multiple column layout facilities
%\usepackage{fancyvrb} % extended verbatim environments
%  \fvset{fontsize=\normalsize}% default font size for fancy-verbatim environments

\usepackage{enumitem}
\usepackage{mhchem}

\makeatletter
% Paragraph indentation and separation for normal text
\renewcommand{\@tufte@reset@par}{%
  \setlength{\RaggedRightParindent}{1.0pc}%
  \setlength{\JustifyingParindent}{1.0pc}%
  \setlength{\parindent}{1pc}%
  \setlength{\parskip}{0pt}%
}
\@tufte@reset@par

% Paragraph indentation and separation for marginal text
\renewcommand{\@tufte@margin@par}{%
  \setlength{\RaggedRightParindent}{0pt}%
  \setlength{\JustifyingParindent}{0.5pc}%
  \setlength{\parindent}{0.5pc}%
  \setlength{\parskip}{0pt}%
}
\makeatother

% Set up the spacing using fontspec features
   \renewcommand\allcapsspacing[1]{{\addfontfeatures{LetterSpace=15}#1}}
   \renewcommand\smallcapsspacing[1]{{\addfontfeatures{LetterSpace=10}#1}}

\title{{\scshape bi} 434 study guide 15}
\author{Ocean warming}
\date{} % without \date command, current date is supplied

\begin{document}

\maketitle	% this prints the handout title, author, and date

%\printclassoptions
%section*{Kelp forest ecosystems}

We\marginnote{\textbf{Read:} Chapter 19.} covered ocean warming and effects on biology of marine organisms.

\section*{Vocabulary}
\begin{multicols}{2}
climate change \\
phenology \\
thermohaline circulation \\
biodiversity hotspot
\end{multicols}
\section*{Concepts}

\begin{enumerate}

\item\label{uneven_warming} Ocean warming does not occur uniformly across oceans.\marginnote{Think broadly about the consequences.} Where is warming occurring most rapidly? What are the consequences of this?

\item How does warming affect phenology\marginnote{See Fig.~19.10, and read pages 437--438.} and the distribution of marine species?

\item How does warming affect phytoplankton community structure?\marginnote{This would be a good time to review your notes and study guide for phytoplankton communities (Lecture 07).} How could this affect primary production in the Gulf of Maine/ Explain.

\item How might warming affect <insert ecosystem>? We covered an example of warming for each ecosystem that is being covered by a management plan. You should think of specific ways in which warming (and acidification) for the ecosystem that was the subject of your specific management plan.

\item How might larval duration and survivorship be affected by warming? Explain.\marginnote{Hint: See question~\ref{uneven_warming}.} Could this increase the \emph{potential} for connectivity between the north Atlantic and north Pacific oceans?

\item Explain how warming might affect sea levels. Explain how this could affect ecosystem services for different ecosystems?

\item Explain how warming might affect thermohaline circulation (the great ocean conveyor).

\item If warming causes loss of coral diversity\marginnote{See also Fig.~19.12.} , will other organisms (or at least fishes) on all coral reef ecosystems all be affected equally? That is, does current non-coral diversity alter how warming might affect non-coral diversity on coral reefs?

\item Are ecosystem functions affected equally by acidification and warming? That is, do acidification and warming each contribute equally to change of ecosystem function? Explain.

\end{enumerate}

\end{document}