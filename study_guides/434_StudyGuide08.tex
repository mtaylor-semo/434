%!TEX TS-program = lualatex
%!TEX encoding = UTF-8 Unicode

\documentclass[letterpaper]{tufte-handout}

%\geometry{showframe} % display margins for debugging page layout

\usepackage{fontspec}
\def\mainfont{Linux Libertine O}
\setmainfont[Ligatures={Common,TeX}, Contextuals={NoAlternate}, BoldFont={* Bold}, ItalicFont={* Italic}, Numbers={OldStyle}]{\mainfont}
\setsansfont[Scale=MatchLowercase, Numbers={OldStyle}]{Linux Biolinum O} 
\setmonofont{Linux Libertine Mono O}
\usepackage{microtype}

\usepackage{graphicx} % allow embedded images
  \setkeys{Gin}{width=\linewidth,totalheight=\textheight,keepaspectratio}
  \graphicspath{{img/}} % set of paths to search for images
\usepackage{amsmath}  % extended mathematics
\usepackage{booktabs} % book-quality tables
\usepackage{units}    % non-stacked fractions and better unit spacing
\usepackage{multicol} % multiple column layout facilities
%\usepackage{fancyvrb} % extended verbatim environments
%  \fvset{fontsize=\normalsize}% default font size for fancy-verbatim environments

\usepackage{enumitem}
\usepackage{mhchem}

\makeatletter
% Paragraph indentation and separation for normal text
\renewcommand{\@tufte@reset@par}{%
  \setlength{\RaggedRightParindent}{1.0pc}%
  \setlength{\JustifyingParindent}{1.0pc}%
  \setlength{\parindent}{1pc}%
  \setlength{\parskip}{0pt}%
}
\@tufte@reset@par

% Paragraph indentation and separation for marginal text
\renewcommand{\@tufte@margin@par}{%
  \setlength{\RaggedRightParindent}{0pt}%
  \setlength{\JustifyingParindent}{0.5pc}%
  \setlength{\parindent}{0.5pc}%
  \setlength{\parskip}{0pt}%
}
\makeatother

% Set up the spacing using fontspec features
   \renewcommand\allcapsspacing[1]{{\addfontfeatures{LetterSpace=15}#1}}
   \renewcommand\smallcapsspacing[1]{{\addfontfeatures{LetterSpace=10}#1}}

\title{{\scshape bi} 434 Study Guide 08}

\date{} % without \date command, current date is supplied

\begin{document}

\maketitle	% this prints the handout title, author, and date

%\printclassoptions
\section*{Rocky intertidal ecosystems}

We\marginnote{\textbf{Read:} Chapter 9. The figures and tables from the text that I use for lecture are given in the slides. You should at least read the text associated with those figures and tables.} covered community structure and function in the rocky intertidal zone.

\section*{Vocabulary}

\vspace{-1\baselineskip}
\begin{multicols}{2}
bottom-up process \\
top-down process \\
intermittent upwelling \\
thermal stress \\
alternate stable states \\
consumptive predator effect \\
non-consumptive predator effect \\
induced defense \\
trait-mediated indirect\\ \hspace*{0.5em} interaction \\
trait-mediated cascade \\
trophic heat 
\end{multicols}

\section*{Concepts}

\begin{enumerate}
	
	\item Explain the difference between top-down and bottom-up processes.  Is the intermittent upwelling hypothesis a bottom-up process, a top-down process, or both? Explain.
	
	\item While you are at it, explain\marginnote{The \textsc{y}-axis of Figure 9.3, page 207, refers to ecological processes, which is the same as ecosystem functions. You should be able to interpret both panels of Fig. 9.3.} the intermittent upwelling hypothesis. Why does it increase ecosystem function?
	
	\item Scientists long predicted that intertidal species at lower latitudes would be subjected to greater thermal stress\marginnote{Figure 9.5} (rapidity and degree of temperature change) than intertidal species living at higher latitudes. Studies suggest this hypothesis is false. Why? 
	
	\item Explain alternative stable states. Is there evidence that alternate stable states exist in the Gulf of Maine? At what scale? Explain.
	
	\item Explain the difference between consumptive and non-consumptive predator effects.
	
	\item Explain what is an inducible defense. Explain \marginnote{Figure 9.9 and pages 211--212.} each of the two ways that inducible defenses can appear in organisms.
	
	\item Explain what is a trait-mediated cascade?\marginnote{Figure 9.11, page 213.} Diagram a trait-mediated cascade. Explain the relationship between inducible defenses and trait-mediated cascades.

	\item Explain trophic heat.\marginnote{Figure 9.13, page 214.} Explain why do rocky intertidal food chains/webs tend to be short, in relation to trophic heat?

	\item Turbulence can affect the ability of mobile organisms to find prey. It may also explain the amount of trophic heat in a population. Explain whether increased turbulence is likely to increase or decrease the amount of trophic heat in the rocky intertidal ecosystem and why.
	
	\item Describe how feeding rate in \textit{Pisaster} sea stars might be affected by climate change. 
	
\end{enumerate}

\end{document}