%!TEX TS-program = lualatex
%!TEX encoding = UTF-8 Unicode

\documentclass[letterpaper]{tufte-handout}

%\geometry{showframe} % display margins for debugging page layout

\usepackage{fontspec}
\def\mainfont{Linux Libertine O}
\setmainfont[Ligatures={Common,TeX}, Contextuals={NoAlternate}, BoldFont={* Bold}, ItalicFont={* Italic}, Numbers={OldStyle}]{\mainfont}
\setsansfont[Scale=MatchLowercase, Numbers={OldStyle}]{Linux Biolinum O} 
\setmonofont{Linux Libertine O}
\usepackage{microtype}

\usepackage{graphicx} % allow embedded images
  \setkeys{Gin}{width=\linewidth,totalheight=\textheight,keepaspectratio}
  \graphicspath{{/Users/goby/Documents/teach/434/lectures/}} % set of paths to search for images
\usepackage{amsmath}  % extended mathematics
\usepackage{booktabs} % book-quality tables
\usepackage{units}    % non-stacked fractions and better unit spacing
\usepackage{multicol} % multiple column layout facilities
%\usepackage{fancyvrb} % extended verbatim environments
%  \fvset{fontsize=\normalsize}% default font size for fancy-verbatim environments

\usepackage{enumitem}
\usepackage{mhchem}

\makeatletter
% Paragraph indentation and separation for normal text
\renewcommand{\@tufte@reset@par}{%
  \setlength{\RaggedRightParindent}{1.0pc}%
  \setlength{\JustifyingParindent}{1.0pc}%
  \setlength{\parindent}{1pc}%
  \setlength{\parskip}{0pt}%
}
\@tufte@reset@par

% Paragraph indentation and separation for marginal text
\renewcommand{\@tufte@margin@par}{%
  \setlength{\RaggedRightParindent}{0pt}%
  \setlength{\JustifyingParindent}{0.5pc}%
  \setlength{\parindent}{0.5pc}%
  \setlength{\parskip}{0pt}%
}
\makeatother

% Set up the spacing using fontspec features
   \renewcommand\allcapsspacing[1]{{\addfontfeatures{LetterSpace=15}#1}}
   \renewcommand\smallcapsspacing[1]{{\addfontfeatures{LetterSpace=10}#1}}

\newcommand\lecturefile{434_lecture04_instructor}

\title{{\scshape bi} 434/634 Study Guide 04}

\date{} % without \date command, current date is supplied

\begin{document}

\maketitle	% this prints the handout title, author, and date

%\printclassoptions
\section*{Marine Restoration Ecology}

We\marginnote{\textbf{Read:} Chapter 22. Any figures and tables from the text that I use for lecture are numbered in the slides. You should at least read the text associated with those figures and tables.} covered the requirements and considerations for marine restoration ecology.

\section*{Vocabulary}

\vspace{-1\baselineskip}
\begin{multicols}{2}
restoration ecology \\
ecosystem structure \\
ecosystem function \\
population-level approach \\
habitat-level approach \\
landscape-level approach \\
ecosystem-level approach \\
\end{multicols}

\section*{Concepts}

\begin{enumerate}

	\item Explain the relationship between ecosystem structure and ecosystem function.
	
	\item Explain the differences and relationships between ecosystem services and ecosystem structure and function.
	
	\item Explain what is restoration ecology. 
	
	\item Does restoration ecology focus on ecosystem services, ecosystem structure and function, or both? Explain.
	
	\item Compare and contrast the four possible approaches for restoration of troubled or threatened marine resources.
	
	\item Why\marginnote{See Fig.~22.1} is it important to compare historical and current biological knowledge and also historical and current bottlenecks? 
	
	\item Why should restoration begin first at a small scale?
	
	\item Why\marginnote{See Fig.~22.2} should facilitation of foundation species be considered for restoration?
	
%	\begin{marginfigure}
%		\includegraphics[page=5]{\lecturefile}
%	\end{marginfigure}
		
	
\end{enumerate}

\end{document}