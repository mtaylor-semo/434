%!TEX TS-program = lualatex
%!TEX encoding = UTF-8 Unicode

\documentclass[letterpaper]{tufte-handout}

%\geometry{showframe} % display margins for debugging page layout

\usepackage{fontspec}
\def\mainfont{Linux Libertine O}
\setmainfont[Ligatures={Common,TeX}, Contextuals={NoAlternate}, BoldFont={* Bold}, ItalicFont={* Italic}, Numbers={OldStyle}]{\mainfont}
\setsansfont[Scale=MatchLowercase, Numbers={OldStyle}]{Linux Biolinum O} 
\setmonofont{Linux Libertine O}
\usepackage{microtype}

\usepackage{graphicx} % allow embedded images
  \setkeys{Gin}{width=\linewidth,totalheight=\textheight,keepaspectratio}
  \graphicspath{{/Users/goby/Documents/teach/434/lectures/}} % set of paths to search for images

\usepackage{amsmath}  % extended mathematics
\usepackage{booktabs} % book-quality tables
\usepackage{units}    % non-stacked fractions and better unit spacing
\usepackage{multicol} % multiple column layout facilities
%\usepackage{fancyvrb} % extended verbatim environments
%  \fvset{fontsize=\normalsize}% default font size for fancy-verbatim environments

\usepackage{enumitem}
\usepackage{mhchem}

\makeatletter
% Paragraph indentation and separation for normal text
\renewcommand{\@tufte@reset@par}{%
  \setlength{\RaggedRightParindent}{1.0pc}%
  \setlength{\JustifyingParindent}{1.0pc}%
  \setlength{\parindent}{1pc}%
  \setlength{\parskip}{0pt}%
}
\@tufte@reset@par

% Paragraph indentation and separation for marginal text
\renewcommand{\@tufte@margin@par}{%
  \setlength{\RaggedRightParindent}{0pt}%
  \setlength{\JustifyingParindent}{0.5pc}%
  \setlength{\parindent}{0.5pc}%
  \setlength{\parskip}{0pt}%
}
\makeatother

% Set up the spacing using fontspec features
   \renewcommand\allcapsspacing[1]{{\addfontfeatures{LetterSpace=15}#1}}
   \renewcommand\smallcapsspacing[1]{{\addfontfeatures{LetterSpace=10}#1}}

\title{{\scshape bi} 434 Study Guides 01 and 02}

\date{} % without \date command, current date is supplied

\newcommand\lecturefileA{434_lecture01_instructor}
\newcommand\lecturefileB{434_lecture02_instructor}

\begin{document}

\maketitle	% this prints the handout title, author, and date

%\printclassoptions
\section*{Marine Ecosystem Services}

We\marginnote{\textbf{Read:} Chapter 18. The figures and tables from the text that I use for lecture are given in the slides. You should at least read the text associated with those figures and tables.} covered how dispersal of planktonic larvae can affect connectivity among marine populations.

\section*{Vocabulary}

\vspace{-1\baselineskip}
\begin{multicols}{2}
ecosystem service \\
ecosystem structure \\
ecosystem function \\
trend \\
pressure \\
resilience
\end{multicols}

\section*{Concepts}

\begin{enumerate}

	\item What are ecosystem services?%
	\marginnote{Read the four detailed examples on pages 411--417.}
	List a variety (at least 5–6) of services 
	provided by different ecosystems. Challenge your friend so see who can come 
	with more ecosystem services for a given ecosystem (e.g., coral reefs, sea
	grass beds, salt marshes, etc.)
	
	\item What is ecosystem structure? What is ecosystem function?
	
	\item Explain the difference between ecosystem function and ecosystem service.
	What does it mean to say that ecosystem functions are the source of 
	ecosystem services but that they are not the same thing?
	
	\item Explain%
	\marginnote{Fig. 18.1}
	how humans placing a value on a marine ecosystem can improve how 
	humans perceive that ecosystem and can improve the services provided by that
	ecosystem. How do negative drivers affect values? What about positive drivers?
	
	\item Look at the examples in Table~18.1 (page~405). Consider each example
	in terms of the valuation loop shown in Figure~18.1. For example, how
	can the ecosystem function of barrier islands (first entry in the table)
	alter human drivers to increase the service of coastal protection. Try this
	for several or all of the entries in Table~18.1.
	
	\item In terms of ecosystem services, what are trade-offs? Why must trade-offs 
	be assessed when trying to quantify services provided by a marine ecosystem?
	
	

	\item Why%
	\marginnote{Fig.~18.4}
	should the cumulative effects of positive and negative impacts be
	estimated when trying to quantify ecosystem services?
	
	\item Explain%
	\marginnote{Fig.~18.5; the image below is the original version.}
	the conceptual framework for calculating the Ocean Health Index. You do not
	have to be able to calculate an index but you must understand what goes in
	to calculating it.

	\begin{marginfigure}
	\includegraphics[page=14]{\lecturefileB}
	\end{marginfigure}
	
	\item For the Ocean Health Index, explain the difference between Pressure and
	Resilience and how each can affect the present state of the ecosystem.
	
	\item How can a measure like the Ocean Health Index (\textsc{ohi}) help 
	conservation managers establish and measure
	the effectiveness of ecosystem management? Do not just state the obvious that 
	the goal should be to increase the value of the \textsc{ohi}. How would
	you set about increasing the index value? \textsc{Hint:}~What are the goals
	that make up the \textsc{ohi}? Will each of these goals be the same for
	a given ecosystem or the same for different countries? Which goals should
	you try to improve first? Explain.
	
	\item Name and explain some of the land-based drivers that should be considered
	when planning for or addressing threats to marine ecosystems.
	
\end{enumerate}

\end{document}