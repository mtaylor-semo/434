%!TEX TS-program = lualatex
%!TEX encoding = UTF-8 Unicode

\documentclass[letterpaper]{tufte-handout}

%\geometry{showframe} % display margins for debugging page layout

\usepackage{fontspec}
\def\mainfont{Linux Libertine O}
\setmainfont[Ligatures={Common,TeX}, Contextuals={NoAlternate}, BoldFont={* Bold}, ItalicFont={* Italic}, Numbers={OldStyle}]{\mainfont}
\setsansfont[Scale=MatchLowercase, Numbers={OldStyle}]{Linux Biolinum O} 
\setmonofont{Linux Libertine Mono O}
\usepackage{microtype}

\usepackage{graphicx} % allow embedded images
  \setkeys{Gin}{width=\linewidth,totalheight=\textheight,keepaspectratio}
  \graphicspath{{img/}} % set of paths to search for images
\usepackage{amsmath}  % extended mathematics
\usepackage{booktabs} % book-quality tables
\usepackage{units}    % non-stacked fractions and better unit spacing
\usepackage{multicol} % multiple column layout facilities
%\usepackage{fancyvrb} % extended verbatim environments
%  \fvset{fontsize=\normalsize}% default font size for fancy-verbatim environments

\usepackage{enumitem}
%\usepackage{mhchem}

\makeatletter
% Paragraph indentation and separation for normal text
\renewcommand{\@tufte@reset@par}{%
  \setlength{\RaggedRightParindent}{1.0pc}%
  \setlength{\JustifyingParindent}{1.0pc}%
  \setlength{\parindent}{1pc}%
  \setlength{\parskip}{0pt}%
}
\@tufte@reset@par

% Paragraph indentation and separation for marginal text
\renewcommand{\@tufte@margin@par}{%
  \setlength{\RaggedRightParindent}{0pt}%
  \setlength{\JustifyingParindent}{0.5pc}%
  \setlength{\parindent}{0.5pc}%
  \setlength{\parskip}{0pt}%
}
\makeatother

% Set up the spacing using fontspec features
   \renewcommand\allcapsspacing[1]{{\addfontfeatures{LetterSpace=15}#1}}
   \renewcommand\smallcapsspacing[1]{{\addfontfeatures{LetterSpace=10}#1}}

\title{{\scshape bi} 434 Study Guide 11}
\author{Salt marsh ecosystems}
\date{} % without \date command, current date is supplied

\begin{document}

\maketitle	% this prints the handout title, author, and date

%\printclassoptions
%section*{Kelp forest ecosystems}

We\marginnote{\textbf{Read:} Chapter 11. The figures and tables from the text that I use for lecture are given in the slides. You should at least read the text associated with those figures and tables.} covered salt marsh ecosystems, emphasizing trophic cascades (especially those driven by crabs) and trait-mediated indirect interactions.

\section*{Vocabulary}

No new vocabulary was introduced.

\section*{Concepts}

\begin{enumerate}

\item What is a salt marsh?\marginnote{Review pages 252--257 for information covered in \textsc{bi}\,348. }

\item What evidence\marginnote{Ecologists long thought that salt marshes were regulated entirely by bottom-up processes.} showed that salt marshes are controlled by top-down processes?
	
\item Describe the trophic cascade caused by crabs and turtle predators. \marginnote{See Fig. 11.7.}  How do alligators modify this cascade and how do they affect the abundance of \textit{Spartina}?

\item
  How does fishing affect this trophic cascade?\marginnote{See Fig. 11.9. For fun, what is wrong with that figure?} 
  
\item Explain how functional redundancy of crab predators affects salt marsh ecosystem function.

\item
  Do crabs always have a positive effect on \textit{Spartina} through the trophic cascade? Explain. Is this a density-mediated or trait-mediated interaction?

\item
  What evidence\marginnote{The study included a snail, a crab, and a fungus. These are the top three consumers in many salt marshes.} suggests that high functional diversity increased ecosystem function? Explain.

\end{enumerate}

\end{document}