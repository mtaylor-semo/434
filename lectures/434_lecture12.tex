%!TEX TS-program = lualatex
%!TEX encoding = UTF-8 Unicode

\documentclass[t]{beamer}

%%%% HANDOUTS For online Uncomment the following four lines for handout
%\documentclass[t,handout]{beamer}  %Use this for handouts.
%\usepackage{handoutWithNotes}
%\pgfpagesuselayout{3 on 1 with notes}[letterpaper,border shrink=5mm]

%\includeonlylecture{student}

%%% Including only some slides for students.
%%% Uncomment the following line. For the slides,
%%% use the labels shown below the command.

%% For students, use \lecture{student}{student}
%% For mine, use \lecture{instructor}{instructor}


%\usepackage{pgf,pgfpages}
%\pgfpagesuselayout{4 on 1}[letterpaper,border shrink=5mm]

% FONTS
\usepackage{fontspec}
\def\mainfont{Linux Biolinum O}
\setmainfont[Ligatures={Common,TeX}, Contextuals={NoAlternate}, BoldFont={* Bold}, ItalicFont={* Italic}, Numbers={OldStyle}]{\mainfont}
\setsansfont[Scale=MatchLowercase, Numbers={OldStyle}]{Linux Biolinum O} 
\usepackage{microtype}


%\addfontfeatures{Numbers=SlashedZero}

\usepackage{graphicx}
	\graphicspath{%
	{/Users/mtaylor/Pictures/teach/434/lectures/}%
	{/Users/mtaylor/Pictures/teach/348/lectures/}%
	{/Users/mtaylor/Pictures/teach/434/handouts/}%
	{/Users/mtaylor/Pictures/teach/common/}%}%
	{img/}} % set of paths to search for images

\usepackage{xcolor}

\usepackage{amsmath,amssymb}

%\usepackage{units}

%\usepackage{siunitx}
%\usepackage{booktabs}
\usepackage{multicol}
%	\setlength{\columnsep=1em}

%\usepackage{chemfig}
\usepackage[version=4]{mhchem}
\usepackage{tikz}
	\tikzstyle{every picture}+=[remember picture,overlay]

\mode<presentation>
{
  \usetheme{Lecture}
  \setbeamercovered{invisible}
  \setbeamertemplate{items}[square]
}

%\usepackage{hyperref}


\usepackage{calc} % Necessary for hidden word function.
\newcommand\HiddenWord[1]{%
	\alt<handout>{\rule{\widthof{#1}}{\fboxrule}}{#1}%
}

% Use the to temporarily set a background grid for positioning.
%\setbeamertemplate{background}[grid][step=1em]



\begin{document}

\lecture{student}{student}

{
\usebackgroundtemplate{\includegraphics[width=\paperwidth]{hydrothermal_vent_black_smoker}}
\begin{frame}[t]

	\vspace*{3\baselineskip}

	\hspace*{65mm}\hangpara\parbox{49mm}{\raggedright \large\textcolor{white}{%
	What are hydrothermal vent ecosystems?\\[\baselineskip]
How are they unique compared to the rest of the deep sea habitat?}%
}

	\vfilll

	\hfill \tiny \textcolor{white}{Ocean Networks Canada, Flickr Creative Commons.}
\end{frame}
}
%
{
\usebackgroundtemplate{\includegraphics[width=\paperwidth]{12_hydrothermal_vent_distribution}}
\begin{frame}[b]{Hydrothermal vents have a patchy distribution along oceanic ridges.}

 \tiny \hfill Fig.~17.2~\copyright\,Sinauer Associates, Inc.
\end{frame}
}
%
{
\usebackgroundtemplate{\includegraphics[width=\paperwidth]{vent_community}}
\begin{frame}[b]{\textcolor{white}{Super-heated water has reduced compounds like \ce{H2S}.}}

\tiny\hfill\textcolor{white}{Ocean Networks Canada, Flickr Creative Commons.}
\end{frame}
}
%
{
\usebackgroundtemplate{\includegraphics[width=\paperwidth]{chemosynthesis}}
\begin{frame}[b]{Bacterial chemosynthesis oxidizes reduced compounds to release energy.}

\hfill \tiny \copyright\,Pearson Education.
\end{frame}
}
%
\begin{frame}[t]{Bacteria are primary producers for vent ecosystems.}

\vspace*{-\baselineskip}

	\begin{multicols}{2}
		{\centering
		\includegraphics[height=0.82\textheight]{vent_community_mosaic}\par}

	\columnbreak

		\hangpara Free-living bacteria

		\hangpara \highlight{Symbiotic bacteria}

	\end{multicols}

\end{frame}
%
{
\usebackgroundtemplate{\includegraphics[width=\paperwidth]{12_vent_temperature_gradient}}
\begin{frame}[b]{The vent community is zoned by a water temperature gradient.}

\hfill \tiny Fig.~17.4~\copyright\,Sinauer Associates, Inc.
\end{frame}
}
%
{
\usebackgroundtemplate{\includegraphics[width=\paperwidth]{12_vent_foundation_species}}
\begin{frame}[b]

\hfill \tiny \textcolor{white}{Fig.~17.1~\copyright\,Sinauer Associates, Inc.}
\end{frame}
}
%
{
\usebackgroundtemplate{\includegraphics[width=\paperwidth]{12_vent_foundation_biogeography}}
\begin{frame}[b]{Different vent regions have different foundation species.}

\hfill \tiny Fig.~17.7~\copyright\,Sinauer Associates, Inc.
\end{frame}
}
%
%\lecture{instructor}{instructor}
%
%{
%\usebackgroundtemplate{\includegraphics[width=\paperwidth]{12_vent_foundation_biogeography_scotia}}
%\begin{frame}[b]{The Scotia Ridge in the southern ocean is dominated by Yeti Crabs.}
%
%\hfill \tiny Fig.~17.8~\copyright\,Sinauer Associates, Inc.
%\end{frame}
%}
%
%\lecture{student}{student}

{
\usebackgroundtemplate{\includegraphics[width=\paperwidth]{vent_community_dead}}
\begin{frame}[b]{\textcolor{white}{Vent communities last only a few hundred years.}}

\tiny\textcolor{white}{Ocean Networks Canada, Flickr Creative Commons.}
\end{frame}
}
%
\lecture{instructor}{instructor}

{
\usebackgroundtemplate{\includegraphics[width=\paperwidth]{12_vent_foundation_biogeography}}
\begin{frame}[b]{Does this pattern suggest panmixis within regions?}

\hfill \tiny Fig.~17.7~\copyright\,Sinauer Associates, Inc.
\end{frame}
}
%
\lecture{student}{student}

{
\usebackgroundtemplate{\includegraphics[width=\paperwidth]{12_vent_genetic_study_area}}
\begin{frame}[b]

	\vspace*{3\baselineskip}

	\hspace*{65mm}\hangpara\parbox{49mm}{\raggedright \large %
	How are vent communities maintained?\\[\baselineskip]
	How are new vents colonized?\\[\baselineskip]
	Connectivity was tested genetically across vent systems in the East Pacific Rise.}


	\vfilll

	\hfill \tiny Vrijenhoek 2010. Molecular Ecology 19: 4391.
\end{frame}
}
%


{
\usebackgroundtemplate{\includegraphics[width=\paperwidth]{12_vent_genetic_results}}
\begin{frame}[b]{Gene flow occurred within regions but not among them.}

\hfill \tiny Vrijenhoek 2010. Molecular Ecology 19: 4391.
\end{frame}
}
%
{
\usebackgroundtemplate{\includegraphics[width=\paperwidth]{12_vent_ridge_currents}}
\begin{frame}[b]{Currents transport larvae along ridges, not across.}

\hfill \tiny Fig.~17.6~\copyright\,Sinauer Associates, Inc.
\end{frame}
}
%

{
\usebackgroundtemplate{\includegraphics[width=\paperwidth]{12_vent_genetic_biogeography}}
\begin{frame}[b]{Larval dispersal occurs along ridges but not across large gaps on those ridges.}

\hfill \tiny Vrijenhoek 2010. Molecular Ecology 19: 4391.
\end{frame}
}
%

\end{document}

