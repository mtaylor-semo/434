%!TEX TS-program = lualatex
%!TEX encoding = UTF-8 Unicode

\documentclass[t]{beamer}

%%%% HANDOUTS For online Uncomment the following four lines for handout
%\documentclass[t,handout]{beamer}  %Use this for handouts.
%\usepackage{handoutWithNotes}
%\pgfpagesuselayout{3 on 1 with notes}[letterpaper,border shrink=5mm]
%	\setbeamercolor{background canvas}{bg=black!5}

%\includeonlylecture{student}

%%% Including only some slides for students.
%%% Uncomment the following line. For the slides,
%%% use the labels shown below the command.

%% For students, use \lecture{student}{student}
%% For mine, use \lecture{instructor}{instructor}


%\usepackage{pgf,pgfpages}
%\pgfpagesuselayout{4 on 1}[letterpaper,border shrink=5mm]

% FONTS
\usepackage{fontspec}
\def\mainfont{Linux Biolinum O}
\setmainfont[Ligatures={Common,TeX}, Contextuals={NoAlternate}, BoldFont={* Bold}, ItalicFont={* Italic}, Numbers={OldStyle}]{\mainfont}
\setsansfont[Scale=MatchLowercase, Numbers={OldStyle}]{Linux Biolinum O} 
\usepackage{microtype}

\addfontfeatures{Numbers=SlashedZero}

\usepackage{graphicx}
	\graphicspath{%
	{/Users/goby/Pictures/teach/434/lectures/}%
	{/Users/goby/Pictures/teach/348/lectures/}%
	{/Users/goby/Pictures/teach/434/handouts/}%
	{/Users/goby/Pictures/teach/common/}%}%
	{img/}} % set of paths to search for images

\usepackage{amsmath,amssymb}

%\usepackage{units}

\usepackage{booktabs}
\usepackage{multicol}
%	\setlength{\columnsep=1em}

\usepackage{chemfig}
\usepackage[version=4]{mhchem}
\usepackage{tikz}
	\tikzstyle{every picture}+=[remember picture,overlay]

\mode<presentation>
{
  \usetheme{Lecture}
  \setbeamercovered{invisible}
  \setbeamertemplate{items}[square]
}

\usepackage{hyperref}


\usepackage{calc} % Necessary for hidden word function.
\newcommand\HiddenWord[1]{%
	\alt<handout>{\rule{\widthof{#1}}{\fboxrule}}{#1}%
}

%%% Creative Commons Licenses. Establish, then add to Beamer template.
%\newcommand{\ccbysa}[1][4]{\textsc{cc by-sa #1.0}} % Use version 4.0 as default.
\newcommand{\ccby}[1]{%
	\ifx&#1&
	{\textsc{cc by}}%
\else
	{\textsc{cc by #1.0}}
\fi}


\newcommand{\ccbysa}[1]{%
	\ifx&#1&
	{\textsc{cc by-sa}}%
\else
	{\textsc{cc by-sa #1.0}} 
\fi}

\newcommand{\ccbyncsa}[1]{%
	\ifx&#1&
	{\textsc{cc by-nc-sa}}%
\else
	{\textsc{cc by-nc-sa #1.0}}
\fi}

\newcommand{\ccbync}[1]{%
	\ifx&#1&
	{\textsc{cc by-nc}}%
\else
	{\textsc{cc by-nc #1.0}}
\fi}

% Use the to temporarily set a background grid for positioning.
%\setbeamertemplate{background}[grid][step=1cm]



\begin{document}

\lecture{student}{student}

{
\usebackgroundtemplate{\includegraphics[width=\paperwidth]{07_phytoplankton_intro}}
\begin{frame}[b]

	\hfill \tiny \textcolor{white}{\textsc{noaa mesa} project, public domain.}
\end{frame}
}
%
\begin{frame}[t]{All plankton can be categorized by size.}

	\begin{center}
	\begin{tabular}{@{}lll@{}}
	
	\toprule
	Size Class & Size Range	& Example Taxa \\
	\midrule
	Megaplankton 	& 	\textgreater20 cm	& Jellyfish \\
	Macroplankton 	&	2–20	cm		& Krill\\
	Mesoplankton	&	0.2–20 mm	& Copepods \\
	Microplankton	&	20–200 µm	& Diatoms \\
	Nanoplankton	&	2–20 µm		& Cocolithophores \\
	Picoplankton	&	0.2–2 µm		& Cyanobacteria \\
	Femtoplankton	&	\textless0.2 µm	& Viruses \\
	\bottomrule
	\end{tabular}
	\end{center}
	
	\hangpara Micro- to megaplankton are collectively called net plankton.
	
	\pause \hangpara \highlight{Trivia:} What is the largest planktonic organism?
	
\end{frame}
%
\lecture{instructor}{instructor}

{
\usebackgroundtemplate{\includegraphics[width=\paperwidth]{07_mola_plankton}}
\begin{frame}[b]

	\hfill \tiny \textcolor{white}{\textsc{noaa}, public domain.}
\end{frame}
}
%
\begin{frame}[t]{They often bask so that birds can remove parasites.}

	\centering\includegraphics[height=0.8\textheight]{07_mola_albatross}

	\vfilll
	
	\hfill \tiny Abe and Sekiguchi 2012. Communicative \& Integrative Biology 5: 395.
\end{frame}

%
\lecture{student}{student}

\begin{frame}[t]{What are the two major classes of \highlight{phytoplankton?} }
	\vspace*{-\baselineskip}
	
	\begin{multicols}{2}
		{\centering
		\includegraphics[width=0.36\textwidth]{diatoms}
		
		\vspace*{0.5\baselineskip}
		
		\includegraphics[width=0.46\textwidth]{dinoflagellates}}
		
	\columnbreak
		
		\vspace*{0.5\baselineskip}
		
		\onslide<2->{
		\hangpara\HiddenWord{\highlight{Diatoms:} Bacillariophyceae}

		\vspace*{\baselineskip}

		\hangpara\HiddenWord{\highlight{Dinoflagellates:} Dinophyceae}}
		
	\end{multicols}

	\vfilll

\tiny top: USGS Photos from Life; bottom: \textcopyright\,McGraw-Hill.
\end{frame}
%
\begin{frame}[t]{Other phytoplankton are important in the tropics.}
	\vspace*{-\baselineskip}
	
	\begin{multicols}{2}
		
		\begin{center}
		\includegraphics[width=0.36\textwidth]{07_synechococcus}
		
		\vspace*{0.5\baselineskip}
		
		\includegraphics[width=0.36\textwidth]{07_coccolithophore}
		\end{center}
		
	\columnbreak
		
		\vspace*{0.5\baselineskip}
		
		\hangpara \highlight{Picoplankton:} Cyanobacteria\\\textit{Prochlorococcus} \& \textit{Synechococcus}

		\vspace*{\baselineskip}

		\hangpara \highlight{Nanoplankton:} Coccolithophyceae
		
	\end{multicols}

	\vfilll

	\tiny top: Masur, Wikimedia, public domain; bottom: \textcopyright Sinauer Associates, Inc.
\end{frame}
%
\begin{frame}[t]{Phytoplankton biology affects community structure.}

\begin{multicols}{2}

		\hangpara \highlight{Diatoms} and \highlight{dinoflagellates}

		\hangpara Low surface:volume ratio.
		
		\hangpara Heavy.

		\hangpara Better competitors in \\
		\hspace*{1em}cold, turbulent, and \\
		\hspace*{1em}high nutrient conditions.
		
\columnbreak

		\hangpara \highlight{Picoplankton}

		\hangpara High surface:volume ratio.

		\hangpara Light.
		
		\hangpara Better competitors in \\
		\hspace*{1em} warm, stratified, and \\
		\hspace*{1em} low nutrient conditions.

\end{multicols}

\end{frame}
%
\begin{frame}[t]{Review: What is primary productivity?}

\begin{center}
	\ce{CO2 + 6H2O ->[energy] C6H12O6 + 6O2}
	
	\vspace*{0.5\baselineskip}
	
	\includegraphics[width=\textwidth]{global_primary_productivity}
	
	\vfilll
	
	\hfill \tiny \textsc{noaa}, public domain.
\end{center}

\end{frame}
%
\begin{frame}[t]{Pico- and nanoplankton are most abundant in \highlight{oligotrophic} water.}

	\vspace*{-0.5\baselineskip}
	\begin{center}
		\includegraphics[height=0.78\textheight]{07_chlorophyll_fractionation}
	\end{center}
	\vfilll
	
\hfill \tiny Fig 16.2, \textcopyright Sinauer Associates, Inc.
\end{frame}
%
{
\usebackgroundtemplate{\includegraphics[width=\paperwidth]{07_pico_temperature}}
\begin{frame}[b]

	\tiny Mara\~non et al. 2012. Limnol. Oceanogr. 57: 1266.
\end{frame}
}
%
{
\usebackgroundtemplate{\includegraphics[width=\paperwidth]{thermocline_halocline}}
\begin{frame}[b]{Review: What is a thermocline?}

	\hfill \tiny \textcopyright McGraw-Hill.
\end{frame}
}
%
\begin{frame}[t]{Review: What is the Redfield ratio?}
	\hangpara C: \onslide<2>{106 atoms}
	
	\hangpara N: \onslide<2>{16 atoms}
	
	\hangpara P: \onslide<2>{1 atoms}
	
	\onslide<2>{\hangpara Which of these elements was thought to be most limiting?}
	
\end{frame}
%
\begin{frame}[t]{Review: Explain the \highlight{``classic model''} of GPP.}

	\vspace*{-0.5\baselineskip}
	\begin{center}
		\includegraphics[height=0.8\textheight]{classic_model_gpp}
	\end{center}
	\vfilll
	
\hfill \tiny\textcopyright Oxford University Press
\end{frame}
%
\lecture{instructor}{instructor}

\begin{frame}[t]{\highlight{Quiz:} Identify one bottom-up and one top-down factor.}

	\vspace*{-0.5\baselineskip}
	\begin{center}
		\includegraphics[height=0.8\textheight]{classic_model_gpp}
	\end{center}
	\vfilll
	
\hfill \tiny\textcopyright Oxford University Press
\end{frame}
%
\lecture{student}{student}

\begin{frame}[t]{The classic model does not work at tropical latitudes.}

	\vspace*{-0.5\baselineskip}
	\begin{center}
		\includegraphics[width=\textwidth]{global_primary_productivity}
	\end{center}
	\vfilll
	
	\hfill \tiny \textsc{noaa}, public domain.
\end{frame}

%
\lecture{instructor}{instructor}
{
\usebackgroundtemplate{\includegraphics[width=\paperwidth]{07_community_structure_model1}}
\begin{frame}[b]

	\hfill \tiny Fig. 16.3, \textcopyright Sinauer Associates, Inc.
\end{frame}
}
%
{
\usebackgroundtemplate{\includegraphics[width=\paperwidth]{07_community_structure_model2}}
\begin{frame}[b]

	\hfill \tiny Fig. 16.3, \textcopyright Sinauer Associates, Inc.
\end{frame}
}
%
{
\usebackgroundtemplate{\includegraphics[width=\paperwidth]{07_community_structure_model3}}
\begin{frame}[b]

	\hfill \tiny Fig. 16.3, \textcopyright Sinauer Associates, Inc.
\end{frame}
}
%
\lecture{student}{student}
{
\usebackgroundtemplate{\includegraphics[width=\paperwidth]{07_community_structure_model4}}
\begin{frame}[b]

	\hfill \tiny Fig. 16.3, \textcopyright Sinauer Associates, Inc.
\end{frame}
}
%
\begin{frame}[t]{Efficiency should select for smaller phytoplankton.}

\hangpara Problem: N and P uptake decreases with larger cell size. Why aren't all phytoplankton communities dominated by picoplankton?

\begin{multicols}{2}

		\hangpara \highlight{Diatoms} and \highlight{dinoflagellates}

		\hangpara Large.
		
		\hangpara Low surface:volume ratio.
				
\columnbreak

		\hangpara \highlight{Picoplankton}

		\hangpara Small.

		\hangpara High surface:volume ratio.
		
\end{multicols}

\pause

\hangpara Solution: Smaller phytoplankton are grazed more quickly by their predators.

\end{frame}
%
\lecture{instructor}{instructor}
{
\usebackgroundtemplate{\includegraphics[width=\paperwidth]{07_size_structured_model1}}
\begin{frame}[b]{Picoplankton abundance is limited initially by nutrients.}

	\hfill \tiny Fig. 16.4, \textcopyright Sinauer Associates, Inc.
\end{frame}
}
%
{
\usebackgroundtemplate{\includegraphics[width=\paperwidth]{07_size_structured_model2}}
\begin{frame}[b]{Picoplankton abundance becomes limited by predators.}

	\hfill \tiny Fig. 16.4, \textcopyright Sinauer Associates, Inc.
\end{frame}
}
%
{
\usebackgroundtemplate{\includegraphics[width=\paperwidth]{07_size_structured_model3}}
\begin{frame}[b]{Nanoplankton abundance is limited by nutrients.}

	\hfill \tiny Fig. 16.4, \textcopyright Sinauer Associates, Inc.
\end{frame}
}
%
{
\usebackgroundtemplate{\includegraphics[width=\paperwidth]{07_size_structured_model4}}
\begin{frame}[b]{Nanoplankton abundance becomes limted by predators.}

	\hfill \tiny Fig. 16.4, \textcopyright Sinauer Associates, Inc.
\end{frame}
}
%
\lecture{student}{student}

{
\usebackgroundtemplate{\includegraphics[width=\paperwidth]{07_size_structured_model5}}
\begin{frame}[b]{Abundant nutrients supports the greatest diversity.}

	\hfill \tiny Fig. 16.4, \textcopyright Sinauer Associates, Inc.
\end{frame}
}
%
{
\usebackgroundtemplate{\includegraphics[width=\paperwidth]{07_vertical_distribution}}
\begin{frame}[b]

	\begin{tikzpicture}
	
		\node at (3.5,7) {low nutrients};
	
		\draw [dashed, blue, ultra thick] (1,4.3) -- (5,4.3) node[above, align=center] {{\footnotesize thermocline}};
		
		\node at (4.1,1.5) {low light};

	\end{tikzpicture}


	\hfill \tiny Fig. 16.5, \textcopyright Sinauer Associates, Inc.
\end{frame}
}
%
\begin{frame}[t]{Patchy distribution of nutrients creates patchy distribution of phytoplankton.}

	\vspace*{-0.5\baselineskip}
	\begin{center}
		\includegraphics[height=0.78\textheight]{07_colimitation}
	\end{center}
	\vfilll
	
	\hfill \tiny Fig. 16.6, \textcopyright Sinauer Associates, Inc.
\end{frame}

%
\begin{frame}[t]{Predators and specialized ``enemies'' exert a top-down effect on community structure.}

	\vspace*{-0.5\baselineskip}
	\begin{center}
		\includegraphics[width=\textwidth]{07_kill_the_winner}
	\end{center}
	\vfilll
	
	\hfill \tiny Fig. 16.9, \textcopyright Sinauer Associates, Inc.
\end{frame}

%
\end{document}

