%!TEX TS-program = lualatex
%!TEX encoding = UTF-8 Unicode

\documentclass[t]{beamer}

%%%% HANDOUTS For online Uncomment the following four lines for handout
%\documentclass[t,handout]{beamer}  %Use this for handouts.
%\includeonlylecture{student}
%\usepackage{handoutWithNotes}
%\pgfpagesuselayout{3 on 1 with notes}[letterpaper,border shrink=5mm]


%%% Including only some slides for students.
%%% Uncomment the following line. For the slides,
%%% use the labels shown below the command.

%% For students, use \lecture{student}{student}
%% For mine, use \lecture{instructor}{instructor}


%\usepackage{pgf,pgfpages}
%\pgfpagesuselayout{4 on 1}[letterpaper,border shrink=5mm]

% FONTS
\usepackage{fontspec}
\def\mainfont{Linux Biolinum O}
\setmainfont[Ligatures={Common,TeX}, Contextuals={NoAlternate}, BoldFont={* Bold}, ItalicFont={* Italic}, Numbers={OldStyle}]{\mainfont}
\setsansfont[Scale=MatchLowercase, Numbers={OldStyle}]{Linux Biolinum O} 
\usepackage{microtype}


\addfontfeatures{Numbers=SlashedZero}

\usepackage{graphicx}
	\graphicspath{%
	{/Users/goby/Pictures/teach/434/lectures/}%
	{/Users/goby/Pictures/teach/348/lectures/}%
	{/Users/goby/Pictures/teach/434/handouts/}%
	{/Users/goby/Pictures/teach/common/}%}%
	{img/}} % set of paths to search for images

\usepackage{amsmath,amssymb}

%\usepackage{units}

\usepackage{booktabs}
\usepackage{multicol}
%	\setlength{\columnsep=1em}

\usepackage{chemfig}
\usepackage[version=4]{mhchem}
\usepackage{tikz}
	\tikzstyle{every picture}+=[remember picture,overlay]
\usetikzlibrary{shapes}

\mode<presentation>
{
  \usetheme{Lecture}
  \setbeamercovered{invisible}
  \setbeamertemplate{items}[square]
}

\usepackage{hyperref}


\usepackage{calc} % Necessary for hidden word function.
\newcommand\HiddenWord[1]{%
	\alt<handout>{\rule{\widthof{#1}}{\fboxrule}}{#1}%
}


% Use the to temporarily set a background grid for positioning.
%\setbeamertemplate{background}[grid][step=1cm]



\begin{document}

\lecture{student}{student}

{\usebackgroundtemplate{\includegraphics[width=\paperwidth]{08_rocky_intertidal_intro}}
\begin{frame}[b]{Rocky Intertidal Ecosystems}
\tiny\textcolor{white}{\textcopyright Oxford University Press}
\end{frame}
}


\begin{frame}[t]{Review: the rocky intertidal has three zones.}

\vspace*{-\baselineskip}

\begin{multicols}{2}

	\begin{center}
		\includegraphics[width=0.48\textwidth]{08_rocky_intertidal_zones}
	\end{center}

\columnbreak

	\hangpara\highlight{Supralittoral fringe}\\
	\hspace*{0.5em}splash zone\\
	\hspace*{0.5em}periwinkles, lichens, and\\
	\hspace*{0.5em}encrusting algae dominate.
	

	\hangpara\highlight{Midlittoral zone}\\
	\hspace*{0.5em}greatest diversity overall\\
	\hspace*{0.5em}barnacles and mussels dominate.

	\hangpara\highlight{Infralittoral zone}\\
	\hspace*{0.5em}aka \highlight{sublittoral fringe}\\
	\hspace*{0.5em}brown macroalgae dominate.

\end{multicols}

	\vfilll

	\tiny\textcopyright\,McGraw-Hill

\end{frame}
%
{\usebackgroundtemplate{\includegraphics[width=\paperwidth]{08_rocky_zones_identified}}
\begin{frame}[t]{\textcolor{white}{Review}}

%	\vspace*{\baselineskip}
	\hspace*{60mm}\hangpara Which zone is dominated by\\
	\hspace*{65mm} biological process?\\
	\hspace*{65mm} physical process?

	\vfilll
	\hfill\tiny Bcasterline, Wikimedia, Public Domain.
\end{frame}
}
%
{
\usebackgroundtemplate{\includegraphics[width=\paperwidth]{08_scale_dependent_processes}}
\begin{frame}[b]{Pattern and process is related to scale.}

	\tiny Fig. 9.4 \textcopyright Sinauer Associates, Inc.
\end{frame}
}
%
{
\usebackgroundtemplate{\includegraphics[width=\paperwidth]{08_intermittant_upwelling_model}}
\begin{frame}[b]{\highlight{Intermittent upwelling} determines community structure from the bottom up.}

	\tiny Menge and Menge 2013. Ecological Monographs 83: 283.
\end{frame}
}
%
{
\usebackgroundtemplate{\includegraphics[width=\paperwidth]{08_intermittant_upwelling}}
\begin{frame}[b]{Intermittent upwelling should maximize ecosystem function.}

	\alt<handout>{}{\begin{tikzpicture}
	
		\onslide<1-2>\draw [fill=white, color=white] (6,2) -- (12.3,2) -- (12.3,7) -- (6,7) -- cycle;
		\onslide<2>\node [align=left, text width=1.5in] at (9.5,6.5) {{\small Look familiar?}};
		
		\onslide<3>\node [align=left, text width=1.5in] at (9.5,6.5) {{\small Intermediate disturbance}};

	\end{tikzpicture}}%\pause

	\hfill \tiny Fig. 9.3 \textcopyright Sinauer Associates, Inc.
\end{frame}
}
%
{
\usebackgroundtemplate{\includegraphics[width=\paperwidth]{08_intermittent_results}}
\begin{frame}[b]

	\hfill \tiny \rotatebox{90}{Menge and Menge 2013. Ecological Monographs 83: 283.} 

	\begin{tikzpicture}
	
		\alt<handout>{}{\onslide<1>\draw [fill=white, color=white] (6,-0.5) -- (11.7,-0.5) -- (11.7,9.6) -- (6,9.6) -- cycle;}
		
	\end{tikzpicture} \pause

\end{frame}
}
%

{
\usebackgroundtemplate{\includegraphics[width=\paperwidth]{08_latitude_thermal_stress_prediction}}
\begin{frame}[b]{Thermal stress should depend on latitudes.}

	\hfill \tiny Fig. 9.5 \textcopyright Sinauer Associates, Inc.
\end{frame}
}
%
{
\usebackgroundtemplate{\includegraphics[width=\paperwidth]{08_latitude_thermal_stress_result}}
\begin{frame}[b]{Instead, large scale variability affects thermal stress.}

	\hfill \tiny Fig. 9.5 \textcopyright Sinauer Associates, Inc.
\end{frame}
}
%
{
\usebackgroundtemplate{\includegraphics[width=\paperwidth]{08_alternative_stable_states}}
\begin{frame}[b]{Do alternate stable states exist in the Gulf of Maine?}

	\hfill \tiny \textcolor{white}{Fig. 9.7 \textcopyright Sinauer Associates, Inc.}
\end{frame}
}
%
{
\usebackgroundtemplate{\includegraphics[width=\paperwidth]{08_ass_model}}
\begin{frame}[b]{Alternative stable states can shift between algal and mussel states after disturbance.}

	\hfill \tiny Petraitis and Dudgeon 2004.\\ \hfill J. Exp. Mar. Bio. Ecol. 300: 343.
\end{frame}
}
%
\begin{frame}[t]{Alternate stable states occurred in patches larger than about 10 m\textsuperscript{2}.}


	\begin{center}
		\includegraphics[width=\textwidth]{08_rocky_alternate_stable_states} 
	\end{center}


	\vfilll

	\hfill \tiny \textcopyright Oxford University Press.

\end{frame}
%
{
\usebackgroundtemplate{\includegraphics[width=\paperwidth]{08_alternate_states_gulf_maine}}
\begin{frame}[t]{Does the pattern hold at larger scales in the Gulf?}

	\begin{tikzpicture}
		
		\draw [fill=blue] (9.5, -0.5) -- (9.7,-0.5) -- (9.6,-0.3) -- cycle node[right, xshift=2mm, yshift=0.5mm] {\footnotesize North region};

		\draw [fill=black] (9.6,-1) circle (0.1cm) node [right, xshift=1mm] {\footnotesize Penobscot};
		
		\draw [fill=red] (9.5,-1.4) rectangle (9.7,-1.6) node [right, yshift=0.5mm] {\footnotesize South region}; 
		
		\node [right] at (9.5, -2.5) {\footnotesize Filled: Exposed};
		
		\node [right] at (9.5, -3) {\footnotesize Open: Sheltered};
		
		
	\end{tikzpicture}

	\vfilll
	
	\hfill \tiny Bryson et al. 2014. Ecological Monographs 84: 579.
\end{frame}
}
%
{
\usebackgroundtemplate{\includegraphics[width=\paperwidth]{08_alternate_states_gulf_maine_nmds}}
\begin{frame}[b]{Northern Gulf of Maine is consistently algal-dominated.}

	\hfill \tiny Bryson et al. 2014. Ecological Monographs 84: 579.
\end{frame}
}
%
{
\usebackgroundtemplate{\includegraphics[width=\paperwidth]{08_subtidal_trophic_cascade}}
\begin{frame}[b]
\end{frame}
}
%
{
\usebackgroundtemplate{\includegraphics[width=\paperwidth]{08_non_consumptive_effects}}
\begin{frame}[b]{\highlight{Non-consumptive effects} induce changes in prey traits.}

	\hfill \tiny Matassa and Trussell 2014. Proc. R. Soc. B 281: 20141952.
\end{frame}
}
%
{
\usebackgroundtemplate{\includegraphics[width=\paperwidth]{08_tmii_cascade}}
\begin{frame}[b]{\textsc{Nce}s can cause trophic cascades through \highlight{trait-mediated indirect interactions.}}

	\onslide<1->\tiny Fig. 9.14 \textcopyright Sinauer Associates, Inc. \hfill \textcopyright McGraw-Hill

	\begin{tikzpicture}
	
		\alt<handout>{}{\onslide<1>\draw [fill=white, color=white] (6,2) -- (12.3,2) -- (12.3,7) -- (6,7) -- cycle;}
		
	\end{tikzpicture} \pause
\end{frame}
}
%
{
\usebackgroundtemplate{\includegraphics[width=\paperwidth]{08_trophic_heat}}
\begin{frame}[t]{\textsc{Tmii}s can decrease the efficiency of energy transfer to higher trophic levels.}

	\vspace*{2\baselineskip}

	\hspace*{75mm}\parbox{45mm}{\raggedright This would increase the amount of \rule{1.5cm}{0.4pt} lost from the system.}

	\vspace*{2\baselineskip}
	
	\hspace*{75mm}\parbox{45mm}{\raggedright Why would this shorten food chains in the rocky intertidal?}

	\vfilll
	
	\hfill \tiny Fig. 9.13 \textcopyright Sinauer Associates, Inc.
\end{frame}
}
%
{
\usebackgroundtemplate{\includegraphics[width=\paperwidth]{08_pisaster_feeding_design}}
\begin{frame}[b]{Will increasing climate variability affect feeding rates of \textit{Pisaster} sea stars?}

\hfill \tiny Fig. 9.19\\\hfill\textcopyright Sinauer
%	\hfill \tiny \rotatebox{90}{Fig. 9.19 \textcopyright Sinauer Associates, Inc.}
\end{frame}
}

%
{
\usebackgroundtemplate{\includegraphics[width=\paperwidth]{08_pisaster_feeding_results}}
\begin{frame}[b]{Greater temperature discordance decreased \textit{Pisaster} feeding rate.}

\tiny Pincebourde et al. 2012.\\
\tiny Ecology Letters 15: 680.
%	\hfill \tiny \rotatebox{90}{Pincebourde et al. 2012. Ecology Letters 15:680.}
\end{frame}
}

%
\end{document}

