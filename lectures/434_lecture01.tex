%!TEX TS-program = lualatex
%!TEX encoding = UTF-8 Unicode

\documentclass[t]{beamer}

%%%% HANDOUTS For online Uncomment the following four lines for handout
%\documentclass[t,handout]{beamer}  %Use this for handouts.
%\usepackage{handoutWithNotes}
%\pgfpagesuselayout{3 on 1 with notes}[letterpaper,border shrink=5mm]

%\includeonlylecture{student}

%% For students, use \lecture{student}{student}
%% For mine, use \lecture{instructor}{instructor}


%\usepackage{pgf,pgfpages}
%\pgfpagesuselayout{4 on 1}[letterpaper,border shrink=5mm]

% FONTS
\usepackage{fontspec}
\def\mainfont{Linux Biolinum O}
\setmainfont[Ligatures={Common,TeX}, Contextuals={NoAlternate}, BoldFont={* Bold}, ItalicFont={* Italic}, Numbers={Proportional}]{\mainfont}
\setsansfont[Scale=MatchLowercase]{Linux Biolinum O} 
\usepackage{microtype}

\usepackage{graphicx}
	\graphicspath{%
	{/Users/goby/Pictures/teach/434/lectures/}%
	{/Users/goby/Pictures/teach/common/}%}%
	{img/}} % set of paths to search for images

\usepackage{amsmath,amssymb}

%\usepackage{units}

\usepackage{booktabs}
\usepackage{multicol}
%	\setlength{\columnsep=1em}

%\usepackage{textcomp}
%\usepackage{setspace}
\usepackage{tikz}
	\tikzstyle{every picture}+=[remember picture,overlay]

\mode<presentation>
{
  \usetheme{Lecture}
  \setbeamercovered{invisible}
  \setbeamertemplate{items}[default]
}

%\usepackage{calc}
\usepackage{hyperref}

\newcommand\HiddenWord[1]{%
	\alt<handout>{\rule{\widthof{#1}}{\fboxrule}}{#1}%
}

\begin{document}

%\lecture{instructor}{instructor}
\lecture{student}{student}

{
\usebackgroundtemplate{\includegraphics[width=\paperwidth]{01_mar_eco_cons_intro}}
\begin{frame}[t]
\end{frame}
}

{
\usebackgroundtemplate{\includegraphics[width=\paperwidth]{01_LV_diving}}
\begin{frame}[t,plain]
\large
\vspace{5ex}

\hangpara Mike Taylor

\hangpara RH 217

\hangpara mtaylor@semo.edu

\end{frame}
}

\begin{frame}[t]{Your \highlight{objectives} for this course are to}

	\hangpara apply ecological principles to conservation of marine ecosystems,

	\hangpara explain ecosystem services provided by different marine ecosystems, 

	\hangpara summarize key findings from the marine scientific literature, and
	
	\hangpara write a conservation plan for a marine ecosystem.
		
\end{frame}


\begin{frame}[t]{You \highlight{earn} your grade with}
	\begin{center}\large\begin{tabular}{@{}lll@{}}
	UG		&	Three 75-point exams 				& 	50\% \\
			&	Assignments 						& 	25\% \\
			&	Conservation plan 					& 	25\% \\
			&										&	\\
	G / HC	&	Three $\geq$ 75-point exams 		& 	40\% \\
			&	Assignments 						& 	15\% \\
			&	Conservation plan and leadership 	&	25\% \\
			&	Conservation plan presentation 		&	20\% \\
	\end{tabular}
	\end{center}
\end{frame}


{\usebackgroundtemplate{\includegraphics[width=\paperwidth]{01_critical_analyses}}
\begin{frame}
\end{frame}
}

{\usebackgroundtemplate{\includegraphics[width=\paperwidth]{01_conservation_plan}}
\begin{frame}
\end{frame}
}

{\usebackgroundtemplate{\includegraphics[width=\paperwidth]{01_textbook}}
\begin{frame}
\end{frame}
}

\begin{frame}[t]{Let us begin with simple questions.}

	\hangpara What is ecology?
	
	\hangpara What is a community? An assemblage? An ecosystem?

	\hangpara How do these apply to conservation?

	\hangpara Take a few minutes to meet the people at your table and answer these questions.

\end{frame}

{\usebackgroundtemplate{\includegraphics[width=\paperwidth]{01_barrier_island}}
\begin{frame}[b]{What are ecosystem services? Why do they matter?}
	
\tiny\hfill\textcolor{white}{Gulf Restoration Network / Southwings, Flickr Creative Commons.}

\end{frame}
}

\lecture{instructor}{instructor}

\begin{frame}[t]
	
	“Ecosystem services are the direct or indirect contributions that ecosystems make to the well-being of human populations.” — U.S. EPA, 2009 \vspace*{\baselineskip}
	
	\includegraphics[width=\textwidth]{01_clam_diggers}
	
	A 2016 algal bloom in Washington state caused \textgreater \$93 million of economic loss for clam and crab fisheries. --- NOAA / National Ocean Service
	
	\vskip0pt plus 1filll
	
	\hfill\tiny NOAA / National Ocean Service, Flickr Creative Commons
	
\end{frame}

\lecture{student}{student}

\begin{frame}[t]{Ecosystem services provide \highlight{value} that influence human activity.}
	\includegraphics[width=\textwidth]{01_ecoservices_quantifying_valuation}
	
	\vfilll
	
	\hfill \tiny Fig.~18.1 \copyright Sinauer Associates, Inc.
\end{frame}

\begin{frame}[t]{The U.S. EPA (2009) identified two characteristics of ecosystem services.}

	\begin{enumerate}

	\item A wide range of goods and services that are valuable to humans arise in myriad ways via the structure and functions of an ecosystem.

	\vspace*{\baselineskip}
		
	\item Although they are the source of ecosystem services, the structure and functions of an ecosystem are not synonymous with such services.
	
	\end{enumerate}
	
	\hangpara What is meant by “structure and functions”?

\end{frame}

\lecture{instructor}{instructor}

{\usebackgroundtemplate{\includegraphics[width=\paperwidth]{01_caribbean_sea}}
\begin{frame}[b]{\textcolor{white}{Can you find Bonaire?}}

\hfill\tiny\textcolor{white}{kmusser, Wikimedia Commons.}
\end{frame}
}

\lecture{student}{student}

{\usebackgroundtemplate{\includegraphics[width=\paperwidth]{01_bonaire_ecosystem_services}}
\begin{frame}[b]

\hfill\tiny\textcolor{white}{Rebe Leubert, Flickr Creative Commons.}
\end{frame}
}

\end{document}
