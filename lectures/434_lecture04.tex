%!TEX TS-program = lualatex
%!TEX encoding = UTF-8 Unicode

\documentclass[t]{beamer}

%%%% HANDOUTS For online Uncomment the following four lines for handout
%\documentclass[t,handout]{beamer}  %Use this for handouts.
%\usepackage{handoutWithNotes}
%\pgfpagesuselayout{3 on 1 with notes}[letterpaper,border shrink=5mm]
%	\setbeamercolor{background canvas}{bg=black!5}


%%% Including only some slides for students.
%%% Uncomment the following line. For the slides,
%%% use the labels shown below the command.
%\includeonlylecture{student}

%% For students, use \lecture{student}{student}
%% For mine, use \lecture{instructor}{instructor}


%\usepackage{pgf,pgfpages}
%\pgfpagesuselayout{4 on 1}[letterpaper,border shrink=5mm]

% FONTS
\usepackage{fontspec}
\def\mainfont{Linux Biolinum O}
\setmainfont[Ligatures={Common,TeX}, Contextuals={NoAlternate}, BoldFont={* Bold}, ItalicFont={* Italic}, Numbers={OldStyle}]{\mainfont}
\setsansfont[Scale=MatchLowercase, Numbers={OldStyle}]{Linux Biolinum O} 
\usepackage{microtype}

\addfontfeatures{Numbers=SlashedZero}

\usepackage{graphicx}
	\graphicspath{%
	{/Users/goby/Pictures/teach/434/lectures/}%
	{/Users/goby/Pictures/teach/434/handouts/}%
	{/Users/goby/Pictures/teach/common/}%}%
	{img/}} % set of paths to search for images

\usepackage{amsmath,amssymb}

%\usepackage{units}

\usepackage{booktabs}
\usepackage{multicol}
%	\setlength{\columnsep=1em}

%\usepackage{textcomp}
%\usepackage{setspace}
\usepackage{tikz}
	\tikzstyle{every picture}+=[remember picture,overlay]

\mode<presentation>
{
  \usetheme{Lecture}
  \setbeamercovered{invisible}
  \setbeamertemplate{items}[square]
}

%\usepackage{calc}
\usepackage{hyperref}

\newcommand\HiddenWord[1]{%
	\alt<handout>{\rule{\widthof{#1}}{\fboxrule}}{#1}%
}



\begin{document}

\lecture{student}{student}

\begin{frame}[t]{What is marine \highlight{restoration ecology?}}
\end{frame}
%

\lecture{instructor}{instructor}

\begin{frame}[t]{Several ideas but all involve \highlight{human intervention.}}

	\hangpara Narrow: Recreate the structure and function of the ecosystem to approximate natural conditions before the system was disturbed (Natural Research Council 1992).
	
	\pause
	
	\hangpara Broad: Strategies to conserve ecosystem diversity and integrity (Jordan 1988, Carrns and Heckman 1996).

	\pause
	
	\hangpara \highlight{Restoration ecology} emphasizes an ecological perspective rather than economic, logistic, or other perspective.
\end{frame}

%
\lecture{student}{student}

\begin{frame}[t]{Restoration can take one or combination of approaches.}

	\onslide<1->\hangpara \highlight{Population:} \onslide<2->May emphasize management more than true restoration.
	
	\onslide<1->\hangpara \highlight{Habitat:} \onslide<3->Emphasizes foundation / ecosystem engineer species.
	
	\onslide<1->\hangpara \highlight{Landscape:} \onslide<4->Emphasizes connectivity between adjacent habitats.
	
	\onslide<1->\hangpara \highlight{Ecosystem:} \onslide<5->Like landscape but emphasizes much larger scale. Has greater economic cost but potentially greater reward.
	
\end{frame}
%
{
\usebackgroundtemplate{\includegraphics[width=\paperwidth]{04_positive_effects_diagram}}
\begin{frame}[b]

\tiny\textcopyright Sinauer Associates, Inc. 
\end{frame}
}
%
{
\usebackgroundtemplate{\includegraphics[width=\paperwidth]{04_restoration_flowchart}}
\begin{frame}[b]

\hfill\tiny\textcopyright Sinauer Associates, Inc. 
\end{frame}
}
%
{
\usebackgroundtemplate{\includegraphics[width=\paperwidth]{04_oysters}}
\begin{frame}[t]{\textcolor{white}{Oysters: a case study in restoration ecology.}}

\vskip0pt plus 1filll

\hfill\tiny\textcopyright jsbaw7160, Pixabay, \textsc{cc0}. 
\end{frame}
}
%
\begin{frame}[t]{Oyster restoration: document historical resources.}

	\includegraphics[width=0.75\textwidth]{04a_flowchart_historical}
	
	\vskip0pt plus 1filll

\hfill\tiny Fig. 22.1 \textcopyright Sinauer Associates, Inc. 
\end{frame}
%
\begin{frame}[t]{Oysters, in the family Ostreidae,}

	\hangpara are bivalve mollusks with calcium carbonate shells,
	
	\hangpara are filter feeders that “clear” phytoplankton from the water,
	
	\hangpara have planktotrophic larve (10--18 days),

	\hangpara have a global distribution in temperate and tropical estuaries,
	
	\hangpara are a foundation species that forms large reefs, and
	
	\hangpara are commercially harvested for food and other uses.
\end{frame}
%
{
\usebackgroundtemplate{\includegraphics[width=\paperwidth]{04_natural_oysterbed}}
\begin{frame}[b]

\hfill\tiny\colorbox{gray!50!white}{Doug DuCap, Flickr \textsc{cc by-nc-sa 2.0}}
\end{frame}
}
%
{
\usebackgroundtemplate{\includegraphics[width=\paperwidth]{04_oyster_farm}}
\begin{frame}[b]

\tiny\textcolor{white}{Mike Peel, Wikimedia \textsc{cc by-sa 4.0}}
\end{frame}
}
%
\begin{frame}[t]{Oyster restoration: assess current status.}

	\includegraphics[width=0.75\textwidth]{04b_flowchart_assess}
	
	\vskip0pt plus 1filll

\hfill\tiny Fig. 22.1 \textcopyright Sinauer Associates, Inc. 
\end{frame}
%
{
\usebackgroundtemplate{\includegraphics[width=\paperwidth]{04_oyster_bed_distribution_condition}}
\begin{frame}[t]{Natural oyster beds are in global decline.}

\vskip0pt plus 1filll

\hfill\tiny Beck et al. 2011. BioScience 61: 107.
\end{frame}
}
%
{
\usebackgroundtemplate{\includegraphics[width=\paperwidth]{04_oyster_harvest}}
\begin{frame}[b]

\tiny Beck et al. 2011. BioScience 61: 107.
\end{frame}
}
%
%
\begin{frame}[t]{Oyster restoration: identify current bottlenecks.}

	\includegraphics[width=0.75\textwidth]{04c_flowchart_bottlenecks}
	
	\vskip0pt plus 1filll

\hfill\tiny Fig. 22.1 \textcopyright Sinauer Associates, Inc. 
\end{frame}
%
{
\usebackgroundtemplate{\includegraphics[width=\paperwidth]{04_netherlands_oyster_bed}}
\begin{frame}[b]{Oyster beds are subject to overexploitation, pollution, and climate change.}

\hfill\tiny\textcolor{white}{Sonty567, Wikimedia, \textsc{cc0}}
\end{frame}
}

%
\begin{frame}[t]{Oyster restoration: analyze results from experiments.}

	\includegraphics[width=0.75\textwidth]{04d_flowchart_experiment}
	
	\hspace{2.5cm}Here is where you come in.
	
	\vskip0pt plus 1filll

\hfill\tiny Fig. 22.1 \textcopyright Sinauer Associates, Inc. 
\end{frame}
%
\begin{frame}[t]{Shoreline armoring has increased dramatically.}

	\centering
		\includegraphics[width=0.85\textwidth]{04_shoreline_armoring}
	
	
	\vskip0pt plus 1filll

\hfill\tiny Scyphers et al. 2011. \textsc{pl}o\textsc{s} one 6(8): e22396.
\end{frame}
%
%
\begin{frame}[t]{They constructed oyster reefs by holding shells in place with mesh.}

	\centering
		\includegraphics[width=0.75\textwidth]{04_breakwater_dimensions}
	
	
	\vskip0pt plus 1filll

\hfill\tiny Scyphers et al. 2011. \textsc{pl}o\textsc{s} one 6(8): e22396.
\end{frame}
%
{
\usebackgroundtemplate{\includegraphics[width=\paperwidth]{04_breakwater_bathymetry}}
\begin{frame}[b]

\hfill\tiny Scyphers et al. 2011. \textsc{pl}o\textsc{s} one 6(8): e22396.
\end{frame}
}

%
\lecture{instructor}{instructor}
{
\setbeamercolor{background canvas}{bg=black}
\begin{frame}
\end{frame}
}

%
\begin{frame}[t]{Can restoration depend on larval transport?}

	\centering
		\includegraphics[width=0.85\textwidth]{oyster_larval_transport}
	
	
	\vskip0pt plus 1filll

\hfill\tiny Kim et al. 2013. Restoration Ecology 21: 353.
\end{frame}
%

%
\begin{frame}[t]{Do other foundation species in oyster reefs matter?}

	\centering
		\includegraphics[width=\textwidth]{oyster_mussel_biomass}
	
	
	\vskip0pt plus 1filll

\hfill\tiny Gedan et al. 2014. Restoration Ecology 22: 517.
\end{frame}
%
%
\begin{frame}[t]{Does oyster distribution change due to predators?}

	\centering
		\includegraphics[width=\textwidth]{oyster_density_predators}
	
	
	\vskip0pt plus 1filll

\hfill\tiny Fodrie et al. 2014. J. Applied Ecology 51: 1314.
\end{frame}
%
%
\begin{frame}[t]{Does type of substrate affect density or other organisms?}

	\centering
		\includegraphics[width=\textwidth]{oyster_nekton_macroinverts}
	
	
	\vskip0pt plus 1filll

\hfill\tiny Brown et al. 2014. Restoration Ecology 22: 214.
\end{frame}
%

%
\begin{frame}[t]{Does fish catch around oyster reefs change?}

	\centering
		\includegraphics[width=0.6\textwidth]{oyster_fish_catch}
	
	
	\vskip0pt plus 1filll

\hfill\tiny Scyphers et al. 2011. \textsc{pl}o\textsc{s} one 6(8): e22396.
\end{frame}
%
\begin{frame}[t]{How might oyster reef restoration be affected by climate change?}

\vspace*{2\baselineskip}

\hangpara “This spiral of engineering and reengineering can be avoided if the first priority becomes conservation and if sustainability is a key element defining success of any restoration initiatives.” — Bertness et al. 2014, \textit{Marine Community Ecology and Conservation}, pg. 511.
\end{frame}
\end{document}
