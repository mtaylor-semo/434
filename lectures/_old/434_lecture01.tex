%!TEX TS-program = lualatex
%!TEX encoding = UTF-8 Unicode

\documentclass[t]{beamer}

%%%% HANDOUTS For online Uncomment the following four lines for handout
%\documentclass[t,handout]{beamer}  %Use this for handouts.
%\usepackage{handoutWithNotes}
%\pgfpagesuselayout{3 on 1 with notes}[letterpaper,border shrink=5mm]
%	\setbeamercolor{background canvas}{bg=black!5}


%%% Including only some slides for students.
%%% Uncomment the following line. For the slides,
%%% use the labels shown below the command.
%\includeonlylecture{student}

%% For students, use \lecture{student}{student}
%% For mine, use \lecture{instructor}{instructor}


%\usepackage{pgf,pgfpages}
%\pgfpagesuselayout{4 on 1}[letterpaper,border shrink=5mm]

% FONTS
\usepackage{fontspec}
\def\mainfont{Linux Biolinum O}
\setmainfont[Ligatures={Common,TeX}, BoldFont={* Bold}, ItalicFont={* Italic}, Numbers={Proportional}]{\mainfont}
\setmonofont[Scale=MatchLowercase]{Inconsolata} 
\setsansfont[Scale=MatchLowercase]{Linux Biolinum O} 
\usepackage{microtype}

\usepackage{graphicx}
	\graphicspath{%
	{/Users/goby/Pictures/teach/434/}%
	{/Users/goby/Pictures/teach/common/}} % set of paths to search for images

\usepackage{amsmath,amssymb}

%\usepackage{units}

\usepackage{booktabs}
\usepackage{multicol}
%	\setlength{\columnsep=1em}
\usepackage{mhchem}

\usepackage{textcomp}
\usepackage{setspace}
\usepackage{tikz}
	\tikzstyle{every picture}+=[remember picture,overlay]

\mode<presentation>
{
  \usetheme{Lecture}
  \setbeamercovered{invisible}
  \setbeamertemplate{items}[square]
}

\begin{document}
%\lecture{instructor}{instructor}
\lecture{student}{student}

%% Intro

{
\usebackgroundtemplate{\includegraphics[width=\paperwidth]{intro_mantis}}
\begin{frame}[b,plain]
%	\hfill\tiny\textcolor{gray}{Chimaera by NOAA Ocean Explorer, Flickr Creative Commons.}
\end{frame}
}

%% Contact Info
{
\usebackgroundtemplate{\includegraphics[width=\paperwidth]{mike_snake}}
\begin{frame}[t,plain]{I will be your host for this class.}
\large
	\vspace{5ex}
	\hangpara\hspace{17em} Mike Taylor

	\hangpara\hspace{17em} RH 217

	\hangpara\hspace{17em} mtaylor@semo.edu

\end{frame}
}

%% Grades
\begin{frame}[t,plain]{You earn your grade with }

	\hangpara \highlight{Three exams} @ 100/\textasciitilde125 points, 

	\hangpara \highlight{Critical Analyses} @ 40 points,
	
	\hangpara \highlight{Home / In-Class Assignments} @ \textasciitilde200 points,
	
	\hangpara \highlight{Presentation and proposals, or evaluations} @120/170 points
	
	\hspace{2em} Grads present on an issue and write a supporting proposal,
	
	\hspace{2em} Undergrads evaluate presentations and proposals.

\end{frame}

%% Presentations and Proposals
\begin{frame}[t,plain]{The presentation will cover nearly any conceptual issue in marine ecology or evolution.}

	\hangpara The presentation will last 10–12 minutes.
	
	\hspace{2em} You will turn in copies of the literature you cite. 

	\hangpara Your proposal will be related to your presentation issue.

	\hangpara \highlight{Due dates are}
	
	\hspace{2em} Topic choice: 19 February,
	
	\hspace{2em} Final assignment: 21 April.
	
	\hangpara Grads will present very late in the semester.
	
	\hangpara Undergrads will evaluate presentations and proposals.

\end{frame}

%% Homework
\begin{frame}[t,plain]{Homework will serve as practice for the exams.}

	\hangpara Essay questions based on assigned reading, other sources, or my imagination. 

		\hspace{2em} The textbook \emph{is} required.
	
	\hangpara They will be related to an upcoming lecture topic; therefore

	\hangpara Due before the start of that lecture (no makeup).
	
	\hangpara In-class group assignments.

\end{frame}

{
\usebackgroundtemplate{\includegraphics[width=\paperwidth]{eco_knowledge}}
\begin{frame}[b,plain]{\textcolor{white}{Review chapter 3 for basic ecological knowledge.}}
	\tiny\textcolor{white!75!black}{Opalescent sea slug by John Albers-Mead, kqedquest, Flickr Creative Commons.}
\end{frame}
}

%% What is the marine environment?
{
\usebackgroundtemplate{\includegraphics[width=\paperwidth]{marine_environment}}
\begin{frame}[b,plain]{}
	\tiny\textcolor{white!75!black}{Oceanic whitetip shark by Michael Aston, Flickr Creative Commons.}
\end{frame}
}

%% Seawater properties
\begin{frame}[t,plain]{Seawater has unique properties related to }
\begin{multicols}{2}

	\hangpara Salinity (table at right)
	
	\hangpara Freezing point
	
	\hangpara Density
	
	\hangpara Dissolved \ce{O2} and \ce{CO2}
	\columnbreak

	\hangpara\begin{tabular}{lr}
	\toprule
	Ion			&	Percent \\	
	\midrule
	\ce{Cl^–}		&	55.04 \\
	\ce{Na+}		&	30.61 \\
	\ce{SO4^–}	&	7.68 \\
	\ce{Mg^2+}	&	3.69 \\
	\ce{Ca^2+}	&	1.16 \\
	\ce{K+}		&	1.10 \\

	\ce{HCO3^–}	&	0.41 \\
	\bottomrule
	\end{tabular}
	
\end{multicols}
\end{frame}

\begin{frame}[t,plain]{Here is your first in-class group assignment.}

	\hangpara Draw a graph that shows how dissolved \ce{O2} (DO) concentration changes as function of depth.

	\hangpara But first, I'll use temperature to show you the proper format.	
	
\end{frame}

%Instructor
\lecture{instructor}{instructor}
{
\usebackgroundtemplate{\includegraphics[width=\paperwidth]{oxygen_profile}}
\begin{frame}[b,plain]{An \highlight{oxygen minimum zone} occurs between 200–1000 m.}
\end{frame}
}

\lecture{student}{student}

%% Carbon buffering
\begin{frame}[t,plain]{\ce{CO2} is an essential part of the bicarbonate buffering system.}

	{\Large \begin{center}
		\ce{CO2 + H2O <=> H2CO3 <=> H+ + HCO3^–}
	\end{center}}

	\hangpara 90\% of inorganic carbon is \ce{HCO3^–}.
	
	\hangpara Ocean pH is \textasciitilde8.2 (7.5–8.4)
	
	\pause\hangpara What happens if atmospheric \ce{CO2} increases?
	
	\hangpara How would this affect marine organisms?
	
\end{frame}

{
\usebackgroundtemplate{\includegraphics[width=\paperwidth]{ocean_basins}}
\begin{frame}[b,plain]{There are five ocean basins.}
\end{frame}
}

{
\usebackgroundtemplate{\includegraphics[width=\paperwidth]{basin_morphology}}
\begin{frame}[b,plain]{The ocean basins share many features.}
\end{frame}
}


\end{document}
