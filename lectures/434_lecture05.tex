%!TEX TS-program = lualatex
%!TEX encoding = UTF-8 Unicode

\documentclass[t]{beamer}

%%%% HANDOUTS For online Uncomment the following four lines for handout
%\documentclass[t,handout]{beamer}  %Use this for handouts.
%\usepackage{handoutWithNotes}
%\pgfpagesuselayout{3 on 1 with notes}[letterpaper,border shrink=5mm]

%\includeonlylecture{student}

%%% Including only some slides for students.
%%% Uncomment the following line. For the slides,
%%% use the labels shown below the command.

%% For students, use \lecture{student}{student}
%% For mine, use \lecture{instructor}{instructor}


%\usepackage{pgf,pgfpages}
%\pgfpagesuselayout{4 on 1}[letterpaper,border shrink=5mm]

% FONTS
\usepackage{fontspec}
\def\mainfont{Linux Biolinum O}
\setmainfont[Ligatures={Common,TeX}, Contextuals={NoAlternate}, BoldFont={* Bold}, ItalicFont={* Italic}, Numbers={OldStyle}]{\mainfont}
\setsansfont[Scale=MatchLowercase, Numbers={OldStyle}]{Linux Biolinum O} 
\usepackage{microtype}

\addfontfeatures{Numbers=SlashedZero}

\usepackage{graphicx}
	\graphicspath{%
	{/Users/mtaylor/Pictures/teach/434/lectures/}%
	{/Users/mtaylor/Pictures/teach/434/handouts/}%
	{/Users/mtaylor/Pictures/teach/common/}%}%
	{img/}} % set of paths to search for images

\usepackage{amsmath,amssymb}

%\usepackage{units}

\usepackage{booktabs}
\usepackage{multicol}
%	\setlength{\columnsep=1em}

%\usepackage{textcomp}
%\usepackage{setspace}
\usepackage{tikz}
	\tikzstyle{every picture}+=[remember picture,overlay]

\mode<presentation>
{
  \usetheme{Lecture}
  \setbeamercovered{invisible}
  \setbeamertemplate{items}[square]
}

\usepackage{hyperref}


\usepackage{calc} % Necessary for hidden word function.
\newcommand\HiddenWord[1]{%
	\alt<handout>{\rule{\widthof{#1}}{\fboxrule}}{#1}%
}


% Use the to temporarily set a background grid for positioning.
%\setbeamertemplate{background}[grid][step=1cm]



\begin{document}

\lecture{student}{student}

{
\usebackgroundtemplate{\includegraphics[width=\paperwidth]{05_biodiversity_ecosystem_function}}
\begin{frame}[b]

	\hfill\tiny \textcolor{white}{wan\_hong, Flickr, \ccbyncsa{2}}
\end{frame}
}
%
{
\usebackgroundtemplate{\includegraphics[width=\paperwidth]{05_ecosystem_functions}}
\begin{frame}[b]{Ecosystem functions affect ecosystem services.}

	\hfill\tiny Strong et al. 2015. Estuarine, Coastal, and Shelf Science 161: 46.
\end{frame}
}
%
{
\usebackgroundtemplate{\includegraphics[width=\paperwidth]{05_what_is_biodiversity}}
\begin{frame}[b]

\hfill\tiny\textcolor{white}{Ocean Networks Canada, Flickr, \ccbyncsa{2}}
\end{frame}
}
%
\begin{frame}[t]{Biodiversity can be assessed by measuring richness and evenness.}

	\hangpara \highlight{Richness:} \onslide<2->{\HiddenWord{the number of species present.}}
	
	\vspace*{1\baselineskip}
	
	\hangpara \highlight{Evenness:} \onslide<3->{\HiddenWord{the relative abundance of each species.}}
	
	\vspace*{1\baselineskip}

	\onslide<4->
	
	\hangpara	The Shannon-Weiner diversity index is a common measure of species diversity: $H' = -\sum{p_i \ln p_i}.$
	
\end{frame}
%
\begin{frame}[t]{Here are two rocky intertidal communities.}
	
	Just by looking, which community do you think is more diverse?
		
	\noindent\includegraphics[width=\textwidth]{05_functional_diversity_communities}
	
	
	\vskip0pt plus 1filll
	
	\hfill\tiny \href{https://jonlefcheck.net/2014/10/20/what-is-functional-diversity-and-why-do-we-care-2/}{\textcopyright Jon Lefcheck}
\end{frame}
%
\begin{frame}[t]{Community 1 has four species:}

	\hspace*{1.5em}Three species of barnacles and a mussel. {\small $H' = 1.39.$}

	\centering\includegraphics[width=0.9\textwidth]{05_functional_diversity_community1}

	\vskip0pt plus 1filll
	
	\hfill\tiny \href{https://jonlefcheck.net/2014/10/20/what-is-functional-diversity-and-why-do-we-care-2/}{\textcopyright Jon Lefcheck}

\end{frame}
%
\begin{frame}[t]{Community 2 has four species:}

	\hspace*{1.5em}A mussel, a seastar, an anemone, and a seagrass.  {\small $H' = 1.39.$}

	\centering\includegraphics[width=0.9\textwidth]{05_functional_diversity_community2}

	\vskip0pt plus 1filll
	
	\hfill\tiny \href{https://jonlefcheck.net/2014/10/20/what-is-functional-diversity-and-why-do-we-care-2/}{\textcopyright Jon Lefcheck}

\end{frame}
%
\begin{frame}[t]{Species diversity treats each species as equally distinct.}

	{\centering\includegraphics[width=0.8\textwidth]{05_equally_distinct}\par}
	
	\vskip0pt plus 1filll
	
	\hfill\tiny \href{https://jonlefcheck.net/2014/10/20/what-is-functional-diversity-and-why-do-we-care-2/}{\textcopyright Jon Lefcheck}
	
\end{frame}
%
\begin{frame}[t]{What if the \highlight{functional diversity} of the species is considered instead?}

	\begin{multicols}{2}
	
	Community 2 has greater functional diversity compared to
	community 1.
	
	\columnbreak
	
	\includegraphics[width=0.45\textwidth]{05_unequally_distinct}
	
	\end{multicols}
	
	\vskip0pt plus 1filll
	
	\hfill\tiny \href{https://jonlefcheck.net/2014/10/20/what-is-functional-diversity-and-why-do-we-care-2/}{\textcopyright Jon Lefcheck}
	
\end{frame}
%
\begin{frame}[t]{Fishes have the greatest species diversity in the Indo-West Pacific.}

	\includegraphics[width=\textwidth]{05_fish_species_diversity}
	
	\vskip0pt plus 1filll 
	
	\hfill\tiny Stuart-Smith et. al. 2013. Nature 501: 539.
	
\end{frame}
%
\begin{frame}[t]{Fishes have the greatest \emph{functional} diversity in East Pacific and Caribbean Sea.}

	\includegraphics[width=\textwidth]{05_fish_functional_diversity}
	
	\vspace*{-0.5\baselineskip}
	
	\hangpara Functional diversity may be more important to \\ecosystem function than species diversity.
	
	\vskip0pt plus 1filll 
	
	\hfill\tiny Stuart-Smith et. al. 2013. Nature 501: 539.
\end{frame}
%
{
\usebackgroundtemplate{\includegraphics[width=\paperwidth]{05_bef_forms}}
\begin{frame}[b]{High \highlight{functional redundancy} results in more efficient resource use.}%The relationship between diversity and function can take many forms.}

	% 
	\hfill\tiny \href{http://dx.doi.org/10.1016/j.ecss.2015.04.008}{Strong et al. 2015. Estuarine, Coastal, and Shelf Science 161: 46.}
\end{frame}
}
%
{
\usebackgroundtemplate{\includegraphics[width=\paperwidth]{05_bef_complementary}}
\begin{frame}[b]{Species \highlight{complementarity} increases ecosystem function.}

\hfill \tiny Fig. 6.3a\,\textcopyright Sinauer Associates, Inc.
\end{frame}
}
%
{
\usebackgroundtemplate{\includegraphics[width=\paperwidth]{05_insurance_hypothesis}}
\begin{frame}[b]{\highlight{Insurance hypothesis:} ecosystem function is maintained in variable environments.}

\hfill\tiny Fig. 6.4\,\textcopyright Sinauer Associates, Inc.
\end{frame}
}
%
\begin{frame}[b]{\highlight{Portfolio effect:} ecosystem function is maintained by most species if any one species declines.}

\centering
\includegraphics[height=0.8\textheight]{05_portfolio_effect}

\hfill\tiny \href{http://www.nature.com/scitable/knowledge/library/biodiversity-and-ecosystem-services-is-it-the-96677163}{\textcopyright Nature Scitable, with permission.}
\end{frame}
%
\begin{frame}[t]{Trophic cascade strength influences extinction effects.}
	\centering
	\includegraphics[height=0.85\textheight]{05_trophic_level_effects}
	
	\vfilll
	
	\hfill \tiny \href{http://www.nature.com/scitable/knowledge/library/biodiversity-and-ecosystem-services-is-it-the-96677163}{\textcopyright Nature Scitable, with permission.}
\end{frame}
%
%
\begin{frame}[t]{Greater diversity may increase resistance to invasive species.}

	\centering\includegraphics[height=0.75\textheight]{05_invasive_species}
	
	\vskip0pt plus 1filll
	
	\hfill \tiny Stachowicz et al. 2002. Ecology 83: 2575.
\end{frame}
%
\begin{frame}[t]{Prediction 1: Biodiversity should increase ecosystem function }
	\begin{multicols}{2}
		where niche partitioning influences coexistence, and
		
		\vspace*{2\baselineskip}
		
		when competitively dominant species make the strongest contribution to ecosystem function.
		
	\columnbreak
		\includegraphics[width=0.45\textwidth]{05_niche_predictionA}
		
	\end{multicols}
	\vskip0pt plus 1filll
	
	\hfill \tiny \href{http://www.int-res.com/articles/meps_oa/m311p179.pdf}{Duffy and Stachowicz 2006. Mar. Ecol. Prog. Ser. 311: 179.}
\end{frame}
%
\begin{frame}[t]{Prediction 2: Ecosystem function should increase over space and time when species vary in response to environmental variation.}
	
	\centering	
	
	\includegraphics[width=0.85\textwidth]{05_time_predictionB}
	
	\vskip0pt plus 1filll
	
	\hfill \tiny \href{http://www.int-res.com/articles/meps_oa/m311p179.pdf}{Duffy and Stachowicz 2006. Mar. Ecol. Prog. Ser. 311: 179.}

\end{frame}
%
\begin{frame}[t]{Diversity affects services through ecosystem functions.}

		\includegraphics[width=\textwidth]{05_diversity_to_service}
		
	\vskip0pt plus 1filll
	
	\hfill\tiny Worm et al. 2006. Science 314: 787.

\end{frame}
%
{
\usebackgroundtemplate{\includegraphics[width=\paperwidth]{05_function_supports_service}}
\begin{frame}[b]{Increased biodiversity can increase ecosystem services.}


\hfill\tiny Fig. 6.12\,\textcopyright Sinauer Associates, Inc.
\end{frame}
}
%
\begin{frame}[t]{Diverse processes increase carbon sequestration and resulting services.}

		\vspace*{-0.85\baselineskip}
		
		\includegraphics[width=\textwidth]{05_carbon_sequestration}
		
	\vskip0pt plus 1filll
	
	\hfill\tiny Snelgrove et al. 2014. Trends Ecol. Evol.  29: 398.

\end{frame}

\end{document}

