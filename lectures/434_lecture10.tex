%!TEX TS-program = lualatex
%!TEX encoding = UTF-8 Unicode

\documentclass[t]{beamer}

%%%% HANDOUTS For online Uncomment the following four lines for handout
%\documentclass[t,handout]{beamer}  %Use this for handouts.
%\usepackage{handoutWithNotes}
%\pgfpagesuselayout{3 on 1 with notes}[letterpaper,border shrink=5mm]

%\includeonlylecture{student}

%%% Including only some slides for students.
%%% Uncomment the following line. For the slides,
%%% use the labels shown below the command.

%% For students, use \lecture{student}{student}
%% For mine, use \lecture{instructor}{instructor}


%\usepackage{pgf,pgfpages}
%\pgfpagesuselayout{4 on 1}[letterpaper,border shrink=5mm]

% FONTS
\usepackage{fontspec}
\def\mainfont{Linux Biolinum O}
\setmainfont[Ligatures={Common,TeX}, Contextuals={NoAlternate}, BoldFont={* Bold}, ItalicFont={* Italic}, Numbers={OldStyle}]{\mainfont}
\setsansfont[Scale=MatchLowercase, Numbers={OldStyle}]{Linux Biolinum O} 
\usepackage{microtype}


\addfontfeatures{Numbers=SlashedZero}

\usepackage{graphicx}
	\graphicspath{%
	{/Users/mtaylor/Pictures/teach/434/lectures/}%
	{/Users/mtaylor/Pictures/teach/348/lectures/}%
	{/Users/mtaylor/Pictures/teach/434/handouts/}%
	{/Users/mtaylor/Pictures/teach/common/}%}%
	{img/}} % set of paths to search for images

\usepackage{xcolor}

\usepackage{amsmath,amssymb}

%\usepackage{units}

\usepackage{siunitx}
%\usepackage{booktabs}
\usepackage{multicol}
%	\setlength{\columnsep=1em}

%\usepackage{chemfig}
%\usepackage[version=4]{mhchem}
\usepackage{tikz}
	\tikzstyle{every picture}+=[remember picture,overlay]

\mode<presentation>
{
  \usetheme{Lecture}
  \setbeamercovered{invisible}
  \setbeamertemplate{items}[default]
  \setbeamercolor{enumerate item}{fg=black}
%\setbeamertemplate{enumerate items}[circle]

}

\usepackage{hyperref}


\usepackage{calc} % Necessary for hidden word function.
\newcommand\HiddenWord[1]{%
	\alt<handout>{\rule{\widthof{#1}}{\fboxrule}}{#1}%
}

% Use the to temporarily set a background grid for positioning.
%\setbeamertemplate{background}[grid][step=1em]



\begin{document}

\lecture{student}{student}

{
\usebackgroundtemplate{\includegraphics[width=\paperwidth]{10_seagrass_ecosystem}}
\begin{frame}[b]{\textcolor{white}{Seagrass ecosystems}}

	\hfill \tiny\textcolor{white}{\textsc{noaa} photo library, Flickr, \ccby{2}}
\end{frame}
}
%
{
\usebackgroundtemplate{\includegraphics[width=\paperwidth]{10_seagrass_foundation}}
\begin{frame}[b]{\textcolor{white}{Seagrasses are foundation species.}}

	\hfill \tiny\textcolor{white}{\textsc{usda}, Flickr, \ccby{2}}
\end{frame}
}
%
{
\usebackgroundtemplate{\includegraphics[width=\paperwidth]{10_seagrass_forage_nursery}}
\begin{frame}[b]{Seagrasses function as home and forage base for some vertebrates\dots}

	\hfill \tiny Orth et al. 2006. \\ \hfill Bioscience 56: 987.
\end{frame}
}
%
{
\usebackgroundtemplate{\includegraphics[width=\paperwidth]{10_seagrass_invertebrate_diversity}}
\begin{frame}[b]{\dots but invertebrates have the greatest diversity.}

	\tiny \textcolor{white}{\textcopyright\,Chronicles of Zostera}  \hfill \textcolor{white}{\textcopyright\,Galice Hoarau, Univ. Nordland}
\end{frame}
}
%
{
\usebackgroundtemplate{\includegraphics[width=\paperwidth]{10_seagrass_substrate}}
\begin{frame}[b]{Seagrasses provide a stable substrate for many organisms.}

	\hfill \tiny\textcopyright\,McGraw-Hill
\end{frame}
}
%
{
\usebackgroundtemplate{\includegraphics[width=\paperwidth]{10_seagrass_productivity}}
\begin{frame}[t]{}

	\vspace*{2\baselineskip}
	
	\hspace{55mm}\parbox{65mm}{\raggedright \Large Seagrass productivity ranges from 800–10\,000 \si{g.C.m^{-2}.yr^{-1}}.\vspace*{\baselineskip} Epiphytes are major producers.}
	
	\vfilll
	
	\tiny Naumann et al. 2013. \textsc{pl}o\textsc{s one} 8(12): e82923 \hfill Moncreiff et al. 1992. Mar. Ecol. Prog. Ser. 87: 161.
\end{frame}
}
%
{
\usebackgroundtemplate{\includegraphics[width=\paperwidth]{10_seagrass_mesograzers}}
\begin{frame}[b]{Mesograzers regulate seagrass ecosystem function.}

	\begin{tikzpicture}
	
		\node [right] at (11,6.6) {\scriptsize direct};
		\node [right] at (11,4.1) {\scriptsize direct};
		\node [right] at (11,1.6) {\scriptsize direct};
		
		\node [rotate=35] at (9.4,1.7){\scriptsize indirect};
		
		\node at (9.5,0.6) {\scriptsize indirect} ;
		\draw [thick,draw=orange6,->] (8,2.5) -- node [above] {{\small \highlight{mesograzers}}} (10,2.5);

		\node [right] at (0,0) {\tiny Figs. 12.5, 12.3\textsc{a} \textcopyright\,Sinauer Associates, Inc.};
	\end{tikzpicture}

\end{frame}
}
%
\begin{frame}[t]{Nutrient addition did not affect seagrass abundance.}
	\includegraphics[width=\textwidth]{10_seagrass_indirect_cascade}
	
	\vspace*{\baselineskip}
	
	This suggests that seagrass ecosystems are driven more by \highlight{mesograzer mutualism.}
	
	\vfilll
	
	\hfill \tiny Fig. 12.5 \textcopyright\,Sinauer Associates, Inc.

\end{frame}
%
{
\begin{frame}[t]{The \highlight{mutualistic mesograzer model} says that mesograzer abundance cascades to affect seagrass abundance.}

	\begin{multicols}{2}
	
	\onslide<2->{%
		\hangpara Algae are competitively dominant to seagrasses. 
	}
	
	\onslide<3->{%
		\hangpara Mesograzers preferentially consume algae.
	}
	
	\onslide<4->{%
		\hangpara Mesograzer abundance \emph{indirectly} determines seagrass dominance.
	}
	
	\columnbreak
	
	\onslide<1->{%
		\includegraphics[height=0.7\textheight]{10_seagrass_simple_web}
	}
	
	\end{multicols}
	
	\vfilll 
	
	\hfill \tiny Fig. 12.3\textsc{a} \textcopyright\,Sinauer Associates, Inc.

	\begin{tikzpicture}
	
		\onslide<3->{%
			\draw [thick,draw=orange6,->] (7.4,3) -- node [above] {{\small \highlight{mesograzers}}} (9.4,3);
		}

		\onslide<4->{%
			\node [rotate=35] at (8.9,2.3){\scriptsize indirect};
			\node at (8.9,0.9) {\scriptsize indirect};
		}

	\end{tikzpicture}

\end{frame}
%
{
\usebackgroundtemplate{\includegraphics[width=\paperwidth]{10_seagrass_complex_web}}
\begin{frame}[b]

	\begin{tikzpicture}
	
		\draw [ultra thick, orange6,->] (2.1,0.6) -- node [above] {mesograzer} node [below] {mutualism}(4.7,0.6);
		

		\node at (1.1,0) {\tiny Fig. 12.3\textsc{b} \textcopyright\,Sinauer};
	\end{tikzpicture}

\end{frame}
}
%
{
\usebackgroundtemplate{\includegraphics[width=\paperwidth]{10_seagrass_mutual_model_complex}}
\begin{frame}[b]

	\hfill \tiny Fig. 12.6\textsc{a} \textcopyright\,Sinauer Associates, Inc.

\end{frame}
}
%
\begin{frame}[t]{Ecosystem function can be modified by other interactions.}

	\includegraphics[width=\textwidth]{10_seagrass_complex_interactions}

	\vfilll

	\hfill \tiny Fig. 12.6 \textcopyright\,Sinauer Associates, Inc.

\end{frame}
%
\lecture{instructor}{instructor}

\begin{frame}{Put away your notebooks for a moment.}

	\pause
	
	\begin{enumerate}
	\item Name the two alternate stable states that can occur in seagrass ecosystems.
			
	\item List three anthropogenic effects that can induce alternate stable states in seagrass beds.
	
	\end{enumerate}
	
\end{frame}

\lecture{student}{student}
%
\begin{frame}[t]{Commercial fishing can alter the cascading effects.}

	\includegraphics[width=\textwidth]{10_seagrass_commercial_fishing_effects}
	
	\vspace*{0.5\baselineskip}
	
	Why were omnivores unaffected?

	\vfilll

	\hfill \tiny Fig. 12.7 \textcopyright\,Sinauer Associates, Inc.

\end{frame}
%
{
\usebackgroundtemplate{\includegraphics[width=\paperwidth]{10_seagrass_alternate_stable_states}}
\begin{frame}[b]{Anthropogenic effects can induce alternate stable states.}

	\hspace*{121mm} \tiny \rotatebox{90}{Fig. 12.10 \textcopyright\,Sinauer Associates, Inc.}

\end{frame}
}
%
\end{document}

