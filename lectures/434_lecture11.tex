%!TEX TS-program = lualatex
%!TEX encoding = UTF-8 Unicode

\documentclass[t]{beamer}

%%%% HANDOUTS For online Uncomment the following four lines for handout
%\documentclass[t,handout]{beamer}  %Use this for handouts.
%\usepackage{handoutWithNotes}
%\pgfpagesuselayout{3 on 1 with notes}[letterpaper,border shrink=5mm]

%\includeonlylecture{student}

%%% Including only some slides for students.
%%% Uncomment the following line. For the slides,
%%% use the labels shown below the command.

%% For students, use \lecture{student}{student}
%% For mine, use \lecture{instructor}{instructor}


%\usepackage{pgf,pgfpages}
%\pgfpagesuselayout{4 on 1}[letterpaper,border shrink=5mm]

% FONTS
\usepackage{fontspec}
\def\mainfont{Linux Biolinum O}
\setmainfont[Ligatures={Common,TeX}, Contextuals={NoAlternate}, BoldFont={* Bold}, ItalicFont={* Italic}, Numbers={OldStyle}]{\mainfont}
\setsansfont[Scale=MatchLowercase, Numbers={OldStyle}]{Linux Biolinum O} 
\usepackage{microtype}


%\addfontfeatures{Numbers=SlashedZero}

\usepackage{graphicx}
	\graphicspath{%
	{/Users/mtaylor/Pictures/teach/434/lectures/}%
	{/Users/mtaylor/Pictures/teach/348/lectures/}%
	{/Users/mtaylor/Pictures/teach/434/handouts/}%
	{/Users/mtaylor/Pictures/teach/common/}%}%
	{img/}} % set of paths to search for images

\usepackage{xcolor}

\usepackage{amsmath,amssymb}

%\usepackage{units}

%\usepackage{siunitx}
%\usepackage{booktabs}
%\usepackage{multicol}
%	\setlength{\columnsep=1em}

%\usepackage{chemfig}
%\usepackage[version=4]{mhchem}
\usepackage{tikz}
	\tikzstyle{every picture}+=[remember picture,overlay]

\mode<presentation>
{
  \usetheme{Lecture}
  \setbeamercovered{invisible}
  \setbeamertemplate{items}[square]
}

%\usepackage{hyperref}


\usepackage{calc} % Necessary for hidden word function.
\newcommand\HiddenWord[1]{%
	\alt<handout>{\rule{\widthof{#1}}{\fboxrule}}{#1}%
}

% Use the to temporarily set a background grid for positioning.
%\setbeamertemplate{background}[grid][step=1em]



\begin{document}

\lecture{student}{student}

{
\usebackgroundtemplate{\includegraphics[width=\paperwidth]{marsh_zonation_intro}}
\begin{frame}[b]{Salt Marsh Ecosystems.}
\hfill\tiny\textcolor{green4}{Source unknown.}
\end{frame}
}
%
\begin{frame}[t]{Review: Biotic and abiotic interactions create zones.}

	\includegraphics[width=\textwidth]{11_marsh_zonation_plants}

	\begin{tikzpicture}
		\draw [|-|] (1em,0) -- node [below] {High marsh} (9.1,0) ;

		\draw [|-|] (9.2,0) -- node [below] {Low marsh} (11.9,0) ;
	\end{tikzpicture}

	\vfilll
	
	\hfill \tiny Fig. 11.2 \textcopyright\,Sinauer Associates, Inc.
\end{frame}
%
{
\usebackgroundtemplate{\includegraphics[width=\paperwidth]{11_marsh_interactions}}
\begin{frame}[b]{Review: Positive and negative interactions maintain ecosystem structure and function.}

	\hfill \tiny Fi.g 11.3 \textcopyright\,Sinauer Associates, Inc.
\end{frame}
}
%
%{
%\usebackgroundtemplate{\includegraphics[width=\paperwidth]{marsh_zonation_abiotic}}
%\begin{frame}[b]{Review}
%\tiny Pennings and Bertness 2001. 
%\end{frame}
%}
%%
%{
%\usebackgroundtemplate{\includegraphics[width=\paperwidth]{marsh_patch_dynamics}}
%\begin{frame}[b]{Review: \highlight{Patch dynamics} are regulated by biotic and abiotic processes.}
%\tiny Pennings and Bertness 2001. 
%\end{frame}
%}
%%
{
\usebackgroundtemplate{\includegraphics[width=\paperwidth]{marsh_predator_prey}}
\begin{frame}[b]{Review: predators affect the distribution of prey.}
\tiny\textcopyright\,Benjamin Pearson.
\end{frame}
}
%
{
\usebackgroundtemplate{\includegraphics[width=\paperwidth]{11_trophic_cascade}}
\begin{frame}[b]

	\hfill \tiny Fi.g 11.7 \textcopyright\,Sinauer Associates, Inc.
\end{frame}
}
%
{
\usebackgroundtemplate{\includegraphics[width=\paperwidth]{11_snail_damage}}
\begin{frame}[b]{Snails can destroy a marsh if not controlled by crabs.}

	\hfill \tiny\textcolor{white}{Fig. 11.8 \textcopyright\,Sinauer Associates, Inc.}
\end{frame}
}
%
\begin{frame}[t]{Alligators are top consumers in some salt marshes.}

	\includegraphics[width=\textwidth]{11_alligator_diet}
	
	Blue crabs are the most common organism in the alligator's diet.

	\vfilll
	
	\hfill \tiny Nifong and Silliman 2013. J. Exp. Mar. Biol. Ecol. 440: 185.
\end{frame}
%
{
\usebackgroundtemplate{\includegraphics[width=\paperwidth]{11_alligator_consumptive_effects}}
\begin{frame}[b]{Alligators can cause consumptive and non-consumptive effects.}

	\tiny Nifong and Silliman 2013. J. Exp. Mar. Biol. Ecol. 440: 185.
\end{frame}
}
%
{
\usebackgroundtemplate{\includegraphics[width=\paperwidth]{11_two_crabs}}
\begin{frame}[b]{How might temperature affect the activity of these two predatory crab species?}


	\tiny \textsc{usgs}, Wikimedia, public domain.\hfill Domicke, Wikimedia, \ccbysa{4}
\end{frame}
}
%

\begin{frame}[t]{How does predation activity vary with temperature, individually and both together?}

	\includegraphics[width=\textwidth]{11_marsh_insurance_hypothesis}
	
	What ecosystem function hypothesis does this support? \\
	\textcolor{gray!50!white}{Hint: redundancy}

	\begin{tikzpicture}
	
	\tikzset{
	mydot/.style={
	  fill,
	  circle,
	  inner sep=1.5pt
	  }
	}
	\tikzset{
	mysquare/.style={
	  fill,
	  rectangle,
	  inner sep=2pt
	  }
	}
		% Left panel
		\draw [thick, dashed, gray] (1.75,2.5) -- node [above] {temp} (2.75,2.5);

		\node [mydot] at (3.1,2.5){};	\node [mydot] at (4.1,2.5){};
		
		\draw [thick] (3.1,2.5) -- node [above] {\textit{Pano.}} (4.1,2.5);

		\node [mysquare, black!50!gray] at (4.45,2.5){};	\node [mysquare, black!50!gray] at (5.45,2.5){};

		\draw [black!50!gray,thick] (4.45,2.5) -- node [above] {\textit{Eury.}} (5.45,2.5);

		% Right panel
		\draw [thick, dashed, gray] (6.75,2.5) -- node [above] {temp} (7.75,2.5);
	
		\node [mydot] at (8,2.5){};	\node [mydot] at (9,2.5){};
		
		\draw [thick] (8,2.5) -- node [above] {both} (9,2.5);


	\end{tikzpicture}

	\vfilll
	
	\hfill \tiny Griffin and Silliman 2011. Biology Letters 7: 79.
\end{frame}
%
{
\usebackgroundtemplate{\includegraphics[width=\paperwidth]{11_crabs_tmii}}
\begin{frame}[b]{Crabs can have negative trait-mediated indirect interactions (\textsc{tmii}s) with plants.}

	\hfill \tiny Griffin et al. 2011. Biology Letters 7: 710.
\end{frame}
}
%
{
\usebackgroundtemplate{\includegraphics[width=\paperwidth]{11_anthropogenic_cascade_modification}}
\begin{frame}[b]{Anthropogenic effects can alter the trophic cascade.}

	\hspace*{120mm} \tiny \rotatebox{90}{Fig. 11.9 \textcopyright\,Sinauer Associates, Inc.}
\end{frame}
}
%
{
\usebackgroundtemplate{\includegraphics[width=\paperwidth]{11_multifunctionality}}
\begin{frame}[b]{Higher functional diversity increased ecosystem function.}

	\tiny Snail (orange); crab (green); fungus (black) \hfill Hensel and Silliman 2013. \textsc{pnas} 110: 20621.
\end{frame}
}
%
\end{document}

