%!TEX TS-program = lualatex
%!TEX encoding = UTF-8 Unicode

%\documentclass[t]{beamer}

%%%% HANDOUTS For online Uncomment the following four lines for handout
\documentclass[t,handout]{beamer}  %Use this for handouts.
\includeonlylecture{student}
\usepackage{handoutWithNotes}
\pgfpagesuselayout{3 on 1 with notes}[letterpaper,border shrink=5mm]


%%% Including only some slides for students.
%%% Uncomment the following line. For the slides,
%%% use the labels shown below the command.

%% For students, use \lecture{student}{student}
%% For mine, use \lecture{instructor}{instructor}


%\usepackage{pgf,pgfpages}
%\pgfpagesuselayout{4 on 1}[letterpaper,border shrink=5mm]

% FONTS
\usepackage{fontspec}
\def\mainfont{Linux Biolinum O}
\setmainfont[Ligatures={Common,TeX}, Contextuals={NoAlternate}, BoldFont={* Bold}, ItalicFont={* Italic}, Numbers={OldStyle}]{\mainfont}
\setsansfont[Scale=MatchLowercase, Numbers={OldStyle}]{Linux Biolinum O} 
\usepackage{microtype}


%\addfontfeatures{Numbers=SlashedZero}

\usepackage{graphicx}
	\graphicspath{%
	{/Users/goby/Pictures/teach/434/lectures/}%
	{/Users/goby/Pictures/teach/348/lectures/}%
	{/Users/goby/Pictures/teach/434/handouts/}%
	{/Users/goby/Pictures/teach/common/}%}%
	{img/}} % set of paths to search for images

\usepackage{xcolor}

\usepackage{amsmath,amssymb}

%\usepackage{units}

%\usepackage{siunitx}
%\usepackage{booktabs}
\usepackage{multicol}
%	\setlength{\columnsep=1em}

%\usepackage{chemfig}
%\usepackage[version=4]{mhchem}
\usepackage{tikz}
	\tikzstyle{every picture}+=[remember picture,overlay]

\mode<presentation>
{
  \usetheme{Lecture}
  \setbeamercovered{invisible}
  \setbeamertemplate{items}[default]
}

%\usepackage{hyperref}


\usepackage{calc} % Necessary for hidden word function.
\newcommand\HiddenWord[1]{%
	\alt<handout>{\rule{\widthof{#1}}{\fboxrule}}{#1}%
}

% Use the to temporarily set a background grid for positioning.
%\setbeamertemplate{background}[grid][step=1em]


\begin{document}

\lecture{student}{student}


{
\usebackgroundtemplate{\includegraphics[width=\paperwidth]{coral_reef_intro}}
\begin{frame}[b]{\hfill\textcolor{white}{What are coral reefs?}}

\hfill \tiny \textcolor{white}{James Watt, \textsc{noaa}}
\end{frame}
}

%
{
\usebackgroundtemplate{\includegraphics[width=\paperwidth]{13_reef_rugosity}}
\begin{frame}[b]{\highlight{Rugosity} creates structural diversity, which supports high species diversity.}

\hfill \tiny \textcolor{white}{National Ocean Service, public domain}
\end{frame}
}
%
{
\usebackgroundtemplate{\includegraphics[width=\paperwidth]{13_reef_diversity}}
\begin{frame}[b]

\hfill \tiny \textcolor{white}{\textsc{noaa} National Ocean Service, Flickr, public domain}
\end{frame}
}

%
{
\usebackgroundtemplate{\includegraphics[width=\paperwidth]{13_rugosity_decline}}
\begin{frame}[b]{Coral rugosity has been declining.}

	\hfill \tiny Fig.~13.9~\copyright\,Sinauer Associates, Inc.

\end{frame}
}

%
{
\usebackgroundtemplate{\includegraphics[width=\paperwidth]{13_caribbean_cover_decline}}
\begin{frame}[b]{Coral cover has been declining. How will this affect diversity on coral reefs?}

	\hfill \tiny Fig.~13.7~\copyright\,Sinauer Associates, Inc.

\end{frame}
}
%%%%%%%%%%%%%%

%{
%\usebackgroundtemplate{\includegraphics[width=\paperwidth]{13_line_islands_global}}
%\begin{frame}[b]
%
%	\begin{tikzpicture}
%
%		\node at (13.8em,16em) {\textcolor{white}{\footnotesize Hawaii}};
%
%		\node [left] at (27em, 20.5em) {\textcolor{white}{\footnotesize Los Angeles}};
%
%		\node [right] at (10em, 8.5em) {\textcolor{yellow}{\small Line Islands}};
%	
%%		\node at (2.7em, 7.4em) {\textcolor{white}{\footnotesize Howland/Baker}};
%
%		\draw[thick, rotate around={42:(9.6em,7.2em)}, yellow] (9.6em,7.2em) ellipse (8pt and 25pt);
%
%%		\draw[thick, white] (2.7 em,6.4em) circle [radius = 5pt];
%	
%%		\draw[thick, white, ->]  (8em,3.7em) node [left] {\footnotesize Jarvis} -- (9.3em, 4.7em);
%
%	\end{tikzpicture}
%\end{frame}
%}
%
{
\usebackgroundtemplate{\includegraphics[width=\paperwidth]{13_line_islands_zoom}}
%\setbeamertemplate{background}[grid][step=1em]
\begin{frame}[b]{\hfill\textcolor{yellow}{Time to answer some questions.}}

	\begin{tikzpicture}

		\node [right] at (5.5em,20.5em) {\textcolor{yellow}{\footnotesize Kingman}};

		\node [right] at (7.3em, 18.5em) {\textcolor{yellow}{\footnotesize Palmyra}};

		\node [right] at (18.8em, 10.5em) {\textcolor{yellow}{\footnotesize Tabuaeran}};

		\node [right] at (27em, 2.5em) {\textcolor{yellow}{\footnotesize Kiritimati}};

%		\draw[thick, white, ->]  (24em,2em) node [left] {\scriptsize Jarvis} -- (24em, 0em);

%		\draw[thick, white, ->]  (2em,12em) node [right] {\scriptsize Howl./Baker}-- (0em, 12em);
		
	\end{tikzpicture}

\end{frame}
}
\lecture{instructor}{instructor}

{
\usebackgroundtemplate{\includegraphics[width=\paperwidth]{13_line_reef_conditions}}
%\setbeamertemplate{background}[grid][step=1em]
\begin{frame}[b]{Which island goes with each pair of images?}

	\begin{tikzpicture}

		\onslide<1>{\node at (7.6em,18.3em) {\Large A};
		\node at (23.8em,18.3em) {\Large B};

		\node at (7.6em, 4.5em) {\Large C};
		\node at (23.8em, 4.5em) {\Large D};}

		\onslide<2>{\node at (7.6em,18.3em) {\Large Kiritimati};
		\node at (23.8em,18.3em) {\Large Kingman};

		\node at (7.6em, 4.5em) {\Large Palmyra};
		\node at (23.8em, 4.5em) {\Large Tabuaeran};}

	\end{tikzpicture}

	\tiny \onslide<2>\href{http://coralreefwatch.noaa.gov/satellite/baa.php}{Coral bleaching animation} \hfill \onslide<1->\tiny Sandin et al. 2008. \textsc{pl}o\textsc{s one} 3(2): e1548.

\end{frame}
}
%
%
{
\usebackgroundtemplate{\includegraphics[width=\paperwidth]{13_phase_shifts}}
\begin{frame}[b]{Does human habitation affect reefs across the Pacific?}

	\hfill \tiny Smith et al.~2016. Proc.~R.~Soc.~B 283: 20151985

\end{frame}
}
%
{
\usebackgroundtemplate{\includegraphics[width=\paperwidth]{13_broader_pacific}}
\begin{frame}[b]

	\hfill \rotatebox{-90}{\tiny Smith et al.~2016. Proc.~R.~Soc.~B 283: 20151985}

\end{frame}
}
%
{
\usebackgroundtemplate{\includegraphics[width=\paperwidth]{13_coral_algal_trends}}
\begin{frame}[b]{What trends can you identify?}

	\hfill \tiny Smith et al.~2016. Proc.~R.~Soc.~B 283: 20151985

\end{frame}
}
%
{
\usebackgroundtemplate{\includegraphics[width=\paperwidth]{13_coral_algal_correlation}}
\begin{frame}[b]{Inhabitated islands are correlated with fleshy algae.}

	\hfill \tiny \rotatebox{-90}{Smith et al.~2016. Proc.~R.~Soc.~B 283: 20151985}

\end{frame}
}
%
{
\usebackgroundtemplate{\includegraphics[width=\paperwidth]{13_sandin_fig5}}
\begin{frame}[b]

	\hfill \tiny Sandin et al. 2008. \textsc{pl}o\textsc{s one} 3(2): e1548.

\end{frame}
}

%
{
\usebackgroundtemplate{\includegraphics[width=\paperwidth]{13_coral_algal_pca}}
\begin{frame}[b]

	\tiny \emph{Never} make a figure like this! \hfill Smith et al.~2016. Proc.~R.~Soc.~B 283: 20151985.

\end{frame}
}
%
\begin{frame}[t]{Which island(s) have the greatest abundance and biomass of top predator and herbivorous fishes?}

	\includegraphics[width=\textwidth]{13_sandin_fig3}

	\vfilll
	
	\hfill \tiny Sandin et al. 2008. \textsc{pl}o\textsc{s one} 3(2): e1548.

\end{frame}
%
{
\usebackgroundtemplate{\includegraphics[width=\paperwidth]{13_stevenson_fig3}}
\begin{frame}[b]{Which fishing type has greatest proportion of large piscivores and herbivores?}

	\tiny \hfill Stevenson et al.~2006. Coral Reefs doi:10.1007/s00338-006-0158-x.

\end{frame}
}
%
%
{
\usebackgroundtemplate{\includegraphics[width=\paperwidth]{13_williams_fig2}}
\begin{frame}[b]{What trends can you identify?}

	\tiny \hfill Williams et al.~2011. J.~Mar.~Bio.~doi:10.1155/2011/826234.

\end{frame}
}
%
{
	\usebackgroundtemplate{\includegraphics[width=\paperwidth]{13_kelly_fig1}}
	\begin{frame}[b]{What trends can you identify?}
	
	\hfill \tiny Kelly et al.~2016.~Oecologia. % doi:10.1007/s00442-016-3724-0.
	
\end{frame}
}
%
{
\usebackgroundtemplate{\includegraphics[width=\paperwidth]{13_kelly_fig5}}
\begin{frame}[b]{Herbivores tended to partition algal resources.}
	
	\hfill \tiny Kelly et al.~2016.~Oecologia. % doi:10.1007/s00442-016-3724-0.
	
\end{frame}
}
%
\begin{frame}[b]{Herbivores fall into distinct functional groups (feeding guides).}

{\centering
	\includegraphics[height=0.8\textheight]{13_kelly_fig2}\par
}

\hfill \tiny Kelly et al.~2016.~Oecologia. % doi:10.1007/s00442-016-3724-0.

\end{frame}
%
%
{
\usebackgroundtemplate{\includegraphics[width=\paperwidth]{13_heenan_fig1}}
	\begin{frame}[b]
	
	\hfill \tiny Heenan et al.~2016.~Proc~R Soc B 283:20161716. % doi:10.1007/s00442-016-3724-0.
	
\end{frame}
}
%
{\usebackgroundtemplate{\includegraphics[width=\paperwidth]{13_edwards_fig4}}
\begin{frame}[b]{Non-fished areas have higher functional diversity of herbivorous fishes.}
	
	\hfill \tiny Edwards et al.~2013.~Proc R Soc B 283:20131835
	
\end{frame}
}
%

%
%\begin{frame}[t]
%	\frametitle<1>{Which island has the greatest size diversity for corals? (upper panel)}
%	\frametitle<2>{Which island has the greatest coral recruitment? Coral disease? (lower panel)}
%
%	\centering \includegraphics[height=0.78\textheight]{13_sandin_fig4}
%
%	\vfilll
%	
%	\hfill \tiny Sandin et al. 2008. \textsc{pl}o\textsc{s one} 3(2): e1548.
%
%\end{frame}
%

% Scotch Diversion

%{
%\usebackgroundtemplate{\includegraphics[width=\paperwidth]{13_scotch_whiskey_profile}}
%\begin{frame}[b]
%
%	\hfill \tiny \textcolor{white}{Wishart 2009. Significance, March.}
%
%\end{frame}
%}
%
%{
%\usebackgroundtemplate{\includegraphics[width=\paperwidth]{13_whiskey_profile}}
%\begin{frame}[b]
%
%	\hfill \tiny Wishart 2009. Significance, March.
%
%\end{frame}
%}
%
%{
%\usebackgroundtemplate{\includegraphics[width=\paperwidth]{13_scotch_whiskey_pca}}
%\begin{frame}[b]
%
%	\hfill \tiny Wishart 2009. Significance, March.
%
%\end{frame}
%}

{

%
\lecture{student}{student}

{
\usebackgroundtemplate{\includegraphics[width=\paperwidth]{13_herbivore_effects}}
\begin{frame}[b]{Herbivores maintain diversity through keystone actions.}

	\hfill \tiny Fig.~13.4~\copyright\,Sinauer Associates, Inc.

\end{frame}
}
%
\begin{frame}[t]{The decline of long-spine sea urchins caused a phase shift in Jamaican reefs.}

	\includegraphics[height=0.78\textheight]{13_urchin_grazing}
	
	\vfilll
	
	\hfill \tiny Precht and Precht 2015. PeerJ Preprints.

\end{frame}

%
\begin{frame}[t]{Urchin recovery is facilitating reef recovery.}

	\includegraphics[width=\textwidth]{13_urchin_recovery}
	
	\vfilll
	
	\hfill \tiny Edmunds and Carpenter 2001. \textsc{pnas} 98: 5067.

\end{frame}
%
\begin{frame}[t]{Coral recruits are present only where urchins are present.}

	{\centering \includegraphics[height=0.82\textheight]{13_urchin_coral_association}
	
	}
	
	\vfilll
	
	\hfill \tiny Edmunds and Carpenter 2001. \textsc{pnas} 98: 5067.
	

\end{frame}
%
{
\usebackgroundtemplate{\includegraphics[width=\paperwidth]{13_alternate_stable_states}}
\begin{frame}[b]

\hfill \tiny \rotatebox{90}{Hoegh-Guldberg et al. 2007. Science 318: 1737}
\end{frame}
}
%
%\begin{frame}[t]{Cleaning species may also increase reef diversity.}
%
%	\includegraphics[width=\textwidth]{13_keystone_cleaners}
%	
%	\vfilll
%	
%	\hfill \tiny Fig.~13.5~\copyright\,Sinauer Associates, Inc.
%	
%
%\end{frame}
%

\end{document}

