%!TEX TS-program = lualatex
%!TEX encoding = UTF-8 Unicode

\documentclass[t]{beamer}

%%%% HANDOUTS For online Uncomment the following four lines for handout
%\documentclass[t,handout]{beamer}  %Use this for handouts.
%\usepackage{handoutWithNotes}
%\pgfpagesuselayout{3 on 1 with notes}[letterpaper,border shrink=5mm]
%	\setbeamercolor{background canvas}{bg=black!5}

%\includeonlylecture{student}

%%% Including only some slides for students.
%%% Uncomment the following line. For the slides,
%%% use the labels shown below the command.

%% For students, use \lecture{student}{student}
%% For mine, use \lecture{instructor}{instructor}


%\usepackage{pgf,pgfpages}
%\pgfpagesuselayout{4 on 1}[letterpaper,border shrink=5mm]

% FONTS
\usepackage{fontspec}
\def\mainfont{Linux Biolinum O}
\setmainfont[Ligatures={Common,TeX}, Contextuals={NoAlternate}, BoldFont={* Bold}, ItalicFont={* Italic}, Numbers={OldStyle}]{\mainfont}
\setsansfont[Scale=MatchLowercase, Numbers={OldStyle}]{Linux Biolinum O} 
\usepackage{microtype}


\addfontfeatures{Numbers=SlashedZero}

\usepackage{graphicx}
	\graphicspath{%
	{/Users/mtaylor/Pictures/teach/434/lectures/}%
	{/Users/mtaylor/Pictures/teach/348/lectures/}%
	{/Users/mtaylor/Pictures/teach/434/handouts/}%
	{/Users/mtaylor/Pictures/teach/common/}%}%
	{img/}} % set of paths to search for images

\usepackage{amsmath,amssymb}

%\usepackage{units}

\usepackage{booktabs}
\usepackage{multicol}
%	\setlength{\columnsep=1em}

%\usepackage{textcomp}
%\usepackage{setspace}
\usepackage{tikz}
	\tikzstyle{every picture}+=[remember picture,overlay]

\mode<presentation>
{
  \usetheme{Lecture}
  \setbeamercovered{invisible}
  \setbeamertemplate{items}[square]
}

\usepackage{hyperref}

\usepackage{calc} % Necessary for hidden word function.
\newcommand\HiddenWord[1]{%
	\alt<handout>{\rule{\widthof{#1}}{\fboxrule}}{#1}%
}

% Use the to temporarily set a background grid for positioning.
%\setbeamertemplate{background}[grid][step=1cm]


\begin{document}

\lecture{student}{student}

{
\usebackgroundtemplate{\includegraphics[width=\paperwidth]{06_coral_spawning}}
\begin{frame}[b]{Larval dispersal and population connectivity.}

	\hfill \tiny \textcolor{white}{Jamie Craggs, Horniman Museum and Gardens.}
\end{frame}
}
%
\begin{frame}[t]{What are the three types of \highlight{planktonic larvae?}}
	\begin{multicols}{2}

	\begin{center}
		\includegraphics[width=95px]{larvae_nonpelagic}
		
		\includegraphics[width=95px]{larvae_lecithotrophic}

		\includegraphics[width=95px]{larvae_planktotrophic}
	\end{center}	
	\columnbreak

		\onslide<2->{Non-pelagic
		
		Lecithotrophic
		
		Planktotrophic
		
		\vspace*{\baselineskip}
		
		Which has the potential to disperse the farthest? Why?}
	
	\end{multicols}
	
	\vfilll
	
	\tiny T-B: \textcopyright\,M. Hellberg; M. Perkins, Flickr CC; NOAA
\end{frame}
%
\begin{frame}[t]{Larval type may correlate with species range size.}

\includegraphics[width=\textwidth]{larval_dispersal}

\vfilll

\tiny \textcopyright\,Levinton, Oxford Univ. Press.
\end{frame}
%
{
\usebackgroundtemplate{\includegraphics[width=\paperwidth]{06_connectivity_extremes}}
\begin{frame}[b]{Population \highlight{connectivity} falls between two extremes.}

	\hfill \tiny Fig. 4.2, \textcopyright Sinauer Associates, Inc.
\end{frame}
}
%
\begin{frame}{$F_\mathrm{ST}$ and variants are measures of connectivity.}

	\hangpara Larval dispersal cannot be observed practically.

	\hangpara Comparison of shared alleles is most common.

	\hangpara $F_{\mathrm{ST}}$ (and related measures) and $\Phi_{\mathrm{ST}}$ compare proportions of shared and private alleles.

	\hangpara $F_{\mathrm{ST}}$ and $\Phi_{\mathrm{ST}}$ range from 0 (no private alleles) to 1 (all private alleles).

	\hangpara $F_{\mathrm{ST}} = \frac{H_\mathrm{T} - H_\mathrm{S}}{H_\mathrm{T}}.$
	

\end{frame}
%

{
\usebackgroundtemplate{\includegraphics[width=\paperwidth]{06_caribbean_epb_prediction}}
\begin{frame}[b]{Which do you predict will have the highest $F_\mathrm{ST}$? Why?}

\end{frame}
}
% PORITES example
{
\usebackgroundtemplate{\includegraphics[width=\paperwidth]{06_epb_porites}}
\begin{frame}[b]{\textit{Porites} is widely distributed across the Pacific Ocean.}

\hfill \tiny\textcolor{white}{Baums Lab, Penn State Univ.}
\end{frame}
}
%
%

{
\usebackgroundtemplate{\includegraphics[width=\paperwidth]{06_porites_structure_results}}
\begin{frame}[b]{\textit{P. lobates} shows strong genetic population structure.}

\tiny Baums et al. 2012.
\end{frame}
}
%
{
\usebackgroundtemplate{\includegraphics[width=\paperwidth]{06_porites_pca_results}}
\begin{frame}[b]

\hfill \tiny Baums et al. 2012. Molecular Ecology 21: 5418.
\end{frame}
}
%
{
\usebackgroundtemplate{\includegraphics[width=\paperwidth]{06_porites_simulation_east}}
\begin{frame}[b]{\textit{P. lobata} cannot disperse across the EPB from the west\dots}

\tiny Wood et al. 2016. Nature Communications 7: 12571.
\end{frame}
}
%
\begin{frame}[t]{\dots but it can disperse across the EPB from the east.}
\centering
\includegraphics[height=0.80\textheight]{06_porites_simulation_west}

	\vfilll
\hfill \tiny Wood et al. 2016. Nature Communications 7: 12571.

\end{frame}

% ELACATINUS EXAMPLE
{
\usebackgroundtemplate{\includegraphics[width=\paperwidth]{06_caribbean_model1}}
\begin{frame}[b]

\hfill \tiny Roberts et al. 1997. Science 278: 5342.

\end{frame}
}
%
{
\usebackgroundtemplate{\includegraphics[width=\paperwidth]{06_caribbean_model2}}
\begin{frame}[b]

\tiny Cowen et al. 2000. Science 287: 5454.

\end{frame}
}
%
\begin{frame}[t]{The models were tested with the sharknose goby, \textit{Elacatinus evelynae.}}
	\centering
	\includegraphics[width=\textwidth]{06_evelynae}
	
	\vfilll
	
	\hfill \tiny \textcopyright Paul Humann. Used with permission.
\end{frame}
%
\begin{frame}[t]{\textit{Elacatinus evelynae} removes parasites from other fishes.}
	\centering
	\includegraphics[width=\textwidth]{06_genie_cleaner}
	
	\vfilll
	
	\hfill \tiny \textcopyright Paul Humann. Used with permission.
\end{frame}
%
\begin{frame}[t]{The adults and eggs are non-dispersing. The larval duration is about 3 weeks.}
		
	\includegraphics[width=0.49\textwidth]{06_oceanops_pair}\hfill
	\includegraphics[width=0.49\textwidth]{06_elacatinus_egg}
	
	\vfilll
	
	\tiny Michael S. Taylor \hfill Colin 1975. \textit{Neon Gobies.}
\end{frame}
%
{
\usebackgroundtemplate{\includegraphics[width=\paperwidth]{06_evelynae_range}}
\begin{frame}[b]

\tiny \textcolor{white}{Colin 1975 and unpublished.}

\end{frame}
}
%
{
\usebackgroundtemplate{\includegraphics[width=\paperwidth]{06_evelynae_results}}
\begin{frame}[b]{\textcolor{white}{\textit{E. evelynae} shows strong genetic population structure.}}

\hfill \tiny \textcolor{white}{Taylor and Hellberg 2002. Science }

\end{frame}
}
%% END EXAMPLES
\begin{frame}[t]{Connectivity can be described by one of three types.}
\centering
\includegraphics[width=\textwidth]{06_population_type_table}

	\vfilll

	\hfill \tiny Table 4.1, \textcopyright Sinauer Associates, Inc.

\end{frame}
%
{
\usebackgroundtemplate{\includegraphics[width=\paperwidth]{06_larval_duration_differentiation}}
\begin{frame}[b]{Larval duration \emph{may} affect population connectivity.}

\hfill \tiny Fig.~4.8 \textcopyright Sinauer Associates, Inc.

\end{frame}
}
%
{
\usebackgroundtemplate{\includegraphics[width=\paperwidth]{06_larval_duration_type1}}
\begin{frame}[b]{Type 1 has the greatest population connectivity.}

\hfill \tiny Fig.~4.8 \textcopyright Sinauer Associates, Inc.

\end{frame}
}
%
{
\usebackgroundtemplate{\includegraphics[width=\paperwidth]{06_larval_duration_type2}}
\begin{frame}[b]{Type 2 has the low population connectivity.}

\hfill \tiny Fig.~4.8 \textcopyright Sinauer Associates, Inc.

\end{frame}
}
%
{
\usebackgroundtemplate{\includegraphics[width=\paperwidth]{06_larval_duration_type3}}
\begin{frame}[b]{Type 3 has has little or no population connectivity.}

\hfill \tiny Fig.~4.8 \textcopyright Sinauer Associates, Inc.

\end{frame}
}
%
{
\usebackgroundtemplate{\includegraphics[width=\paperwidth]{06_dispersal_interactions}}
\begin{frame}[b]{Abiotic and biotic interactions affect connectivity.}

	\tiny Fig. 4.11, \textcopyright Sinauer Associates, Inc.
\end{frame}
}
%

%% CONSERVATION

\begin{frame}[t]{Larval dispersal informs management decisions by:}

	\hangpara identifying the scale of dispersal,
	
	\hangpara identifying the potential for local adaptation,

	\hangpara aiding design of marine protected areas,
	
	\hangpara aiding ecosystem-based management decisions, and
	
	\hangpara identifying local and regional fisheries stocks.
		
\end{frame}

{
\usebackgroundtemplate{\includegraphics[width=\paperwidth]{06_california_mpa_network}}
\begin{frame}[b]

	\tiny ca.gov
\end{frame}
}
%
{
\usebackgroundtemplate{\includegraphics[width=\paperwidth]{06_predator_prey_dispersal}}
\begin{frame}[b]{Management must account for larval dispersal of predator and prey.}

	\hfill \tiny Stier et al. 2014. Nature Communications 5: 6575.
\end{frame}
}
%
{
\usebackgroundtemplate{\includegraphics[width=\paperwidth]{06_fisheries_stocks}}
\begin{frame}[b]{Identify local and regional fisheries stocks.}

	\hfill \tiny Araujo et al. 2014. Trans. Am. Fish. Soc. 143: 479.
\end{frame}
}
%

\end{document}

