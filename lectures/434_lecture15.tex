%!TEX TS-program = lualatex
%!TEX encoding = UTF-8 Unicode

\documentclass[t]{beamer}

%%%% HANDOUTS For online Uncomment the following four lines for handout
%\documentclass[t,handout]{beamer}  %Use this for handouts.
%\includeonlylecture{student}
%\usepackage{handoutWithNotes}
%\pgfpagesuselayout{3 on 1 with notes}[letterpaper,border shrink=5mm]


%%% Including only some slides for students.
%%% Uncomment the following line. For the slides,
%%% use the labels shown below the command.

%% For students, use \lecture{student}{student}
%% For mine, use \lecture{instructor}{instructor}


%\usepackage{pgf,pgfpages}
%\pgfpagesuselayout{4 on 1}[letterpaper,border shrink=5mm]

% FONTS
\usepackage{fontspec}
\def\mainfont{Linux Biolinum O}
%\setmainfont[Ligatures={Common,TeX}, Contextuals={NoAlternate}, BoldFont={* Bold}, ItalicFont={* Italic}, Numbers={OldStyle}]{\mainfont}
\setsansfont[Scale=MatchLowercase, Numbers={OldStyle}, ItalicFont={* Italic}, Ligatures={Common, TeX}]{\mainfont} 
\newfontface\liningnums[Numbers=Lining, Scale=MatchLowercase]{\mainfont}

%\usepackage{microtype}


%\addfontfeatures{Numbers=SlashedZero}

\usepackage{graphicx}
	\graphicspath{%
	{/Users/goby/Pictures/teach/434/lectures/}%
	{/Users/goby/Pictures/teach/348/lectures/}%
%	{/Users/goby/Pictures/teach/434/handouts/}%
%	{/Users/goby/Pictures/teach/common/}%}%
	{img/}} % set of paths to search for images

\usepackage{xcolor}

\usepackage{amsmath,amssymb}

%\usepackage{units}

%\usepackage{siunitx}
\usepackage{booktabs}
\usepackage{multicol}
%	\setlength{\columnsep=1em}

\usepackage{array}
\newcolumntype{L}[1]{>{\raggedright\let\newline\\\arraybackslash\hspace{0pt}}p{#1}}
\newcolumntype{C}[1]{>{\centering\let\newline\\\arraybackslash\hspace{0pt}}p{#1}}
\newcolumntype{R}[1]{>{\raggedleft\let\newline\\\arraybackslash\hspace{0pt}}p{#1}}

%\usepackage{chemfig}
\usepackage[version=4]{mhchem}
\mhchemoptions{textfontcommand=\liningnums}

\usepackage{tikz}
	\tikzstyle{every picture}+=[remember picture,overlay]
	\usetikzlibrary{arrows,calc}

\mode<presentation>
{
  \usetheme{Lecture}
  \setbeamercovered{invisible}
  \setbeamertemplate{items}[default]
}

\usepackage{hyperref}


\usepackage{calc} % Necessary for hidden word function.
\newcommand\HiddenWord[1]{%
	\alt<handout>{\rule{\widthof{#1}}{\fboxrule}}{#1}%
}

% Use the to temporarily set a background grid for positioning.
%\setbeamertemplate{background}[grid][step=1em]


\begin{document}

\lecture{instructor}{instructor}

{
\usebackgroundtemplate{\includegraphics[width=\paperwidth]{14_climate_change_impacts}}
\begin{frame}[b]{Climate change affects many oceanic processes.}

	\tiny Fig.~19.1 \copyright\,Sinauer Associates, Inc.
\end{frame}
}
%
\lecture{student}{student}

\begin{frame}{Global temperatures have always changed, often dramatically.}

	{\centering
		\includegraphics[width=\textwidth]{15_climate_palaeotemps}\par
	}
	
	\bigskip
	
	Note that past temperatures have been much warmer than current temperatures.
	
	\vfilll
	
	\hfill \tiny Glen Fergus, Wikimedia, \ccbysa{2}.  

\end{frame}

\begin{frame}[t]{Global temperatures were relatively steady until recently.}

	{\centering
		\includegraphics[height=0.82\textheight]{15_historic_temperatures}\par
	}
	
	\vfilll
	
	\hfill \tiny Mann et al. 2008. PNAS 105: 13252.  
	
\end{frame}
%
\begin{frame}[t]{Global temperatures are increasing rapidly.}

	\includegraphics[width=\textwidth]{15_historic_temperature_change_rate}

	\invisible<2->{%
	\begin{tikzpicture}
	
		\draw [white, fill=white] (7,0.5) rectangle (12.1,4.5);
		
	\end{tikzpicture}
	}
	\vfilll
	
	\hfill \tiny Tamino 2013, modified Shakun et al. 2012, Marcott et al. 2013

\end{frame}
%
\lecture{instructor}{instructor}

\begin{frame}{About 25\% of \ce{CO2} released since the Industrial Revolution is sequestered in the oceans.}

	\includegraphics[width=\textwidth]{14_industrial_revolution}
	
	\vfilll 
	
	\hfill \tiny D.\,W.\,F.~Hardie, Wikimedia, public domain
	
\end{frame}
%
\lecture{student}{student}

{
\usebackgroundtemplate{\includegraphics[width=\paperwidth]{15_ocean_heat_sink}}
\begin{frame}[b]{The oceans are also absorbing much of the excess heat.}

	\hfill \tiny Fig.~19.2\textsc{a} \copyright\,Sinauer Associates, Inc.

\end{frame}
}
%
{
\usebackgroundtemplate{\includegraphics[width=\paperwidth]{15_ocean_heat_distribution}}
\begin{frame}[b]{The temperature increase is not uniform.}

	\hfill \tiny Fig.~19.2\textsc{b} \copyright\,Sinauer Associates, Inc.

\end{frame}
}
%
{
\usebackgroundtemplate{\includegraphics[width=\paperwidth]{15_heat_distribution_current}}
\begin{frame}[b]

	\hfill \tiny \href{https://climatereanalyzer.org/wx/DailySummary/\#t2anom}{Link to ClimateReanalyzer.org}

\end{frame}
}
%
{
\usebackgroundtemplate{\includegraphics[width=\paperwidth]{15_phenology_distribution}}
\begin{frame}[b]

	\hfill \tiny Fig.~19.10 \copyright\,Sinauer Associates, Inc.

\end{frame}
}
%
\begin{frame}[t]{Thermal stress reduces primary productivity of seagrasses.}
	
	\includegraphics[width=\textwidth]{15_seagrass_thermal_stress}\par
	
	\vfilll
	
	\hfill \tiny Short et al. 2016. Aquatic Botany 135: 3.

\end{frame}
%
\begin{frame}[t]{Warming reduces kelp coverage.}

	\includegraphics[width=\textwidth]{15_kelp_warming}
	
	\vfilll
	
	\hfill \tiny Wernberg et al. 2013. Nature Climate Change 3: 78.
	
\end{frame}
%
\begin{frame}[t]{Phytoplankton community structure may change.}

	\includegraphics[width=\textwidth]{15_plankton_replacement}
	
	\vfilll
	
	\hfill \tiny Hoegh-Guldberg and Bruno 2010. Science 328: 1523.
	
\end{frame}
%
%{
%\usebackgroundtemplate{\includegraphics[width=\paperwidth]{15_plankton_community}}
%\begin{frame}[b]{The entire plankton community structure may change.}
%
%	\hfill \tiny \rotatebox{90}{Hoegh-Guldberg and Bruno 2010. Science 328: 1523.}
%
%\end{frame}
%}
%
\begin{frame}[t]{Larval duration and survivorship may change.}

	\includegraphics[width=\textwidth]{15_larval_duration}
	
	\vfilll
	
	\hfill \tiny Hoegh-Guldberg and Bruno 2010. Science 328: 1523.
	
\end{frame}
%
\begin{frame}[t]{The extent of Arctic sea has decreased steadily.}

	{\centering \includegraphics[height=0.82\textheight]{15_arctic_sea_ice_extent}\par
	}

	\vfilll
	
	\hfill \tiny Box.~19.1\textsc{b} \copyright\,Sinauer Associates, Inc.

\end{frame}
%
{
\usebackgroundtemplate{\includegraphics[width=\paperwidth]{15_arctic_sea_ice_extent_current}}
\begin{frame}[b]

	\tiny \href{http://nsidc.org/arcticseaicenews/charctic-interactive-sea-ice-graph/}{Interactive Graph}

%	\tiny Fig.~19.1 \copyright\,Sinauer Associates, Inc.
\end{frame}
}
%
{
\usebackgroundtemplate{\includegraphics[width=\paperwidth]{15_sea_level_change}}
\begin{frame}[b]{Loss of sea ice and glaciers is causing sea level rise.}

	\hfill \tiny Box.~19.1\textsc{a} \copyright\,Sinauer Associates, Inc.
\end{frame}
}
%
\begin{frame}[t]{Sea levels may rise 2.4–13.1 mm year\textsuperscript{-1} between now and 2100.}
	\includegraphics[width=\textwidth]{15_sea_level_rise_forecast}
	
	\hangpara Some models predict even higher rates. The current rate of increase is about 3.4 mm year\textsuperscript{-1} (\href{http://climate.nasa.gov/vital-signs/sea-level/}{\textsc{nasa}).}
	
	\vfilll
	
	\hfill \tiny Kopp et al. 2016. \textsc{pnas} 113: E1434.
\end{frame}
%
{
\usebackgroundtemplate{\includegraphics[width=\paperwidth]{15_marsh_accretion_rate}}
\begin{frame}[b]{Sea level may rise faster than salt marsh accretion rate.}

	\hfill \tiny FitzGerald et al. 2008. Annu. Rev. Earth Planet. Sci. 36: 601.
\end{frame}
}
%
\begin{frame}[t]{A similar fate may be in store for mangrove forests.}

	\includegraphics[width=\textwidth]{15_mangroves_sea_level_rise}

	\vfilll
	
	\hfill \tiny McIvore et al. 2013. Natural Coastal Protection Series: Report 3.
\end{frame}
%
{
\usebackgroundtemplate{\includegraphics[width=\paperwidth]{thermohaline_circulation}}
\begin{frame}[t]{\hfill The great ocean conveyor is slowing.}

	\vspace*{11\baselineskip}
	
	\hangpara \parbox{45mm}{\raggedright Warming and increased freshwater input stops the north Atlantic water from sinking.}

\end{frame}
}
%
{
\usebackgroundtemplate{\includegraphics[width=\paperwidth]{15_coral_fish_diversity}}
\begin{frame}[b]

	\hfill \tiny Holbrook et al. 2015. \textsc{pl}o\textsc{s one} 10: e0124054.

\end{frame}
}
%
{
\usebackgroundtemplate{\includegraphics[width=\paperwidth]{15_hotspot_diversity}}
\begin{frame}[b]

	\hfill \tiny Holbrook et al. 2015. \textsc{pl}o\textsc{s one} 10: e0124054.

\end{frame}
}
%
\begin{frame}[t]{Decreased coral diversity caused \emph{significant} differences of fish assemblages among sites.}

	\includegraphics[width=\textwidth]{15_hotspot_diversity_CAP}
	
	\vfilll
	
	\hfill \tiny Holbrook et al. 2015. \textsc{pl}o\textsc{s one} 10: e0124054.

\end{frame}
%
\begin{frame}[t]{Acidification and warming interactions affect ecosystem functions.}

	\includegraphics[width=\textwidth]{15_acidification_warming_effects}
	
	\vfilll
	
	\hfill \tiny Nagelkerken and Connell 2015. \textsc{pnas} 112: 13272.

\end{frame}
%
\end{document}

