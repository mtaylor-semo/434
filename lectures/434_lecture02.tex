%!TEX TS-program = lualatex
%!TEX encoding = UTF-8 Unicode

\documentclass[t]{beamer}

%%%% HANDOUTS For online Uncomment the following four lines for handout
%\documentclass[t,handout]{beamer}  %Use this for handouts.
%\includeonlylecture{student}
%\usepackage{handoutWithNotes}
%\pgfpagesuselayout{3 on 1 with notes}[letterpaper,border shrink=5mm]


%% For students, use \lecture{student}{student}
%% For mine, use \lecture{instructor}{instructor}


% FONTS
\usepackage{fontspec}
\def\mainfont{Linux Biolinum O}
\setmainfont[Ligatures={Common,TeX}, Contextuals={NoAlternate}, BoldFont={* Bold}, ItalicFont={* Italic}, Numbers={Proportional}]{\mainfont}
\setsansfont[Scale=MatchLowercase]{Linux Biolinum O} 
\usepackage{microtype}

\usepackage{graphicx}
	\graphicspath{%
	{/Users/goby/Pictures/teach/434/lectures/}%
	{/Users/goby/Pictures/teach/common/}%}%
	{img/}} % set of paths to search for images

\usepackage{amsmath,amssymb}

%\usepackage{units}

\usepackage{booktabs}
\usepackage{multicol}
%	\setlength{\columnsep=1em}

%\usepackage{textcomp}
%\usepackage{setspace}
\usepackage{tikz}
	\tikzstyle{every picture}+=[remember picture,overlay]

\mode<presentation>
{
  \usetheme{Lecture}
  \setbeamercovered{invisible}
  \setbeamertemplate{items}[square]
}

%\usepackage{calc}
\usepackage{hyperref}

\newcommand\HiddenWord[1]{%
	\alt<handout>{\rule{\widthof{#1}}{\fboxrule}}{#1}%
}



\begin{document}

\lecture{instructor}{instructor}

\begin{frame}[c]{Is there method to my madness?}

\begin{center}
	\href{https://www.youtube.com/watch?v=qybUFnY7Y8w}{\includegraphics[width=2cm]{ok_go_logo}}

\end{center}

\end{frame}

\lecture{student}{student}

{
\usebackgroundtemplate{\includegraphics[width=\paperwidth]{02_mississippi_watershed}}
\begin{frame}[b]

\hfill \tiny Wikimedia Commons \ccbysa{3}.
\end{frame}
}

{
\usebackgroundtemplate{\includegraphics[width=\paperwidth]{02_inland_drivers}}
\begin{frame}[b]

\hfill \tiny Fredston-Hermann et al. 2016. Frontiers in Marine Science 3: 273.
\end{frame}
}


{
\usebackgroundtemplate{\includegraphics[width=\paperwidth]{deepwater_horizon}}
\begin{frame}[b]

\hfill\tiny\textcolor{white}{Public domain, Wikimedia Commons.}
\end{frame}
}

{
\usebackgroundtemplate{\includegraphics[width=\paperwidth]{spray_fertilizer}}
\begin{frame}[b]%{Fertilizer in the midwest may do more long-term damage to the Gulf of Mexico.}

\tiny\textcolor{white}{ Bill Meier, Flickr Creative Commons.}
\end{frame}
}

{
\usebackgroundtemplate{\includegraphics[width=\paperwidth]{gulf_mexico_hypoxia}}
\begin{frame}[b]


\hspace*{122mm}\rotatebox{90}{\tiny Rabotyagov et al. 2014.}
\end{frame}
}

{
\usebackgroundtemplate{\includegraphics[width=\paperwidth]{global_hypoxia_areas_2010}}
\begin{frame}[b]{Eutrophic and hypoxic areas are a global problem.}
\hfill\tiny\textcopyright\,World Resources Institute.
\end{frame}
}

\begin{frame}[t]{The size of the GoM dead zone varies annually.}

	\includegraphics[width=\textwidth]{dead_zone_size_graph_2014}
	
	\vskip0pt plus 1filll

\hfill\tiny U.S. EPA.
\end{frame}

\begin{frame}[t]{In 2015, the dead zone was 16,760 km\textsuperscript{2} (6,474 mi\textsuperscript{2}).}

	\includegraphics[width=\textwidth]{dead_zone_2015}
	
	{\large What marine ecosystem services are affected by the processes that cause the Gulf of Mexico dead zone? How are they affected?}
	
	\vskip0pt plus 1filll

\hfill\tiny U.S. EPA.
\end{frame}


\lecture{instructor}{instructor}

{
\usebackgroundtemplate{\includegraphics[width=\paperwidth]{louisiana_coastal_loss}}
\begin{frame}[b]
\hfill\tiny NOAA.
\end{frame}
}

\lecture{student}{student}


{
\usebackgroundtemplate{\includegraphics[width=\paperwidth]{quantify_tradeoffs}}
\begin{frame}[t]{\highlight{Assess trade-offs} among different objectives.}
\end{frame}
}

{
\usebackgroundtemplate{\includegraphics[width=\paperwidth]{quantify_compatibility}}
\begin{frame}[t]{\highlight{Evaluate cumulative impacts} on ecosystem health.}
\end{frame}
}

\begin{frame}[t]{\highlight{Evaluate cumulative impacts} on ecosystem health.}
	\includegraphics[width=\paperwidth]{quantify_compatibility_condensed}
\end{frame}

{
\usebackgroundtemplate{\includegraphics[width=\paperwidth]{quantify_impacts}}
\begin{frame}[b]{Mapping impacts can reveal vulnerable areas.}

\tiny Kappel et al. 2012 \hfill Fig.~18.4
\end{frame}
}

{
\usebackgroundtemplate{\includegraphics[width=\paperwidth]{quantify_health_index}}
\begin{frame}[b]{\highlight{Measure management effectiveness} to identify trends and set goals.}

\tiny\hfill Halpern et al. 2012. Nature 488: 615.; Fig.~18.5
\end{frame}
}

{
\usebackgroundtemplate{\includegraphics[width=\paperwidth]{quantify_trends}}
\begin{frame}[b]{Negative trends may need to be addressed quickly.}

\tiny\hfill Halpern et al. 2012. Nature 488: 615.
\end{frame}
}

{
\usebackgroundtemplate{\includegraphics[width=\paperwidth]{quantify_overall_health}}
\begin{frame}[b]{Coastal ocean health varies widely.}

\tiny\hfill Halpern et al. 2012. Nature 488: 615.
\end{frame}
}


\begin{frame}[t]{More challenges remain for quantifying ecosystem services.}

\hangpara What is the role of biodiversity in supporting ecosystem services?

\hangpara How can we assess ecosystem resilience?

\hangpara How can we link public action with scientific evidence?

\end{frame}

\end{document}
