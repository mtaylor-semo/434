%!TEX TS-program = lualatex
%!TEX encoding = UTF-8 Unicode

\documentclass[t]{beamer}

%%%% HANDOUTS For online Uncomment the following four lines for handout
%\documentclass[t,handout]{beamer}  %Use this for handouts.
%\includeonlylecture{student}
%\usepackage{handoutWithNotes}
%\pgfpagesuselayout{3 on 1 with notes}[letterpaper,border shrink=5mm]

%% For students, use \lecture{student}{student}
%% For mine, use \lecture{instructor}{instructor}

% FONTS
\usepackage{fontspec}
\def\mainfont{Linux Biolinum O}
\setmainfont[Ligatures={Common,TeX}, Contextuals={NoAlternate}, BoldFont={* Bold}, ItalicFont={* Italic}, Numbers={OldStyle}]{\mainfont}
\setsansfont[Scale=MatchLowercase, Numbers={OldStyle}]{Linux Biolinum O} 
\usepackage{microtype}


\addfontfeatures{Numbers=SlashedZero}

\usepackage{graphicx}
	\graphicspath{%
	{/Users/goby/Pictures/teach/434/lectures/}%
	{/Users/goby/Pictures/teach/348/lectures/}%
	{/Users/goby/Pictures/teach/434/handouts/}%
	{/Users/goby/Pictures/teach/common/}%}%
	{img/}} % set of paths to search for images

\usepackage{xcolor}

\usepackage{amsmath,amssymb}

%\usepackage{units}

\usepackage{booktabs}
\usepackage{multicol}
%	\setlength{\columnsep=1em}

\usepackage{chemfig}
\usepackage[version=4]{mhchem}
\usepackage{tikz}
	\tikzstyle{every picture}+=[remember picture,overlay]

\mode<presentation>
{
  \usetheme{Lecture}
  \setbeamercovered{invisible}
  \setbeamertemplate{items}[square]
}

\usepackage{hyperref}


\usepackage{calc} % Necessary for hidden word function.
\newcommand\HiddenWord[1]{%
	\alt<handout>{\rule{\widthof{#1}}{\fboxrule}}{#1}%
}

% Use the to temporarily set a background grid for positioning.
%\setbeamertemplate{background}[grid][step=1cm]



\begin{document}

\lecture{student}{student}

{
\usebackgroundtemplate{\includegraphics[width=\paperwidth]{09_kelp_forest_ecosystems}}
\begin{frame}[b]

	\tiny\textcolor{white}{Peter Southwood, Wikimedia, \ccbysa{3}}
\end{frame}
}
%
\begin{frame}[t]{Review: How do ocean gyres affect the distribution of surface temperatures?}

	\includegraphics[width=\textwidth]{09_ocean_gyres_surface_temperatures}

	\vfilll
	
	\hfill \tiny \textcopyright McGraw-Hill, Inc.

\end{frame}
%
%
\begin{frame}[t]{Describe the distribution of kelp relative to surface water temperature.}

	\includegraphics[width=\textwidth]{09_kelp_distribution}

	\vfilll

	\hfill \tiny \textcopyright Sinauer Associates, Inc.

\end{frame}
%
{
\usebackgroundtemplate{\includegraphics[width=\paperwidth]{09_sublittoral_divisions}}
\begin{frame}[b]{Review: Light penetration divides sublittoral into \highlight{infralittoral} and \highlight{circalittoral} zones.}

	\hfill \tiny \textcopyright Oxford University Press.
\end{frame}
}
%
\lecture{instructor}{instructor}
{
\usebackgroundtemplate{\includegraphics[width=\paperwidth]{09_sublittoral_divisions}}
\begin{frame}[b]{Review: Does light fully explain this pattern?}

	\hfill \tiny \textcopyright Oxford University Press.
\end{frame}
}
%
\lecture{student}{student}
{
\usebackgroundtemplate{\includegraphics[width=\paperwidth]{09_subtidal_urchins_seastars}}
\begin{frame}[b]%{Keystone species also determine the zonation.}

%	\hfill \tiny \textcopyright Oxford University Press.
\end{frame}
}

{
\usebackgroundtemplate{\includegraphics[width=\paperwidth]{09_subtidal_alternate_stable_states}}
\begin{frame}[t]{Review: Kelp-urchin interactions result in \highlight{alternate stable states.}}

	\vfilll
	
	\hfill \tiny \textcopyright McGraw-Hill, Inc.
\end{frame}
}
%
{
\usebackgroundtemplate{\includegraphics[width=\paperwidth]{09_discontinuous_phase_shifts}}
\begin{frame}[b]

	\vfilll
	
	\hfill \tiny Box 14.1 \textcopyright Sinauer Associates, Inc.
\end{frame}
}
%
{
\usebackgroundtemplate{\includegraphics[width=\paperwidth]{09_discontinuous_perturbations}}
\begin{frame}[b]{\highlight{Hysteresis} is the existence of different stable states under the same environmental conditions.}

	\vfilll
	
	\tiny Box 14.1 \textcopyright Sinauer Associates, Inc. \hfill Scheffer and Carpenter. 2003. \textsc{tree} 18: 648.
\end{frame}
}
%
{
\usebackgroundtemplate{\includegraphics[width=\paperwidth]{09_kelp_temporal_trends}}
\begin{frame}[t]{How would these alternate states change due to commercial harvesting of urchins?}

	\vfilll
	
	\hfill \tiny Fig. 14.6 \textcopyright Sinauer Associates, Inc.
\end{frame}
}
%
{
\usebackgroundtemplate{\includegraphics[width=\paperwidth]{09_urchin_harvest}}
\begin{frame}[t]{Kelp abundance and associated diversity increased rapidly after urchin harvest.}

	\vfilll
	
	\hfill \tiny Fig. 14.5 \textcopyright Sinauer Associates, Inc.
\end{frame}
}
%
{
\usebackgroundtemplate{\includegraphics[width=\paperwidth]{09_alaska_trophic_cascade}}
\begin{frame}

	\vfilll
	
	\hfill \tiny Fig. 14.7 \textcopyright Sinauer Associates, Inc.
\end{frame}
}
%
{
\usebackgroundtemplate{\includegraphics[width=\paperwidth]{09_diversity_predator_cascade}}
\begin{frame}[t]{A diverse predator assemblage can stabilize a cascade through consumptive and non-consumptive effects.}

	\vfilll
	
	\hfill \tiny Byrnes et al. 2006. Ecology Letters 9: 61.
\end{frame}
}
%
{
\usebackgroundtemplate{\includegraphics[width=\paperwidth]{09_kelp_cascade_el_nino}}
\begin{frame}[t]{Predator diversity enhances \highlight{resilience.}}

	\vfilll
	
	\tiny Fig. 14.8 \\ \textcopyright Sinauer
\end{frame}
}
%
{
\usebackgroundtemplate{\includegraphics[width=\paperwidth]{09_centrostephanus}}
\begin{frame}[t]{\textit{Centrostephanus} is creating barrens along eastern Tasmania.}

	\definecolor{shadecolor}{RGB}{242,253,252}
	\vfilll
	\hfill \tiny \colorbox{shadecolor}{\textcopyright Scott Ling, University of Tasmania.}
\end{frame}
}
%
{
\usebackgroundtemplate{\includegraphics[width=\paperwidth]{09_tasmania}}
\begin{frame}[b]

	\vfilll
	\tiny tas.gov.au (left). Fig. 14.11 \textcopyright Sinauer Associates, Inc. (right).
\end{frame}
}
%
{
\usebackgroundtemplate{\includegraphics[width=\paperwidth]{09_tasmania_water_temperature}}
\begin{frame}[b]{Water temperature has increased past a critical threshold for survival of urchin larvae.}

	\vfilll
	\hfill \tiny Fig. 14.10 \\ \hfill \textcopyright Sinauer
\end{frame}
}
%
\begin{frame}[t]{Lobster harvesting has removed the urchin's primary predator.}

	\includegraphics[width=\textwidth]{09_tasmania_lobster_catch}
	
	\vfilll
	
	\hfill \tiny Fig. 14.10 \textcopyright Sinauer Associates, Inc.
\end{frame}
%
{
\usebackgroundtemplate{\includegraphics[width=\paperwidth]{09_tasmania_resilience_loss}}
\begin{frame}[b]{Overfishing (and climate change) reduced resilience for an ecosystem with high hysteresis.}

%	\vfilll
	\tiny Ling et al. 2009. \textsc{pnas} 106: 22341. \hfill Selkoe et al. 2015. Ecosystem Health and Sustainability 1: 17.
\end{frame}
}
%

\end{document}

