%!TEX TS-program = lualatex
%!TEX encoding = UTF-8 Unicode

\documentclass[t]{beamer}

%%%% HANDOUTS For online Uncomment the following four lines for handout
%\documentclass[t,handout]{beamer}  %Use this for handouts.
%\includeonlylecture{student}
%\usepackage{handoutWithNotes}
%\pgfpagesuselayout{3 on 1 with notes}[letterpaper,border shrink=5mm]


%%% Including only some slides for students.
%%% Uncomment the following line. For the slides,
%%% use the labels shown below the command.

%% For students, use \lecture{student}{student}
%% For mine, use \lecture{instructor}{instructor}


%\usepackage{pgf,pgfpages}
%\pgfpagesuselayout{4 on 1}[letterpaper,border shrink=5mm]

% FONTS
\usepackage{fontspec}
\def\mainfont{Linux Biolinum O}
%\setmainfont[Ligatures={Common,TeX}, Contextuals={NoAlternate}, BoldFont={* Bold}, ItalicFont={* Italic}, Numbers={OldStyle}]{\mainfont}
\setsansfont[Scale=MatchLowercase, Numbers={OldStyle}, ItalicFont={* Italic}, Ligatures={Common, TeX}]{\mainfont} 
%\newfontface\liningnums[Numbers=Lining, Scale=MatchLowercase]{\mainfont}

%\usepackage{microtype}


%\addfontfeatures{Numbers=SlashedZero}

\usepackage{graphicx}
	\graphicspath{%
	{/Users/mtaylor/Pictures/teach/434/lectures/}%
	{/Users/mtaylor/Pictures/teach/348/lectures/}%
%	{/Users/mtaylor/Pictures/teach/434/handouts/}%
%	{/Users/mtaylor/Pictures/teach/common/}%}%
	{img/}} % set of paths to search for images

\usepackage{xcolor}

\usepackage{amsmath,amssymb}

%\usepackage{units}

%\usepackage{siunitx}
\usepackage{booktabs}
\usepackage{multicol}
%	\setlength{\columnsep=1em}

\usepackage{array}
\newcolumntype{L}[1]{>{\raggedright\let\newline\\\arraybackslash\hspace{0pt}}p{#1}}
\newcolumntype{C}[1]{>{\centering\let\newline\\\arraybackslash\hspace{0pt}}p{#1}}
\newcolumntype{R}[1]{>{\raggedleft\let\newline\\\arraybackslash\hspace{0pt}}p{#1}}

%\usepackage{chemfig}
\usepackage[version=4]{mhchem}
\mhchemoptions{textfontcommand=\liningnums}

\usepackage{tikz}
	\tikzstyle{every picture}+=[remember picture,overlay]
	\usetikzlibrary{arrows,calc}

\mode<presentation>
{
  \usetheme{Lecture}
  \setbeamercovered{invisible}
  \setbeamertemplate{items}[default]
}

\usepackage{hyperref}


\usepackage{calc} % Necessary for hidden word function.
\newcommand\HiddenWord[1]{%
	\alt<handout>{\rule{\widthof{#1}}{\fboxrule}}{#1}%
}

% Use the to temporarily set a background grid for positioning.
%\setbeamertemplate{background}[grid][step=1em]


\begin{document}

%\lecture{instructor}{instructor}

{
\usebackgroundtemplate{\includegraphics[width=\paperwidth]{16_what_lies_under}}
\begin{frame}[b]{Plastics: A Devil's Bargain}

	\tinynofill \href{https://ferdi-rizkiyanto.blogspot.com/2011/06/what-lies-under.html}{\textit{What Lies Under} by Ferdi Rizkiyanto}, \ccbync{} 
\end{frame}
}
%
%\lecture{student}{student}
{
\usebackgroundtemplate{\includegraphics[width=\paperwidth]{16_petrochemicals}}
\begin{frame}

\tinyfill \textcolor{white}{British Plastics Federation}
\end{frame}
}
%
\begin{frame}{Marine plastic pollution is divided into several categories.}
\vspace{-\baselineskip}

\begin{multicols}{2}

\quad \highlight{Size}

\hangpara Macroplastics > 5 mm

\hangpara Microplastics 1–5 mm

\hangpara Nanoplastics < 1 mm

\columnbreak

\highlight{Composition}

\hangpara H: hard plastic, plastic sheet or film,

\hangpara N: plastic lines, ropes, and fishing nets,

\hangpara P: pre-production plastic pellets (cylinder, disk or sphere), and

\hangpara F: foamed material

\end{multicols}

\vfilll

\tiny Mariano et al.~2021. Front.~Toxicol.~3:636640. \textsc{doi}: 10.3389/ftox.2021.636640 \\ Lebreton et al.~2018. Sci.\ Rep.\ 8:4666. \textsc{doi}:10.1038/s41598-018-22939-w
\end{frame}
%

{
\usebackgroundtemplate{\includegraphics[width=\paperwidth]{16_plastic_generation}}
\begin{frame}

\end{frame}
}
%
{
\usebackgroundtemplate{\includegraphics[width=\paperwidth]{16_plastic_emissions}}
\begin{frame}

\end{frame}
}
%
{
\usebackgroundtemplate{\includegraphics[width=\paperwidth]{16_decomposition_rates}}
\begin{frame}

\end{frame}
}
%

\begin{frame}{Garbage patches accumulate in the ocean gyres.}

		\includegraphics[width=\textwidth]{16_garbage_gyres}
	
	\tinyfill Leal Filho, et al.~2021. J.~Mar.~Sci.~Eng., 9, 1289.  

\end{frame}
%

\begin{frame}{The Great Pacific Garbage Patch is the largest patch.}

	\onslide*<1>{
		\includegraphics[width=\textwidth]{16_gpgp}}

	\onslide*<2>{
		\includegraphics[width=\textwidth]{16_gpgp_enhanced}}
	
	\vfilll
	
	\tiny \onslide*<1-2>{Egger et al.~2020. Sci.\ Rep.\ 10:7495 \textsc{doi:} 10.1038/s41598-020-64465-8} \onslide*<2>{\hfill Lebreton et al.~2018. Sci.\ Rep.\ 8:4666. \textsc{doi}:10.1038/s41598-018-22939-w}

\end{frame}
%
{
\usebackgroundtemplate{\includegraphics[width=\paperwidth]{16_surface_plastic_mass}}
\begin{frame}

\end{frame}
}
%
{
\usebackgroundtemplate{\includegraphics[width=\paperwidth]{16_surface_plastic_particles}}
\begin{frame}

\end{frame}
}
%
{
\usebackgroundtemplate{\includegraphics[width=\paperwidth]{16_plastics_collected}}
\begin{frame}

\end{frame}
}
%
\begin{frame}[b]{Plastic density and mass were sampled from HI to CA.}

\includegraphics[width=\textwidth]{16_egger_stations}

\tinyfill Egger et al.~2020. Sci.\ Rep.\ 10:7495

\end{frame}
%
\begin{frame}[b]{Density of plastics was sampled from surface to 2000 m.}

\includegraphics[width=\textwidth]{16_sampled_plastics_depth}

\hangpara Mass is similar.

\tinyfill Egger et al.~2020. Sci.\ Rep.\ 10:7495

\end{frame}
%

\begin{frame}[b]{Plastics extend at low density into the bathypelagic\dots}

\includegraphics[width=\textwidth]{16_egger_depth}

\hangpara Mass is similar.

\tinyfill Egger et al.~2020. Sci.\ Rep.\ 10:7495

\end{frame}
%

\begin{frame}{\dots but do they extend deeper?}
\includegraphics[width=\textwidth]{16_marianas_stations}

\tinyfill Peng et al.\ 2018. Geochem.\ Persp.\ Let.\ 9, 1-5, \textsc{doi:} 10.7185/geochemlet.1829
\end{frame}
%
\begin{frame}{Microplastics ranged from 2.06 to 13.51 pieces/liter in the bottom water\dots}
\includegraphics[width=\textwidth]{16_marianas_ab}

\tinyfill Peng et al.\ 2018. Geochem.\ Persp.\ Let.\ 9, 1-5, \textsc{doi:} 10.7185/geochemlet.1829
\end{frame}
%
\begin{frame}{\dots and from 200 to 2200 pieces/liter in the sediments. \phantom{Just some long words to keep spacing consistent}}
\includegraphics[width=\textwidth]{16_marianas_cd}

\hangpara \highlight{The deep sea floor is likely one of the largest sinks for microplastics.}

\tinyfill Peng et al.\ 2018. Geochem.\ Persp.\ Let.\ 9, 1-5, \textsc{doi:} 10.7185/geochemlet.1829
\end{frame}
%
{
\usebackgroundtemplate{\includegraphics[width=\paperwidth]{16_plastic_effects}}
\begin{frame}

\tinyfill \textcolor{white}{Ocean Conservancy}
\end{frame}
}

{
\usebackgroundtemplate{\includegraphics[width=\paperwidth]{16_scarred_grunt}}
\begin{frame}


%\vfilll

\tinyfill \textcolor{yellow}{\href{https://www.pewtrusts.org/en/research-and-analysis/articles/2018/09/24/plastic-pollution-affects-sea-life-throughout-the-ocean}{\textcopyright\ Karen Doody/Stocktrek Images via \textsc{p}ew}} \end{frame}
}
% 

{
\usebackgroundtemplate{\includegraphics[width=\paperwidth]{16_bird_ingestion}}
\begin{frame}


\vfilll

\tinynofill \textcolor{white}{\href{http://chrisjordan.com/gallery/midway/}{\textcopyright\ Chris Jordan Photographic Arts}} \end{frame}
}
%
 
\begin{frame}{Marine families have greatest risk of ingestion.}

\includegraphics[width=\textwidth]{16_family_ingestion}

\tinyfill Santos et al.\ 2021. Science 373: 56.
\end{frame}
%
 
\begin{frame}{Plastics can magnify up food webs.}

\centering
\includegraphics[height=0.8\textheight]{16_magnification}

\vfilll

\tiny Teal: Documented plastic ingestion. \hfill Santos et al.\ 2021. Science 373: 56.
\end{frame}

\begin{frame}{Microplastics ingested by and adhered to zooplankton.}
\centering
\includegraphics[height=0.85\textheight]{16_ingestion_adhesion}

\tinyfill Cole et al.\ 2013. Environ.\ Sci.\ Technol.\ 47: 6646.

\end{frame}

%

\begin{frame}{Some copepod species take up smaller nanoplastics.}

\includegraphics[width=\textwidth]{16_copepod_uptake}

\hangpara Numbers below bars are particle sizes in µm. Asterisks indicates
significant difference of uptake relative to 7.3µm size class.

\tinyfill Cole et al.\ 2013. Environ.\ Sci.\ Technol.\ 47: 6646.

\end{frame}

%

\begin{frame}{Plastics slow sinking rate of copepod faecal pellets.}

\centering
\includegraphics[height=0.82\textheight]{16_sinking_rates}

\tinyfill Cole et al.\ 2016. Environ.\ Sci.\ Technol.\ 50: 3239.

\end{frame}

%

\begin{frame}{Models suggest plastic export to deep sea is high.}
\includegraphics[width=\textwidth]{16_plastic_export}

\tinyfill Kvale et al.\ 2020. Sci.\ Rep.\ 10:16670.

\end{frame}
%

\begin{frame}{Models suggest plastics decrease oxygen availability.}
\centering
\includegraphics[height=0.82\textheight]{16_oxygen_consumption}

\tinyfill Kvale et al.\ 2021. Nature Comm.\  12:2358.

\end{frame}
%


\begin{frame}{Plastics do most harm at the surface and the sea floor.}
\centering
\includegraphics[height=0.82\textheight]{16_plastics_water_column}

\tinyfill Stubbins et al.\ 2021. Science 373: 51.

\end{frame}
%

{
\usebackgroundtemplate{\includegraphics[width=\paperwidth]{16_great_pacific_closing}}
\begin{frame}

	\vfilll
	
	\tiny \textcolor{green}{10 Rivers 1 Ocean}
\end{frame}
}

\end{document}

