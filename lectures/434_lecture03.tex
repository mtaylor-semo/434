%!TEX TS-program = lualatex
%!TEX encoding = UTF-8 Unicode

\documentclass[t]{beamer}

%%%% HANDOUTS For online Uncomment the following four lines for handout
%\documentclass[t,handout]{beamer}  %Use this for handouts.
%\usepackage{handoutWithNotes}
%\pgfpagesuselayout{3 on 1 with notes}[letterpaper,border shrink=5mm]
%	\setbeamercolor{background canvas}{bg=black!5}


%%% Including only some slides for students.
%%% Uncomment the following line. For the slides,
%%% use the labels shown below the command.
%\includeonlylecture{student}

%% For students, use \lecture{student}{student}
%% For mine, use \lecture{instructor}{instructor}


%\usepackage{pgf,pgfpages}
%\pgfpagesuselayout{4 on 1}[letterpaper,border shrink=5mm]

% FONTS
\usepackage{fontspec}
\def\mainfont{Linux Biolinum O}
\setmainfont[Ligatures={Common,TeX}, Contextuals={NoAlternate}, BoldFont={* Bold}, ItalicFont={* Italic}, Numbers={OldStyle}]{\mainfont}
\setsansfont[Scale=MatchLowercase, Numbers={OldStyle}]{Linux Biolinum O} 
\usepackage{microtype}

\addfontfeatures{Numbers=SlashedZero}

\usepackage{graphicx}
	\graphicspath{%
	{/Users/mtaylor/Pictures/teach/434/lectures/}%
	{/Users/mtaylor/Pictures/teach/common/}%}%
	{img/}} % set of paths to search for images

\usepackage{amsmath,amssymb}

%\usepackage{units}

\usepackage{booktabs}
\usepackage{multicol}
%	\setlength{\columnsep=1em}

%\usepackage{textcomp}
%\usepackage{setspace}
\usepackage{tikz}
	\tikzstyle{every picture}+=[remember picture,overlay]

\mode<presentation>
{
  \usetheme{Lecture}
  \setbeamercovered{invisible}
  \setbeamertemplate{items}[square]
}

%\usepackage{calc}
\usepackage{hyperref}

\newcommand\HiddenWord[1]{%
	\alt<handout>{\rule{\widthof{#1}}{\fboxrule}}{#1}%
}



\begin{document}

{
\usebackgroundtemplate{\includegraphics[width=\paperwidth]{rapture_reef}}
\begin{frame}[b]{\textcolor{white}{Papah\={a}naumoku\={a}kea National Marine Monument was just made the largest \textsc{mpa} in the world.}}

\hfill\tiny\textcolor{white}{Greg McFall, National Ocean Service, Flickr  \textsc{cc by} 2.0.}
\end{frame}
}

{
\usebackgroundtemplate{\includegraphics[width=\paperwidth]{papahanaumokuakea_size}}
\begin{frame}[b]{}


\hfill\tiny\textcolor{white}{National Ocean Service / Pew Charitable Trusts (inset).}
\end{frame}
}


{
\usebackgroundtemplate{\includegraphics[width=\paperwidth]{papahanaumokuakea_plastic}}
\begin{frame}[b]{The \textsc{mpa} suffers from the negative effects from humans.}


\hfill\tiny\textcolor{white}{Kris Krüg, Flickr \textsc{cc by-sa} 2.0.}
\end{frame}
}

{
\usebackgroundtemplate{\includegraphics[width=\paperwidth]{papahanaumokuakea_albatross}}
\begin{frame}[b]{Restoration requires an \textcolor{orange7}{ecosystem-based approach.}}


\hfill\tiny\textcolor{white}{NOAA National Ocean Service, Flickr \textsc{cc by} 2.0.}
\end{frame}
}

\begin{frame}[t]{Ecosystem-based management requires that }

\hangpara goals encompass all ecosystem services,

\hangpara the spatial scale is based on natural boundaries across multiple ecosystems,

\hangpara all sectors of human use are integrated,

\hangpara cumulative effects across sectors are estimated, and

\hangpara strategies adapt over time to account for uncertainty.

\vskip0pt plus 1filll

\hfill\tiny Rosenberg and Sandifer (2009), in \textit{Ecosystem-Based Management for Oceans.}
\end{frame}

%
{
\usebackgroundtemplate{\includegraphics[width=\paperwidth]{ebm_valuing_ecosystem_services}}
\begin{frame}[b]{Management goals should encompass all ecosystem services.}


\tiny UNEP 2013.
\end{frame}
}

%
{
\usebackgroundtemplate{\includegraphics[width=\paperwidth]{ebm_recognizing_connections}}
\begin{frame}[b]{The spatial scale uses natural boundaries across multiple ecosystems.}


\hfill\tiny UNEP 2013.
\end{frame}
}

%
\begin{frame}[t]{All sectors used by humans must be integrated into a cohesive plan.}

\includegraphics[width=\textwidth]{ebm_sector_coordination}

\vskip0pt plus 1filll

\hfill\tiny UNEP 2013.
\end{frame}

%
\begin{frame}[t]{The plan should account for the objectives of each sector.}

\includegraphics[width=\textwidth]{ebm_multiple_objectives}

\vskip0pt plus 1filll

\hfill\tiny UNEP 2013.
\end{frame}

%
\begin{frame}[t]{Predict and account for the accumulative effects of each sector.}

\includegraphics[width=\textwidth]{ebm_cumulative_effects}

\vskip0pt plus 1filll

\hfill\tiny UNEP 2013.
\end{frame}

%
{
\usebackgroundtemplate{\includegraphics[width=\paperwidth]{ebm_phoenix_island_global_location}}
\begin{frame}[b]{The Phoenix Islands Protected Area is the second largest \textsc{mpa} in the world.}

\end{frame}
}

%
{
\usebackgroundtemplate{\includegraphics[width=\paperwidth]{ebm_phoenix_islands_3d}}
\begin{frame}[b]{P\textsc{ipa} is a complex of atolls and sea mounts. What uncertainties might they face?}

\end{frame}
}

%
\lecture{instructor}{instructor}

{
\setbeamercolor{background canvas}{bg=black}
\begin{frame}[b,plain]{}
\end{frame}
}
%
\begin{frame}[t]{Uncertainties include, but are not limited to,}

	\hangpara coral bleaching due to warming waters,
	\pause
	
	\hangpara sea level rise,
	\pause
	
	\hangpara terrestrial vegetation and sea bird colonies vulnerable to salinization of ground water, 
	\pause

	\hangpara ocean acidification, and
	\pause
	
	\hangpara invasive species.
	
\vskip0pt plus 1filll

\hfill\tiny Phoenix Island Protected Area Management Plan 2010--2014.

\end{frame}
%

\end{document}
