%!TEX TS-program = lualatex
%!TEX encoding = UTF-8 Unicode

\documentclass[letterpaper]{tufte-handout}

%\geometry{showframe} % display margins for debugging page layout

\usepackage{fontspec}
\def\mainfont{Linux Libertine O}
\setmainfont[Ligatures={Common,TeX}, Contextuals={NoAlternate}, BoldFont={* Bold}, ItalicFont={* Italic}, Numbers={OldStyle}]{\mainfont}
\setsansfont[Scale=MatchLowercase, Numbers={OldStyle}]{Linux Biolinum O} 
\setmonofont{Linux Libertine O}
\usepackage{microtype}

\usepackage{graphicx} % allow embedded images
  \setkeys{Gin}{width=\linewidth,totalheight=\textheight,keepaspectratio}
  \graphicspath{{/Users/goby/Documents/teach/434/lectures/}} % set of paths to search for images

\usepackage{amsmath}  % extended mathematics
\usepackage{booktabs} % book-quality tables
\usepackage{units}    % non-stacked fractions and better unit spacing
\usepackage{multicol} % multiple column layout facilities
%\usepackage{fancyvrb} % extended verbatim environments
%  \fvset{fontsize=\normalsize}% default font size for fancy-verbatim environments

\usepackage{enumitem}
\usepackage{mhchem}

\makeatletter
% Paragraph indentation and separation for normal text
\renewcommand{\@tufte@reset@par}{%
  \setlength{\RaggedRightParindent}{1.0pc}%
  \setlength{\JustifyingParindent}{1.0pc}%
  \setlength{\parindent}{1pc}%
  \setlength{\parskip}{0pt}%
}
\@tufte@reset@par

% Paragraph indentation and separation for marginal text
\renewcommand{\@tufte@margin@par}{%
  \setlength{\RaggedRightParindent}{0pt}%
  \setlength{\JustifyingParindent}{0.5pc}%
  \setlength{\parindent}{0.5pc}%
  \setlength{\parskip}{0pt}%
}
\makeatother

% Set up the spacing using fontspec features
   \renewcommand\allcapsspacing[1]{{\addfontfeatures{LetterSpace=15}#1}}
   \renewcommand\smallcapsspacing[1]{{\addfontfeatures{LetterSpace=10}#1}}

\title{Chapter 04 discussion guide}

\date{} % without \date command, current date is supplied

\newcommand\lecturefileA{434_lecture06_instructor}

\begin{document}

\maketitle	% this prints the handout title, author, and date

%\printclassoptions
\section*{Larval dispersal and population connectivity}

\begin{enumerate}


	\item How do these potentially affect population connectivity?
	
	\textit{Potentially, the longer the dispersal stage, the greater the connectivity.}
	
	\item What is panmixis? Explain how planktonic larvae, at least conceivably, relates to panmixis of populations.
	
	\textit{Panmixis refers to complete gene flow among all 
	populations. Highly dispersive larvae or a small geographic 
	range can be connect potentially by dispersal.}
	
	\item What do $F_\mathrm{ST}$ \marginnote[-3\baselineskip]{$F_\mathrm{ST} = \frac{H_\mathrm{T}-H_\mathrm{S}}{H_\mathrm{T}}$, where $H_\mathrm{S}$ is the mean heterozygosity among subpopulations and $H_\mathrm{T}$ is the total heterozygosity across all populations. Also, $F_\mathrm{ST} = \frac{1}{4Nm + 1}$ where \textit{N} is population size and \textit{m} is the migration rate.} and similar measures like $\Phi_{\mathrm{ST}}$ measure? 
	
	\item Relate the three types\marginnote{page 74.} of population connectivity from Table~4.1 to Fig.~4.8. Which points in the figure would most likely correspond to a “Type 1” marine population from the table? Which to “Type 2” and ”Type 3”?
	
	%	\begin{marginfigure}
	%	\includegraphics[page=14]{\lecturefileB}
	%	\end{marginfigure}
	
	\textit{\textbf{Type 1} has $F_{ST} = 0$ so are effectively panmictic. These would correspond to the points on or near zero in the figure.}
	
	\textit{\textbf{Type 2} has $F_{ST} \le 0.20$ so are evolutionarily connected but ecologically disconnected. These are in the Waples zone. I chose 0.2 as a somewhat arbitrary value because this corresponds to 1 migrant per generation, independent of population size. These are connected by gene flow enough that they are less likely to speciation but local adaptation can occur.}
	
	\textit{\textbf{Type 3} have high $F_{ST}$ values so are evolutionarily and ecologically independent. There is strong population structure with high potential for speciation.}
	
	\item Explain each\marginnote{Pages 68–72} of the ``key determinants'' discussed and relate them to the diagram in Figure 4.11.\marginnote{Page 78.} 
	
	\textbf{Pelagic duration:} good place to discuss duration\marginnote{Durations of 14-180 days are typical of marine spp.} and connectivity.
	
	\textbf{Larval feeding:} Good place to discuss non-pelagic, lecithotrophic, and planktotrophic larvae and connectivity among populations.
	
	\textbf{Larval behavior}
	
	\textbf{Adult spawning behavior:} currents and other oceanographic conditions can affect larval dispersal. Thus, the timing of spawning relative to the conditions will affect dispersal.
	
	\textbf{Temporal variation in oceanographic conditions:} can be related to adult spawning behavior.
	
	\textbf{Habitat patchiness}
	
	\textbf{Retention vs immigration}
	
	\textbf{Selection for or against immigrants:} Evolutionarily connected, ecological isolated
	
	\textbf{Influence of population size.}

	\item Your group should think\marginnote{See pages 75–78. See also pages 58–61.} about the importance and assumptions of larval dispersal and how they will inform your management plan.
	

	
	
\end{enumerate}

\end{document}