%!TEX TS-program = lualatex
%!TEX encoding = UTF-8 Unicode

\documentclass[letterpaper]{tufte-handout}

%\geometry{showframe} % display margins for debugging page layout

\usepackage{fontspec}
\def\mainfont{Linux Libertine O}
\setmainfont[Ligatures={Common,TeX}, Contextuals={NoAlternate}, BoldFont={* Bold}, ItalicFont={* Italic}, Numbers={OldStyle}]{\mainfont}
\setsansfont[Scale=MatchLowercase, Numbers={OldStyle}]{Linux Biolinum O} 
\setmainfont[SmallCapsFeatures={LetterSpace=0}]{Linux Libertine O}
\setmonofont{Linux Libertine O}
\usepackage{microtype}

\usepackage{graphicx} % allow embedded images
  \setkeys{Gin}{width=\linewidth,totalheight=\textheight,keepaspectratio}
  \graphicspath{{/Users/goby/Documents/teach/434/lectures/}} % set of paths to search for images

\usepackage{amsmath}  % extended mathematics
\usepackage{booktabs} % book-quality tables
\usepackage{units}    % non-stacked fractions and better unit spacing
\usepackage{multicol} % multiple column layout facilities
%\usepackage{fancyvrb} % extended verbatim environments
%  \fvset{fontsize=\normalsize}% default font size for fancy-verbatim environments

\usepackage{enumitem}
%\usepackage{mhchem}

\makeatletter
% Paragraph indentation and separation for normal text
\renewcommand{\@tufte@reset@par}{%
  \setlength{\RaggedRightParindent}{1.0pc}%
  \setlength{\JustifyingParindent}{1.0pc}%
  \setlength{\parindent}{1pc}%
  \setlength{\parskip}{0pt}%
}
\@tufte@reset@par

% Paragraph indentation and separation for marginal text
\renewcommand{\@tufte@margin@par}{%
  \setlength{\RaggedRightParindent}{0pt}%
  \setlength{\JustifyingParindent}{0.5pc}%
  \setlength{\parindent}{0.5pc}%
  \setlength{\parskip}{0pt}%
}
\makeatother

% Set up the spacing using fontspec features
% See https://tex.stackexchange.com/questions/18947/letterspacing-minion-pro-and-tufte-latex-problem-with-running-header
   \renewcommand\allcapsspacing[1]{{\addfontfeatures{LetterSpace=15}#1}}
   \renewcommand\smallcapsspacing[1]{{\addfontfeatures{LetterSpace=10}#1}}



\title{Chapter 06 discussion guide}

\date{} % without \date command, current date is supplied

\newcommand\lecturefile{434_lecture05_instructor}

\begin{document}

\maketitle	% this prints the handout title, author, and date

%\printclassoptions
\section*{Biodiversity and ecosystem function}

\begin{enumerate}

	\item Who are the authors of this chapter? Did you check this time?%
	\marginnote{O'Connor and Byrnes.}
	
	\item Explain\marginnote{A review never hurts.} the difference between richness and evenness. Explain how each relates to typical measures of species diversity (e.g., as measured by the Shannon diversity index).
	
	\item What do O'Connor and Byrnes mean by "diversity and ecosystem function?" What is their specific meaning for "biodiversity?"
	
	\textit{They introduce the concept of functional diversity, meaning species with different ecological roles (e.g., predator and herviore rather than two predators), which increases ecological function.}
		
	\item What is the Michaelis-Menten\marginnote{Fig.~6.1, pg 110. See T.~Rao, \textit{A Curve for all Reasons.}} function known for outside of the example in Figure~6.1?

	\textit{Michaelis Menten is most widely known for enzyme kinematics. As substrate increases, so do enzyme function, but to a maximum. This overall function applies broadly to many areas of biology, such as species-area curves and, here, species richess and ecosystem function.}
	
	\item According to O'Connor and Byrnes, why is functional diversity better than typical measures of species diversity when considering the\,\textsc{bef} model.
	
	\textit{Functional diversity gives a better indication of how well the ecosystem will function instead of just how many species are present. The species present determine ecosystem function so a greater range of resource use translates to higher ecosystem function.}
	
	\item What is overyielding?\marginnote{Defined on page 115. “Output is greater than the sum of its parts.”} Interpret Fig.~6.3, panels~\textsc{a} and~\textsc{b}, in light of this definition.
	
	\textit{Overvielding refers to a higher ecosystem functional output relative to the average output of each species individually. Two types: Panel~\textsc{a} shows transgressive overyielding, where output is higher than highest monoculture. Panel~\textsc{b} is non-transgressive, where max output equals highest monoculture. Interactions and complementarity increase overall function in transgressive situations.}
	

	\item What is complementarity?\marginnote{page 112} What is selection effect? What is redundancy?
	
	\textit{\textbf{Complementarity} refers to differential resource use by coexisting species. Species use complementary parts of niche space rather than duplicate one another. }
	
	\textit{\textbf{Selection effect}\marginnote{Selection here does not refer to natural selection.} refers to random assembly of species affecting ecosystem function. As long as a competitively dominant species with high ecosystem function is present within the assemblage, then function is likely to be high. If the dominant species is not efficient, then function will be reduced.}
		
	\textit{\textbf{Redundancy} refers to assemblages with several species that use the same range of niche space. They are functionally redundant.}

	\begin{marginfigure}
		\includegraphics[page=15]{\lecturefile}
	\end{marginfigure}

	\item Compare and contrast the insurance hypothesis and the portfolio effect. Relate each of these to complementarity.
	
	\textit{\textbf{Insurance hypothesis}\marginnote{Fig.~6.4, pg 113} was proposed specifically for a variable environment. As long as high production species is present over time, ecosystem function will remain. This is more likely when richness is high. The dominant species will likely change as the environment changes.}
	
	\textit{\textbf{Portfolio effect} refers to stability of ecosystem function due when population levels of one species declines. Redundant species tend to maintain ecosystem function. This is independent of variable environments. For example, a population may decline due to disease or frequency-dependent predation pressure.}
	
	\textbf{The two predictions} of\,\textsc{bef}\marginnote{page 113} follow from the above discussion:
		\begin{enumerate}
			\item Functional diversity increases ecosystem function when niche partitioning is present along with a high-function competitively dominant species, and
			\item function increases over space and time when different species vary in their response to a variable environment.
		\end{enumerate}
	
	\begin{marginfigure}
		\includegraphics[page=18]{\lecturefile}
	\end{marginfigure}
	
	
%	\item Why should ecosystem function increase in space and time as species respond to variable environments? Relate this to complementarity.

%	\item Explain how strong versus weak trophic cascades affects maintenance of ecosystem function in the presence of a top-level species extinction.
	
%	\item Explain\marginnote{Read about sampling and selection effects on page 112.} why niche partitioning and competitive dominance in those partitioned niches should increase ecosystem function.
	
%	\item Describe ways that increased ecosystem function can increase ecosystem services. Think of ways for many different types of ecosystems.
	
	\item The focus of marine restoration ecology (previous lecture) is restoration of ecosystem structure and function, not ecosystem services. Based on what you have learned, does restoration ecology exclude improvement of ecosystem services?  Explain.
	
\end{enumerate}

\end{document}