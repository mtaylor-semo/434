Oysters are not always the only foundation species in an oyster reef. Keryn Gedan and her colleagues\footnote{Gedan et al. 2014. Accounting for multiple foundation species in oyster reef restoration benefits. Restoration Ecology 22: 517--524.} studied how the presence of the hooked mussel, (\textit{Ischadium recurvum}), another foundation species, effected the biomass of both species and the ability of both species to remove phytoplankton from the water column. The figure below shows the biomass of the foundation species between restored and control sites (panel a) and the ``clearance rate'' of phytoplankton with and without the mussel. 

\includegraphics[width=\textwidth]{oyster_mussel_biomass}

\question[5]\label{question:project}
How did biomass differ between control and restored sites? Which species actually had greater biomass? Describe how phytoplankton clearance differed between between beds with and without the mussel. 

\vspace*{\stretch{1}}

{\footnotesize Be sure to answer the question on the next page.}

\newpage
