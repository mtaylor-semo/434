Predators of oysters may affect restoration of oyster reefs by reducing the density of individuals on a reef. Joel Fodrie and colleagues\footnote{Fodrie et al. 2014. Classic paradigms in a novel environment: inserting food web and productivity lessons from rocky shores and saltmarshes into biogenic reef restoration. Journal of Applied Ecology 51: 1314--1325.} studied the density of oysters on restored reefs at different depths. They also studied the depth distribution of several predators, including predatory snails and crabs. Some of their results are shown below. Letters to the right of each bar indicate statistical differences. Two bars with the same letter are not significantly different from each other. Two letters (e.g., \textsc{a,b}) indicate that sample is not different from a bar with \textsc{a} or a bar with \textsc{b}. The absence of letters for predatory gastropods indicates that density did not differ significantly with depth.

\includegraphics[width=\textwidth]{oyster_density_predators}

\question[5]\label{question:project}
Describe how oyster density changes with depth for large and small reefs. Propose an explanation for this distribution based on the predators.

\vspace*{\stretch{1}}

{\footnotesize Be sure to answer the question on the next page.}

\newpage
