%!TEX TS-program = lualatex
%!TEX encoding = UTF-8 Unicode

\documentclass[12pt]{article}
%\usepackage{graphicx}
%	\graphicspath{{/Users/goby/Pictures/teach/153/lab/}} % set of paths to search for images

\usepackage{geometry}
\geometry{letterpaper}                   
\geometry{bottom=1in, left=1.5in}
%\geometry{landscape}                % Activate for for rotated page geometry
%\usepackage[parfill]{parskip}    % Activate to begin paragraphs with an empty line rather than an indent
%\usepackage{amssymb}
%\usepackage{mathtools}
%	\everymath{\displaystyle}

%\pagenumbering{gobble}

\usepackage{fontspec}
\setmainfont[Ligatures={Common,TeX}, BoldFont={* Bold}, ItalicFont={* Italic}, Numbers={Proportional}]{Linux Libertine O}
\setsansfont[Scale=MatchLowercase,Ligatures=TeX, Numbers=OldStyle]{Linux Biolinum O}
%\setmonofont[Scale=MatchLowercase]{Inconsolata}
\usepackage{microtype}

\usepackage{unicode-math}
\setmathfont[Scale=MatchLowercase]{Asana-Math.otf}
%\setmathfont{XITS Math}

% To define fonts for particular uses within a document. For example, 
% This sets the Libertine font to use tabular number format for tables.
%\newfontfamily{\tablenumbers}[Numbers={Monospaced}]{Linux Libertine O}
%\newfontfamily{\libertinedisplay}{Linux Libertine Display O}


%\usepackage{booktabs}
%\usepackage{multicol}
%\usepackage{tabularx}
%\usepackage{longtable}
%\usepackage{siunitx}
%\usepackage[justification=raggedright, singlelinecheck=off]{caption}
%\captionsetup{labelsep=period} % Removes colon following figure / table number.
%\captionsetup{tablewithin=none}  % Sequential numbering of tables and figures instead of
%\captionsetup{figurewithin=none} % resetting numbers within each chapter (Intro, M&M, etc.)
%\captionsetup[table]{skip=0pt}

\usepackage{array}
\newcolumntype{L}[1]{>{\raggedright\let\newline\\\arraybackslash\hspace{0pt}}p{#1}}
\newcolumntype{C}[1]{>{\centering\let\newline\\\arraybackslash\hspace{0pt}}p{#1}}
\newcolumntype{R}[1]{>{\raggedleft\let\newline\\\arraybackslash\hspace{0pt}}p{#1}}

\usepackage{enumitem}
%\usepackage{hyperref}
%\usepackage{placeins} %PRovides \FloatBarrier to flush all floats before a certain point.
%\usepackage{hanging}
%\usepackage{color}
%\usepackage{calc}

%\usepackage{titling}
%\setlength{\droptitle}{-60pt}
%\posttitle{\par\end{center}}
%\predate{}\postdate{}

\usepackage[sc]{titlesec}


\usepackage{fancyhdr}
	\setlength{\headheight}{14.5pt}
\fancyhf{}
\pagestyle{fancy}
\lhead{}
\chead{}
\rhead{\footnotesize pg. \thepage }
\renewcommand{\headrulewidth}{0.4pt}

\fancypagestyle{plain}{%
	\fancyhf{}
	\lhead{\textsc{bi} 434: Marine Evolutionary Ecology}
	\rhead{Reprint Requests}
	\renewcommand{\headrulewidth}{0pt}
}
	
%\newcommand{\VSpace}{\vspace{0.5\baselineskip}}
%\newcommand{\BigVSpace}{\vspace{2\baselineskip}}

\title{Reprint Request Assignment}
\author{Marine Evolutionary Ecology}
\date{}                                           % Activate to display a given date or no date

\begin{document}
%\maketitle
\thispagestyle{plain}


\subsection*{Obtain free copies of scientific publications (20 points)}

In this course, you are required
to develop a conservation based extensively on the
scientific literature. Students often search the literature using the
databases available through Kent Library or Google Scholar. These
searches may yield dozens of useful  articles but access to them is often limited to the abstract or first page. This limitation
prevents students from reading and understanding the material fully, 
which causes students to produce substandard results.

Library subscriptions to journals are extremely expensive (often \$2000--\$4000
per year per journal for current issues) which is why small
libraries like Kent Library are limited to few journals.
Fortunately, this limitation need not affect your ability to obtain
current scientific publications. In fact, students and scientists can
easily obtain publication reprints directly from the scientists that
published the research at no cost.

Science is an open process. Scientific research is effectively worthless
unless published for scrutiny by the larger scientific community. This
allows other scientists to evaluate the validity of the methods used to
obtain the results, to inspect the actual results, and to consider the
results in a broader biological context. This open process of peer
review advances science by weeding out subpar research and generating
new and exciting questions for further research. As a byproduct of this
open process, scientists are allowed to freely give away copies of their
publications. These copies are called reprints, which can now be readily
and easily obtained by sending an e-mail to the scientist and asking for
a \textsc{pdf} reprint of the publication.  This assignment teaches you how to obtain a reprint, so that you can
obtain reprints for much of the literature required for your
presentation, and for future assignments and research.

\begin{enumerate}[leftmargin=*]

\item Find one or more journal articles that you need. \emph{For this exercise, the
article you request must have been published in 2014 or later.} You can
find articles through Google
Scholar (a very good way), the Kent Library databases (not the best way), or other types of internet searches. For
this assignment, the article you request should be related to some 
aspect of your conservation plan..

\item Send an e-mail reprint request to the author. Put ``Reprint request'' in the subject field so the authors know what your email is about. Include my
e-mail address (mtaylor@semo.edu) in either the ``bcc'' or ``cc'' field
so that I know you have sent the e-mail. Use the following format for
your e-mail:

\begin{quote}
Dear Dr. \textless{}last name\textgreater{}:\vspace*{0.5\baselineskip}

I would appreciate receiving a \textsc{pdf} reprint of 
\textless{}Title, journal and volume: pages.\textgreater{} \vspace*{0.5\baselineskip}

Thank you for this courtesy, \vspace*{0.5\baselineskip}

\textless{}your name\textgreater{}
\end{quote}

\newpage

For example,

\begin{quote}
Dear Dr. Coates:\vspace*{0.5\baselineskip}

I would appreciate receiving a \textsc{pdf} reprint of %\vspace*{0.5\baselineskip}
Ever since Owen: Changing perspectives on the early evolution of
tetrapods. Annual Review of Ecology, Evolution and Systematics 39:~571--592.\vspace*{0.5\baselineskip}

Thank you for this courtesy,\vspace*{0.5\baselineskip}

Mike Taylor
\end{quote}

\item Forward to me the response that you receive, including the
attached reprint. Usually, you will receive your response within 24–48
hours. Sometimes, a week may elapse before you receive a response and,
unfortunately (and rarely), you may never receive a response. Keep track
of when you sent your e-mail. If you do not receive a response within
seven days, then send a new e-mail reprint request to a different
scientist. You must still cc me and forward the received response.

\emph{Hint}: To be sure of receiving a response within a reasonable time
frame for this exercise, you should consider sending several reprint
requests to different authors for different articles. This ensures that you 
meet the requirements of this assignment as well as help you to gather
the literature necessary for your presentation or report.

\item Points are awarded as you accomplish two tasks:

\begin{enumerate}[leftmargin=\parindent]
\item (10 points) Send your e-mail, cc'd to me. You must send your
request within three days of receiving this assignment to receive the
full 10 points for this part of the assignment. I will deduct 2 points for
each day after the first three days have elapsed. If you do not send a
request within 8 days of receiving this assignment, you will not receive
\emph{any} credit for this assignment.

\item (10 points) Forward the response and reprint to me. When you
receive the reprint from the author, forward both the message and the
attached reprint to me. You should forward this to me within three days
of receipt to receive the full 10 points. I will deduct 2 points per day
for every day beyond your receipt of the reprint. Failure to attach the
reprint will result in a 3 point deduction.
\end{enumerate}
\end{enumerate}

\subsubsection*{Frequently Asked Questions}

\emph{How do I find an author's e-mail address?} Usually, the
abstracting service or journal website will the
author's e-mail address or a link. You can copy and paste the e-mail address or
click on the link to activate your e-mail software. On rare occasions,
the e-mail address may not be included. You can perform a Google search
on the author's name to find the scientist's website, which may include
his or her e-mail address.\vspace*{\baselineskip}

\emph{What if multiple authors are listed on the article?} Generally,
one author is listed as the ``contact author'' or ``corresponding
author.'' The corresponding author is the author most responsible for
the research. Reprint requests, as well as any questions about the
research, are directed to the corresponding author. In most cases, the
corresponding author will be the first author listed (the primary
author) or the last author listed (the senior author). Unless noted
otherwise, the other listed authors contributed to the research but in
lesser ways. Send your e-mail to only the corresponding author.\vspace*{\baselineskip}

\emph{What if the author says a reprint is not available?} For the
purposes of this assignment, send out another e-mail reprint request to
another author for a different article. You can obtain the first article
for free via interlibrary loan from Kent Library.\vspace*{\baselineskip}

\emph{Can I obtain a \textsc{pdf} reprint of anything the author has
published?} Not always. P\textsc{df} reprints are often only available for
journal articles published about 1998 or later. Prior to this, reprints
were available in paper format, which are sometimes still
available by mail from the author. You can also obtain earlier publications
through interlibrary loan at no cost to you.\\ \indent
Scientists sometimes publish research and reviews as a book chapter in an
edited book, like your textbook. Book chapters are sometimes available as a \textsc{pdf} reprint, but
not always. You should inquire with the author whether a \textsc{pdf} reprint is
available. You can also obtain the book via interlibrary loan and then
photocopy the relevant chapter.\vspace*{\baselineskip}

\emph{How else can I find a \textsc{pdf} of an article not available through
Kent Library?} Most often, I copy and paste the title into Google
Scholar. That sometimes returns a link to a downloadable \textsc{pdf} file. If you do this
search on campus, the search will may include a ``get full text'' link. Clicking that will usually provide links to the article through various university-subscribed databases.

If
this doesn't work, a Google search for the author's full name
 may lead to the scientist's web page. Many but not all
scientists list their publications on their web sites, often with links
to \textsc{pdf} files. Click on the link and download the \textsc{pdf} (these are the
links often returned by Google Scholar). This is also a great way to
find related publications from the same author. I also try to search on
the title of the journal article. On occasion, I have stumbled across a
web site with a link to a downloadable \textsc{pdf}.\vspace{\baselineskip}

Despite these hints, you must use the e-mail reprint request method to
complete this assignment.

\end{document}  