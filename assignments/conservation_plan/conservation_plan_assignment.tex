%!TEX TS-program = lualatex
%!TEX encoding = UTF-8 Unicode

\documentclass[12pt, hidelinks]{exam}
\usepackage{graphicx}
	\graphicspath{{/Users/goby/Pictures/teach/434/handouts/}
	{img/}} % set of paths to search for images

\usepackage{geometry}
\geometry{letterpaper, left=1.5in, bottom=1in}                   
%\geometry{landscape}                % Activate for for rotated page geometry
\usepackage[parfill]{parskip}    % Activate to begin paragraphs with an empty line rather than an indent
\usepackage{amssymb, amsmath}
\usepackage{mathtools}
	\everymath{\displaystyle}

\usepackage{fontspec}
\setmainfont[Ligatures={TeX}, BoldFont={* Bold}, ItalicFont={* Italic}, BoldItalicFont={* BoldItalic}, Numbers={OldStyle}]{Linux Libertine O}
\setsansfont[Scale=MatchLowercase,Ligatures=TeX]{Linux Biolinum O}
\setmonofont[Scale=MatchLowercase]{Inconsolatazi4}
\usepackage{microtype}


% To define fonts for particular uses within a document. For example, 
% This sets the Libertine font to use tabular number format for tables.
 %\newfontfamily{\tablenumbers}[Numbers={Monospaced}]{Linux Libertine O}
% \newfontfamily{\libertinedisplay}{Linux Libertine Display O}

\usepackage{booktabs}
\usepackage{multicol}
\usepackage[normalem]{ulem}

\usepackage{longtable}
%\usepackage{siunitx}
\usepackage{array}
\newcolumntype{L}[1]{>{\raggedright\let\newline\\\arraybackslash\hspace{0pt}}p{#1}}
\newcolumntype{C}[1]{>{\centering\let\newline\\\arraybackslash\hspace{0pt}}p{#1}}
\newcolumntype{R}[1]{>{\raggedleft\let\newline\\\arraybackslash\hspace{0pt}}p{#1}}

\usepackage{enumitem}
\usepackage{hyperref}
%\usepackage{placeins} %PRovides \FloatBarrier to flush all floats before a certain point.
\usepackage{hanging}

\usepackage[sc]{titlesec}

%% Commands for Exam class
\renewcommand{\solutiontitle}{\noindent}
\unframedsolutions
\SolutionEmphasis{\bfseries}

\renewcommand{\questionshook}{%
	\setlength{\leftmargin}{-\leftskip}%
}

%Change \half command from 1/2 to .5
\renewcommand*\half{.5}

\pagestyle{headandfoot}
\firstpageheader{\textsc{bi}\,434 MarEvoEco}{}{\ifprintanswers\textbf{KEY}\else Name: \enspace \makebox[2.5in]{\hrulefill}\fi}
\runningheader{}{}{\footnotesize{pg. \thepage}}
\footer{}{}{}
\runningheadrule

\newcommand*\AnswerBox[2]{%
    \parbox[t][#1]{0.92\textwidth}{%
    \begin{solution}#2\end{solution}}
%    \vspace*{\stretch{1}}
}

\newenvironment{AnswerPage}[1]
    {\begin{minipage}[t][#1]{0.92\textwidth}%
    \begin{solution}}
    {\end{solution}\end{minipage}
    \vspace*{\stretch{1}}}

\newlength{\basespace}
\setlength{\basespace}{5\baselineskip}


%\usepackage{mdframed}
%\mdfsetup{%
%	innerleftmargin=0pt,%
%	innerrightmargin=0pt,
%	innertopmargin=0pt,
%	innerbottommargin=0pt,
%	hidealllines=true
%}%end mdfsetup

%
%\makeatletter
%\def\SetTotalwidth{\advance\linewidth by \@totalleftmargin
%\@totalleftmargin=0pt}
%\makeatother


%\printanswers


\begin{document}

\subsection*{Conservation Plan Assignment (150 points)}

During this semester we will discuss several different marine 
ecosystems, and conservation concerns and risks that could 
affect the future of the ecosystems. You will find that
conservation and restoration efforts are important to many of the
ecosystems we will study. To emphasize this, you will be a member of a team
that must design a conservation plan (or restoration plan if appropriate) for an assigned ecosystem.
Each team will consist of a graduate (or honors contract) student serving as team leader. Other undergraduate students will be valuable members of the team.

Your team has been assigned one of these ecosystems.% for conservation or restoration.

\begin{multicols}{2}

Coral Reef\\ 
Kelp Forest\\ 
%Mangrove Forest\\
Salt Marsh\\
Seagrasses

\end{multicols}

Your team will take one of the four approaches outlined in class and explained in chapter 22 of your text.

\begin{multicols}{2}
\begin{enumerate}
\def\labelenumi{\arabic{enumi}.}
\item
  Population level
\item
  Habitat level
\item
  Landscape level
\item
  Ecosystem level
\end{enumerate}
\end{multicols}

Your team must be composed of different types of scientists that each contributes
to the development of the plan. You will need a \emph{unique} position for each student. One student must serve in a public outreach position to educate the public on the importance of ecosystem services and why the conservation plan must be implemented. 

\subsubsection*{Requirements for Public Outreach Official}

This person is required to analyze and understand the ecosystem
services and to portray their importance to the public. Without
investment in the plan from the public sector there is little hope for
funding or success. This person is in charge of designing ways to
pique interest and gain support for your plan. This can be through advertisements, events,
public education, etc. Your conservation plan must outline the ecosystem services and provide
empirical support to justify them. This position will be crucial to
completing that section of your plan.

Examples of other scientists you might need to involve are
oceanographers, biochemists, fish biologists, ecologists, toxicologists,
and more. The positions you choose for your team will depend on your plan and the goals of your plan. Each student in your group must be have a unique role.

The overall assignment for your team is to write a conservation plan designed to solicit public funding. Writing the plan will be a group effort. Each student must make a significant contribution
to the research and writing of the conservation plan. You must support your ideas and overall
goals with empirical evidence from primary research. The graduate or honor student leader will have the
added responsibility of presenting the plan to the class at the end of the semester. 

Within your conservation plan you need to include the following things:

\begin{enumerate}
\def\labelenumi{\arabic{enumi}.}
\item
  Name of primary investigators and supporting investigators
\item
  Project title that includes ecosystem and approach
\item
  Statement of the problem: Ecosystem services and public importance
\item
  Statement of purpose and goals
\item
  Project location and duration
\item
  Plan of action
\item
  Literature cited
\item
  Proposed budget with justification
\end{enumerate}

%I will give you more information on the requirements and grading rubric in separate handouts.

\subsubsection*{Conservation Plan Requirements}

The requirements for your conservation plan are outlined. The required number of pages for each section are estimates. Breakdown of scoring is in a separate rubric.

\begin{questions}
	
	\question[5]
	Title page: 1 page.
	
	The title page must include a project title that includes ecosystem and approach, the name of the primary investigator, and supporting investigators. Each investigator should assume a title (e.g., Chief Biologist, Public Outreach Coordinator) to indicate each team member's role. 
	
	\question[15]
	Introduction: 2--3 pages.
	
	The introduction should introduce the overall conservation plan, including a summary of the habitat, landscape,  or ecosystem, depending on the approach your team is taking.  You should clearly state whether the goal is to conserve or protect existing resources or to restore the ecosystem to ``original'' conditions. Include a summary of the specific region where your plan will be implemented, such as Florida Keys, North Carolina coastal marshes, Florida Bay seagrass beds, California kelp forests, or Florida mangrove forests. I list these only as ideas. You are not restricted to these places or even the United States. You should also describe the anthropogenic effects that have been imposed on the system. Use this to justify the need for the conservation plan. This section should be well supported by the scientific literature (scientific publications,  and technical reports from government agencies and non-governmental organizations). 
	
	
	\question[20]
	Ecosystem services, public sectors, and trade-offs: 2--3 pages.
	
	Identify the ecosystem services and provide a detailed explanation of how the ecosystem provides each service. Include a reasonable estimate of value provided by each service. Identify the public sectors that will have a direct or indirect interest in the conservation plan. Identify potential trade-offs among sectors. Justify which sectors may be negatively affected (or at least not benefit) to the benefit of other sectors. This section should be well supported by the scientific literature.
	
	\question[50]
	Objectives: 5 pages minimum.
	
	You must have a \emph{minimum} of five objectives independent of the public outreach objective (see below). Begin this section with a 1--2 paragraph summary of the objectives that establishes their priority. Then, state each objective, the  goal of the objective, and provide a detailed explanation and justification for each objective. Your objectives may be organized by team member roles or integrated (each team member contributing partially to each objective). Objectives must paint a cohesive picture detailing how the conservation plan will be implemented and expected outcomes of each objective.
	
	Each objective should clearly demonstrate how it will conserve or protect exemplar species, the habitat (foundation or engineer species), or other pertinent biological entities. As appropriate, address abiotic factors such as pollution, erosion control, climate change, etc. Relate your objectives back to the ecosystem services listed earlier. Each objective must be fully supported by the scientific literature. Each objective should detail the expected impact if it cannot be implemented to tie back the priority order in the opening summary for this section.
	
	
	\question[20]
	Public outreach: 2 pages.
	
	This is a separate objective from those above. You should provide detailed descriptions of public outreach goals. Describe several types of outreach activities to help you achieve the goals. May include activities for elementary and secondary schools as well as for adults of all educational levels. Should include outreach to sectors negatively affected, not just those that benefit. 
	
	\question[20]
	Long-term monitoring: 2 pages.
	
	Describe how implementation and success of the plan will be monitored. Describe how success will be measured. A good plan will identify ways that it can adapt to unforeseen circumstances.
	
	\question[10]
	Budget and time frame: 2 pages.
	
	The budget should be justified with reasonable cost estimates. The time frame should be a 5–10 years, in most cases. More of your budget is likely to be spent early in the plan for implementation and public awareness. Less will be spent on monitoring and continuing outreach.
	
	\question[10]
	Literature cited: 40 citations minimum.
	
	Must be consistently formatted following a standard scientific literature cited format. You may follow the format from \emph{one} of your cited articles. Establish your format early so that team members use the same format. 
	
\end{questions}

\subsection*{Other requirements}

\begin{enumerate}[label=\textsc{\alph*}.]
	
	\item Include a final page that details the contribution of each team member. Failure to include this page will result in a 10\% deduction from the total possible points.
	
	\item Use proper spelling, grammar, and mechanics. Excessive mistakes will result in a 5\% deduction from the total possible points.
	
	\item The contributions from each team member must be fully integrated into a cohesive plan. Do not take each member's contribution and place them one after the other (although the objectives may be arranged this way). 
	
	\item Your plan must be typed, single-spaced, and uploaded to the drop box.
	
	\item Ask me questions for anything you unsure about.
	
\end{enumerate}

\textsc{Important note about participation}: This project requires significant effort by all team members. Project leaders should initiate contact with a team member who is not participating. If that fails to resolve the issue, then project leaders should notify me. If I become involved, then that team member will lose 15\% of the total possible
points for this assignment. Each subsequent discussion with the same team member will increase the penalty by another 10\%. 

Do not put your fellow team members in a bind. Be responsible and do your part.
\end{document}  