An important aspect of oyster reef restoration is the type of substrate (rock, old shells, etc.) used to establish the oysters. In addition to affecting oyster density, the substrate may also affect the other types of organisms that recolonize the reef. The type of organisms that could be affected include free-swimming organisms (nekton) and benthic macroinvertebrates that live on the substrate. Laura Brown and her colleagues\footnote{Brown et al. 2014. Oyster reef restoration in the northern Gulf of Mexico: effect of artificial substrate and age on nekton and benthic macroinvertebrate assemblage use. Restoration Ecology 22: 214--222.} tested this by establishing oyster reefs on various substrates across the northern Gulf of Mexico. Some of their results are shown in the figure below. They looked at oyster density (a), and diversity of nekton (b), and benthic macroinvertebrates (c). Letters above each bar indicate statistical differences. Two bars with the same letter are not significantly different from each other. Two letters (e.g., \textsc{ab}) indicate that sample is not different from a bar with \textsc{a} or a bar with \textsc{b}.

\includegraphics[width=\textwidth]{oyster_nekton_macroinverts}

\question[5]\label{question:project}
How was oyster density and diversity of both organismal groups affected by the type of substrate? Explain.

\vspace*{\stretch{1}}

{\footnotesize Be sure to answer the question on the next page.}

\newpage
