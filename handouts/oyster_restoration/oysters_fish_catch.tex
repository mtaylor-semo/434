If the approach to restoration is habitat-based, then restoration of the foundation species should lead to increase abundance of other species. Scyphers et al. also looked at the abundance of the most common fish species at restored and unrestored (control) oyster reefs.  They used 5- and 10-cm gill nets. 5-cm gill nets catch only smaller individuals and 10-cm gill nets catch large individuals. Their results are shown below. CPUE is catch per unit effort. More individuals caught per unit effort (e.g., time net was in the water) indicates greater abundance. Asterisks indicates a significant difference between restored and control sites.

\includegraphics[width=\textwidth]{oyster_fish_catch}

\newpage

\question[5]\label{question:project}
Based on this figure, did restoration change the relative abundance of young or adult fishes between restored and control reefs? Explain differences between restored and control sites and between young and adult fishes. Propose reasons for any observed differences.

\vspace{\stretch{1}}
