%!TEX TS-program = lualatex
%!TEX encoding = UTF-8 Unicode

\documentclass[12pt, addpoints, hidelinks]{exam}
\usepackage{graphicx}
	\graphicspath{{/Users/goby/Pictures/teach/434/handouts/}
	{img/}} % set of paths to search for images

\usepackage{geometry}
\geometry{letterpaper, left=1.5in, bottom=1in}                   
%\geometry{landscape}                % Activate for for rotated page geometry
\usepackage[parfill]{parskip}    % Activate to begin paragraphs with an empty line rather than an indent
\usepackage{amssymb, amsmath}
\usepackage{mathtools}
	\everymath{\displaystyle}

\usepackage{fontspec}
\setmainfont[Ligatures={TeX}, BoldFont={* Bold}, ItalicFont={* Italic}, BoldItalicFont={* BoldItalic}, Numbers={OldStyle}]{Linux Libertine O}
\setsansfont[Scale=MatchLowercase,Ligatures=TeX]{Linux Biolinum O}
\setmonofont[Scale=MatchLowercase]{Inconsolatazi4}
\usepackage{microtype}


% To define fonts for particular uses within a document. For example, 
% This sets the Libertine font to use tabular number format for tables.
 %\newfontfamily{\tablenumbers}[Numbers={Monospaced}]{Linux Libertine O}
% \newfontfamily{\libertinedisplay}{Linux Libertine Display O}

\usepackage{booktabs}
\usepackage{multicol}
\usepackage[normalem]{ulem}

\usepackage{longtable}
%\usepackage{siunitx}
\usepackage{array}
\newcolumntype{L}[1]{>{\raggedright\let\newline\\\arraybackslash\hspace{0pt}}p{#1}}
\newcolumntype{C}[1]{>{\centering\let\newline\\\arraybackslash\hspace{0pt}}p{#1}}
\newcolumntype{R}[1]{>{\raggedleft\let\newline\\\arraybackslash\hspace{0pt}}p{#1}}

\usepackage{enumitem}
\usepackage{hyperref}
%\usepackage{placeins} %PRovides \FloatBarrier to flush all floats before a certain point.
\usepackage{hanging}

\usepackage[sc]{titlesec}

%% Commands for Exam class
\renewcommand{\solutiontitle}{\noindent}
\unframedsolutions
\SolutionEmphasis{\bfseries}

\renewcommand{\questionshook}{%
	\setlength{\leftmargin}{-\leftskip}%
}

%Change \half command from 1/2 to .5
\renewcommand*\half{.5}

\pagestyle{headandfoot}
\firstpageheader{\textsc{bi}\,434 MarEvoEco}{}{\ifprintanswers\textbf{KEY}\else Name: \enspace \makebox[2.5in]{\hrulefill}\fi}
\runningheader{}{}{\footnotesize{pg. \thepage}}
\footer{}{}{}
\runningheadrule

\newcommand*\AnswerBox[2]{%
    \parbox[t][#1]{0.92\textwidth}{%
    \begin{solution}#2\end{solution}}
%    \vspace*{\stretch{1}}
}

\newenvironment{AnswerPage}[1]
    {\begin{minipage}[t][#1]{0.92\textwidth}%
    \begin{solution}}
    {\end{solution}\end{minipage}
    \vspace*{\stretch{1}}}

\newlength{\basespace}
\setlength{\basespace}{5\baselineskip}



%\printanswers


\begin{document}

\subsection*{Oyster reef restoration: a case study (\numpoints\ points)}

Oysters are bivalve mollusks that humans have long exploited for many purposes. Human consume the animal for food and use the shell for buttons, animal food supplements, and more. Humans may have used oysters or other shellfish for at least 150,000 years and have cultivated them for at least 2000 years.\footnote{Beck et al. 2011. Oyster reefs at risk and recommendations for conservation, restoration, and management. BioScience 61: 107--116.}

Oyster beds are large aggregations of oysters that form biogenic reefs under natural conditions. Oyster reefs, or “beds” develop naturally along coastal shores in soft sediments, often in proximity to salt marshes, mangrove forests, seagrass beds, and other estuarine environments. Humans now cultivate large beds oysters for food and other uses. 

Work in groups to answer the following questions. I will collect one question set per group.

\begin{questions}

\question[5]
Is oyster reef restoration taking a population, habitat, landscape, or ecosystem level approach? Think carefully. More than one answer is possible. Explain your reasoning.

\vspace*{\stretch{1}}

\question[5]
Identify ecosystem services provided by oyster reefs. For each service, describe how oyster reefs provide that service. List as many services as your group can describe. Think broadly, accounting for nearby habitats.

\vspace*{\stretch{1}}

\newpage

Cities along the Gulf of Mexico and the Atlantic Coast have to contend with hurricanes, strong storms, and winds that increase the likelihood of coastal erosion and damage to infrastructure. To offset the effects of storms, cities have build seawalls, dykes, and other structures to reduce the energy from wave action and storm surge. This is known as shoreline armoring. The problem is that concrete structures do not dissipate the energy of the waves. Instead, the energy is redirected and can make the problem worse. 

Biogenic structures like oyster reefs naturally reduce the energy of waves as they approach shore. Steven Scyphers and his colleagues\footnote{Scyphers et al. 2011. Oyster reefs as natural breakwaters mitigate shoreline loss and facilitate fisheries. PLoS ONE 6(8):e22396. doi:10.1371/journal.pone.0022396} created natural breakwaters with restored oyster reefs to determine how they affected coastal erosion in Mobile Bay, Alabama. They established two sites, each with a constructed oyster reef breakwater and a control without a breakwater.  They measured water depth (bathymetry) at breakwater and control sites. They also measured vegetation retreat from a reference point. The bathymetry results are shown on screen. The lighter the blue, the shallower the water. The distance of vegetation retreat is shown below.

\includegraphics[width=\textwidth]{oyster_breakwater_results}

\question[5]
Did the the restored reefs change bathymetry or vegetation retreat compared to the control sites? Explain.

\newpage

%% Print enough versions of each one to cover the number of groups. No more than 4 per group.

Predators of oysters may affect restoration of oyster reefs by reducing the density of individuals on a reef. Joel Fodrie and colleagues\footnote{Fodrie et al. 2014. Classic paradigms in a novel environment: inserting food web and productivity lessons from rocky shores and saltmarshes into biogenic reef restoration. Journal of Applied Ecology 51: 1314--1325.} studied the density of oysters on restored reefs at different depths. They also studied the depth distribution of several oyster predators, including snails and crabs. Some of their results are shown below. Letters to the right of each bar indicate statistical differences. Two bars with the same letter are not significantly different from each other. Two letters (e.g., \textsc{a,b}) indicate that sample is not different from a bar with \textsc{a} or a bar with \textsc{b}. The absence of letters for predatory gastropods indicates that density did not differ significantly with depth.

\includegraphics[width=\textwidth]{oyster_density_predators}

\question[5]\label{question:project}
Describe how oyster density changes with depth for large and small reefs. Propose an explanation for this distribution based on the predators.

\vspace*{\stretch{1}}

{\footnotesize Be sure to answer the question on the next page.}

\newpage

%Oysters are not always the only foundation species in an oyster reef. Keryn Gedan and her colleagues\footnote{Gedan et al. 2014. Accounting for multiple foundation species in oyster reef restoration benefits. Restoration Ecology 22: 517--524.} studied how the presence of the hooked mussel, (\textit{Ischadium recurvum}), another foundation species, effected the biomass of both species and the ability of both species to remove phytoplankton from the water column. The figure below shows the biomass of the foundation species between restored and control sites (panel a) and the ``clearance rate'' of phytoplankton with and without the mussel. 

\includegraphics[width=\textwidth]{oyster_mussel_biomass}

\question[5]\label{question:project}
How did biomass differ between control and restored sites? Which species actually had greater biomass? Describe how phytoplankton clearance differed between between beds with and without the mussel. 

\vspace*{\stretch{1}}

{\footnotesize Be sure to answer the question on the next page.}

\newpage

%If the approach to restoration is habitat-based, then restoration of the foundation species should lead to increase abundance of other species. Scyphers et al. also looked at the abundance of the most common fish species at restored and unrestored (control) oyster reefs.  They used 5- and 10-cm gill nets. 5-cm gill nets catch only smaller individuals and 10-cm gill nets catch large individuals. Their results are shown below. CPUE is catch per unit effort. More individuals caught per unit effort (e.g., time net was in the water) indicates greater abundance. Asterisks indicates a significant difference between restored and control sites.

\includegraphics[width=\textwidth]{oyster_fish_catch}

\newpage

\question[5]\label{question:project}
Based on this figure, did restoration change the relative abundance of young or adult fishes between restored and control reefs? Explain differences between restored and control sites and between young and adult fishes. Propose reasons for any observed differences.

\vspace{\stretch{1}}

%As important aspect of marine restoration ecology is ensuring that different areas can recruit new individuals from surrounding areas, a process called connectivity. Choong-Ki Kim and colleagues used computer simulations to predict connectivity of oyster beds in Mobile Bay, Alabama by dispersal of planktotrophic larvae.\footnote{Kim et al. 2013. Establishing restoration strategy of Eastern Oyster via a coupled biophysical transport model. Restoration Ecology 21: 353.} Larval movement among beds was modeled using tidal flow, river discharge, and wind direction for all seasons across the bay. Their study focused on two areas, Cedar Point (\textsc{cp}) and Bon Secour Bay (\textsc{bsb}), located west and east of the bay, respectively. The figure below shows the results of larval transport modeled on tidal flow.

\includegraphics[width=\textwidth]{oyster_larval_transport}

In the figure above, equatorial tides are winter tides. Tropic tides are summer tides. Bars on area boundaries (dashed lines) indicate percent transport of larvae between adjacent areas. Direction of horizontal bars show direction of larval movement.  Bars in the center of an area indicate percent larval retention. For example, during \textsc{t1} (white bars), \textsc{cp} and 10\% larval retention and 71\% larval transport from \textsc{cp} to \textsc{ems}.  Results for river discharge and wind direction are similar. Larval production in \textsc{bsb} was not modeled because the population size is too small. That is why restoration is necessary.

\newpage

\question[5]\label{question:project}
Can restoration efforts in \textsc{bsb} depend on larval transport from \textsc{cp} for the recruitment of new individuals. Explain why or why not.

\vspace{\stretch{1}}
%An important aspect of oyster reef restoration is the type of substrate (rock, old shells, etc.) used to establish the oysters. In addition to affecting oyster density, the substrate may also affect the other types of organisms that recolonize the reef. The type of organisms that could be affected include free-swimming organisms (nekton) and benthic macroinvertebrates that live on the substrate. Laura Brown and her colleagues\footnote{Brown et al. 2014. Oyster reef restoration in the northern Gulf of Mexico: effect of artificial substrate and age on nekton and benthic macroinvertebrate assemblage use. Restoration Ecology 22: 214--222.} tested this by establishing oyster reefs on various substrates across the northern Gulf of Mexico. Some of their results are shown in the figure below. They looked at oyster density (a), and diversity of nekton (b), and benthic macroinvertebrates (c). Letters above each bar indicate statistical differences. Two bars with the same letter are not significantly different from each other. Two letters (e.g., \textsc{ab}) indicate that sample is not different from a bar with \textsc{a} or a bar with \textsc{b}.

\includegraphics[width=\textwidth]{oyster_nekton_macroinverts}

\question[5]\label{question:project}
How was oyster density and diversity of both organismal groups affected by the type of substrate? Explain.

\vspace*{\stretch{1}}

{\footnotesize Be sure to answer the question on the next page.}

\newpage


\question[5]
Assume an agency wanted to attempt restoration of shallow oyster reefs using a habitat-based approach, based on your conclusions above. How do you think the project would be affected by climate change? Think broadly about climate change (temperature, greenhouse gases, etc.) and relate one aspect of climate change specifically to your answer to question \ref{question:project}.

\vspace{\stretch{1}}

\end{questions}

\end{document}  