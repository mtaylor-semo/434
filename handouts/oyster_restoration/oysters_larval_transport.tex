As important aspect of marine restoration ecology is ensuring that different areas can recruit new individuals from surrounding areas, a process called connectivity. Choong-Ki Kim and colleagues used computer simulations to predict connectivity of oyster beds in Mobile Bay, Alabama by dispersal of planktotrophic larvae.\footnote{Kim et al. 2013. Establishing restoration strategy of Eastern Oyster via a coupled biophysical transport model. Restoration Ecology 21: 353.} Larval movement among beds was modeled using tidal flow, river discharge, and wind direction for all seasons across the bay. Their study focused on two areas, Cedar Point (\textsc{cp}) and Bon Secour Bay (\textsc{bsb}), located west and east of the bay, respectively. The figure below shows the results of larval transport modeled on tidal flow.

\includegraphics[width=\textwidth]{oyster_larval_transport}

In the figure above, equatorial tides are winter tides. Tropic tides are summer tides. Bars on area boundaries (dashed lines) indicate percent transport of larvae between adjacent areas. Direction of horizontal bars show direction of larval movement.  Bars in the center of an area indicate percent larval retention. For example, during \textsc{t1} (white bars), \textsc{cp} and 10\% larval retention and 71\% larval transport from \textsc{cp} to \textsc{ems}.  Results for river discharge and wind direction are similar. Larval production in \textsc{bsb} was not modeled because the population size is too small. That is why restoration is necessary.

\newpage

\question[5]\label{question:project}
Can restoration efforts in \textsc{bsb} depend on larval transport from \textsc{cp} for the recruitment of new individuals. Explain why or why not.

\vspace{\stretch{1}}