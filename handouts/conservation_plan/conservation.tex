0 Over the course of the semester we will be discussing several
different marine environments, their ecology, and certain risks for
each. As we go through our lectures you will find that
conservation/restoration efforts are important in many of the
environments we will be studying. Your job as a graduate/honors student
in this course is to design your own conservation/restoration plan with
the help of some of your undergraduate classmates.

Your job will be to determine which ecosystem you feel is the most
important to try and conserve or restore, what type of approach you are
wanting to take, and what type of scientists would be required to make
your conservation plan successful.

Whoops. Did I say you got to choose the ecosystem? Well, I fibbed a
little. Please choose one of the following ecosystems:

\textbf{Salt Marsh Coral Reef Deep Sea Pelagic }

Once you have decided upon an ecosystem, it is important to choose the
type of approach you want to take within your plan. As a reminder here
is a review of the different approaches:

\begin{enumerate}
\def\labelenumi{\arabic{enumi}.}
\item
  Population level
\item
  Habitat level
\item
  Landscape level
\item
  Ecosystem level
\end{enumerate}

If you don't remember discussing these in class please review them in
chapter 22 of your book.

Once you have determined those two things it is time to start
considering which type of scientists (at least 3 -- 4 positions) you'd
need to implement a conservation/restoration plan. These positions will
be filled by undergraduate students in your class. They will be there to
assist you in writing and finalizing your plan. In any plan of this kind
it is extremely important to consider the ecosystem services. Therefore,
I am requiring that one position within your team be a public outreach
position.

\textbf{\emph{Requirements for Public Outreach Official}}

\textbf{This person is required to analyze and understand the ecosystem
services and to portray their importance to the ``public''. Without
somehow getting the public invested in the plan there is little hope for
funding or success. This person is in charge of figuring out ways to
pique interest in your plan. This can be through advertisements, events,
public education, etc. There will be a required section within your
conservation plan that outlines the ecosystem services and provides
empirical support behind them. This position will be crucial to
completing that section of your plan.}

Examples of other scientists you might need to involve are
oceanographers, biochemists, fish biologist, ecologist, toxicologist,
etc. These positions will depend on your plan and the goals of your
plan. Everyone in your group must be designated a specific function.
Yes, that includes you.

So at the end of the day you have your ecosystem, your approach, and the
supporting scientists to help you implement your plan. The overall
assignment for your group is to write a conservation plan proposal which
you as the graduate student and leading investigator will need to
present to the class and persuade your classmates your proposal deserves
funding over the others because in the science world nothing can be done
without funding.

Within your conservation plan you need to include the following things:

\begin{enumerate}
\def\labelenumi{\arabic{enumi}.}
\item
  Name of primary investigators and supporting investigators
\item
  Project title that includes ecosystem and approach
\item
  Statement of the problem: Ecosystem services and public importance
\item
  Statement of purpose and goals
\item
  Project location and duration
\item
  Plan of action
\item
  Literature cited
\item
  Proposed budget with justification
\end{enumerate}

Writing the proposal will be a group effort while presenting the plan to
the class will be the responsibility of primary investigators. It is
important to support your plan with empirical evidence from primary
research throughout the proposal to give your group credibility.
