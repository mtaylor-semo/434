%!TEX TS-program = lualatex
%!TEX encoding = UTF-8 Unicode

\documentclass[12pt, hidelinks]{exam}
\usepackage{graphicx}
	\graphicspath{{/Users/goby/Pictures/teach/434/handouts/}
	{img/}} % set of paths to search for images

\usepackage{geometry}
\geometry{letterpaper, left=1.5in, bottom=1in}                   
%\geometry{landscape}                % Activate for for rotated page geometry
\usepackage[parfill]{parskip}    % Activate to begin paragraphs with an empty line rather than an indent
\usepackage{amssymb, amsmath}
\usepackage{mathtools}
	\everymath{\displaystyle}

\usepackage{fontspec}
\setmainfont[Ligatures={TeX}, BoldFont={* Bold}, ItalicFont={* Italic}, BoldItalicFont={* BoldItalic}, Numbers={OldStyle}]{Linux Libertine O}
\setsansfont[Scale=MatchLowercase,Ligatures=TeX]{Linux Biolinum O}
\setmonofont[Scale=MatchLowercase]{Inconsolatazi4}
\usepackage{microtype}


% To define fonts for particular uses within a document. For example, 
% This sets the Libertine font to use tabular number format for tables.
 %\newfontfamily{\tablenumbers}[Numbers={Monospaced}]{Linux Libertine O}
% \newfontfamily{\libertinedisplay}{Linux Libertine Display O}

\usepackage{booktabs}
\usepackage{multicol}
\usepackage[normalem]{ulem}

\usepackage{longtable}
%\usepackage{siunitx}
\usepackage{array}
\newcolumntype{L}[1]{>{\raggedright\let\newline\\\arraybackslash\hspace{0pt}}p{#1}}
\newcolumntype{C}[1]{>{\centering\let\newline\\\arraybackslash\hspace{0pt}}p{#1}}
\newcolumntype{R}[1]{>{\raggedleft\let\newline\\\arraybackslash\hspace{0pt}}p{#1}}

\usepackage{enumitem}
\usepackage{hyperref}
%\usepackage{placeins} %PRovides \FloatBarrier to flush all floats before a certain point.
\usepackage{hanging}

\usepackage[sc]{titlesec}

%% Commands for Exam class
\renewcommand{\solutiontitle}{\noindent}
\unframedsolutions
\SolutionEmphasis{\bfseries}

\renewcommand{\questionshook}{%
	\setlength{\leftmargin}{-\leftskip}%
}

%Change \half command from 1/2 to .5
\renewcommand*\half{.5}

\pagestyle{headandfoot}
\firstpageheader{\textsc{bi}\,434 MarEvoEco}{}{\ifprintanswers\textbf{KEY}\else Name: \enspace \makebox[2.5in]{\hrulefill}\fi}
\runningheader{}{}{\footnotesize{pg. \thepage}}
\footer{}{}{}
\runningheadrule

\newcommand*\AnswerBox[2]{%
    \parbox[t][#1]{0.92\textwidth}{%
    \begin{solution}#2\end{solution}}
%    \vspace*{\stretch{1}}
}

\newenvironment{AnswerPage}[1]
    {\begin{minipage}[t][#1]{0.92\textwidth}%
    \begin{solution}}
    {\end{solution}\end{minipage}
    \vspace*{\stretch{1}}}

\newlength{\basespace}
\setlength{\basespace}{5\baselineskip}


%\usepackage{mdframed}
%\mdfsetup{%
%	innerleftmargin=0pt,%
%	innerrightmargin=0pt,
%	innertopmargin=0pt,
%	innerbottommargin=0pt,
%	hidealllines=true
%}%end mdfsetup

%
%\makeatletter
%\def\SetTotalwidth{\advance\linewidth by \@totalleftmargin
%\@totalleftmargin=0pt}
%\makeatother


%\printanswers


\begin{document}

\subsection*{Conservation Plan Assignment}

During this semester we will discuss several different marine 
ecosystems, and conservation concerns and risks that could 
affect the future of the ecosystems. You will find that
conservation and restoration efforts are important to many of the
ecosystems we will study. To emphasize this, you will be a member of a team
that must design a conservation plan (or restoration plan if appropriate) for an assigned ecosystem.
Each team will consist of a graduate (or honors contract) student serving as team leader. Other undergraduate students will be valuable members of the team.

Your team has been assigned one of these ecosystems for conservation or restoration.

\begin{multicols}{2}

Coral Reef\\ 
Kelp Forest\\ 
Mangrove Forest\\
Salt Marsh\\
Seagrasses

\end{multicols}

Your team will take one of the four approaches outlined in class and explained in chapter 22 of your text.

\begin{multicols}{2}
\begin{enumerate}
\def\labelenumi{\arabic{enumi}.}
\item
  Population level
\item
  Habitat level
\item
  Landscape level
\item
  Ecosystem level
\end{enumerate}
\end{multicols}

Your team must be composed of different types of scientists that each contributes
to the development of the plan. You will need a \emph{unique} position for each student. One student must serve in a public outreach position to educate the public on the importance of ecosystem services and why the conservation plan must be implemented. 

\subsubsection*{Requirements for Public Outreach Official}

This person is required to analyze and understand the ecosystem
services and to portray their importance to the ``public.'' Without
somehow getting the public invested in the plan there is little hope for
funding or success. This person is in charge of figuring out ways to
pique interest and gain support for your plan. This can be through advertisements, events,
public education, etc. There will be a required section within your
conservation plan that outlines the ecosystem services and provides
empirical support behind them. This position will be crucial to
completing that section of your plan.

Examples of other scientists you might need to involve are
oceanographers, biochemists, fish biologists, ecologists, toxicologists,
and more. The positions you choose for your team will depend on your plan and the goals of your plan. Each student in your group must be have a unique role.

The overall assignment for your team is to write a conservation plan designed to solicit public funding. Writing the plan will be a group effort. Each student must make a significant contribution
to the research and writing of the conservation plan. You must support your ideas and overall
goals with empirical evidence from primary research. The graduate student leader will have the
added responsibility of presenting the plan to the class near the end of the semester. 

Within your conservation plan you need to include the following things:

\begin{enumerate}
\def\labelenumi{\arabic{enumi}.}
\item
  Name of primary investigators and supporting investigators
\item
  Project title that includes ecosystem and approach
\item
  Statement of the problem: Ecosystem services and public importance
\item
  Statement of purpose and goals
\item
  Project location and duration
\item
  Plan of action
\item
  Literature cited
\item
  Proposed budget with justification
\end{enumerate}

I will give you more information on the requirements and grading rubric soon.

\end{document}  