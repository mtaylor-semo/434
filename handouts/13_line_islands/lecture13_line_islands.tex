%!TEX TS-program = lualatex
%!TEX encoding = UTF-8 Unicode

\documentclass[12pt, hidelinks]{exam}
\usepackage{graphicx}
	\graphicspath{{/Users/goby/Pictures/teach/434/handouts/}
	{img/}} % set of paths to search for images

\usepackage{geometry}
\geometry{letterpaper, left=1.5in, bottom=1in}                   
%\geometry{landscape}                % Activate for for rotated page geometry
\usepackage[parfill]{parskip}    % Activate to begin paragraphs with an empty line rather than an indent
\usepackage{amssymb, amsmath}
\usepackage{mathtools}
	\everymath{\displaystyle}

\usepackage{fontspec}
\setmainfont[Ligatures={TeX}, BoldFont={* Bold}, ItalicFont={* Italic}, BoldItalicFont={* BoldItalic}, Numbers={OldStyle}]{Linux Libertine O}
\setsansfont[Scale=MatchLowercase,Ligatures=TeX]{Linux Biolinum O}
\setmonofont[Scale=MatchLowercase]{Inconsolatazi4}
\usepackage{microtype}


% To define fonts for particular uses within a document. For example, 
% This sets the Libertine font to use tabular number format for tables.
 %\newfontfamily{\tablenumbers}[Numbers={Monospaced}]{Linux Libertine O}
% \newfontfamily{\libertinedisplay}{Linux Libertine Display O}

\usepackage{booktabs}
\usepackage{multicol}
\usepackage[normalem]{ulem}

\usepackage{longtable}
%\usepackage{siunitx}
\usepackage{array}
\newcolumntype{L}[1]{>{\raggedright\let\newline\\\arraybackslash\hspace{0pt}}p{#1}}
\newcolumntype{C}[1]{>{\centering\let\newline\\\arraybackslash\hspace{0pt}}p{#1}}
\newcolumntype{R}[1]{>{\raggedleft\let\newline\\\arraybackslash\hspace{0pt}}p{#1}}

\usepackage{enumitem}
\usepackage{hyperref}
%\usepackage{placeins} %PRovides \FloatBarrier to flush all floats before a certain point.
\usepackage{hanging}

\usepackage[sc]{titlesec}

%% Commands for Exam class
\renewcommand{\solutiontitle}{\noindent}
\unframedsolutions
\SolutionEmphasis{\bfseries}

\renewcommand{\questionshook}{%
	\setlength{\leftmargin}{-\leftskip}%
}

%Change \half command from 1/2 to .5
\renewcommand*\half{.5}

\pagestyle{headandfoot}
\firstpageheader{\textsc{bi}\,434 MarEvoEco}{}{\ifprintanswers\textbf{KEY}\else Name: \enspace \makebox[2.5in]{\hrulefill}\fi}
\runningheader{}{}{\footnotesize{pg. \thepage}}
\footer{}{}{}
\runningheadrule

\newcommand*\AnswerBox[2]{%
    \parbox[t][#1]{0.92\textwidth}{%
    \begin{solution}#2\end{solution}}
%    \vspace*{\stretch{1}}
}

\newenvironment{AnswerPage}[1]
    {\begin{minipage}[t][#1]{0.92\textwidth}%
    \begin{solution}}
    {\end{solution}\end{minipage}
    \vspace*{\stretch{1}}}

\newlength{\basespace}
\setlength{\basespace}{5\baselineskip}



%\printanswers


\begin{document}

\subsection*{Line Islands: a case study (10 points)}

The Line Islands, part of the Republic of Kiribiti, are located in the southwest Pacific. Your task is to predict the status of coral reefs on four of the islands. The islands, in alphabetical order, are Kingman, Kiritimati, Palmyra, and Tabuaeran.  Use the table provided to answer the following questions. The questions will guide you to your final predictions.\footnote{Table 1, Sandin et al. \textsc{pl}o\textsc{s one} 3(2):e1548.} 

A few things to note about the table. Read the footnotes of the table to learn the meaning of some column headers. Some columns have numbers, followed by numbers in parentheses. The numbers in parentheses are some type of variability measure. You only need to read about the number \emph{not} in parentheses.

\begin{questions}

\question
List the islands in order of population size, from smallest to largest.

\vspace*{\stretch{1}}

\question
Which islands allow only subsistence fishing? Which islands allow commercial fishing? Which islands do not allow fishing?

\vspace*{\stretch{1}}

\question
List the islands in order of sea surface temperature, from warmest to coolest.

\vspace*{\stretch{1}}

\question
List the islands in order of Degree Heating Weeks, from fewest to most. Degree Heating Weeks is a measure of thermal stress. The higher the number of degree heating weeks, the more days the water temperature has been higher than ideal conditions for coral growth.

\vspace*{\stretch{1}}

\question
List the islands in order of \textsc{noaa} Level 2 bleaching event frequency, from lowest to highest. Level 2 events are the strongest (worst) bleaching events. Which islands have experienced the greatest number of Level 2 events

\vspace*{\stretch{1}}

\newpage

\question
Rank the islands in order of eutrophication, from most oligotrophic to most eutrophic. Look at the amounts of nitrogen and phosphorus. Remember that chlorophyll is a measure of photosynthetic activity.

\vspace*{\stretch{1}}

\subsection*{Make your predictions}

\question
Which island(s) has the healthiest reefs (rugosity and coral cover)? Which island(s) has the least healthiest reefs (rugosity and coral cover)?

\vspace*{\stretch{1}}

\question
Which island(s) has the greatest amount of macroalgae coverage? Which island(s) has the least amount of macroalgae coverage?

\vspace*{\stretch{1}}

\question
Which island(s) has the highest coral species richness? Which island(s) has the lowest coral species richness?

\vspace*{\stretch{1}}

\question
Which island(s) has the greatest number of top predator fishes? Which island(s) has the greatest number of herbivorous fishes?  Does the same hold true for biomass?

\vspace*{\stretch{1}}

\end{questions}

\end{document}  