%!TEX TS-program = lualatex
%!TEX encoding = UTF-8 Unicode

\documentclass[11pt]{article}
%\usepackage{graphicx}
%	\graphicspath{{/Users/goby/Pictures/teach/153/lab/}} % set of paths to search for images

\usepackage{geometry}
\geometry{letterpaper}                   
\geometry{bottom=1in}
%\geometry{landscape}                % Activate for for rotated page geometry
%\usepackage[parfill]{parskip}    % Activate to begin paragraphs with an empty line rather than an indent
%\usepackage{amssymb}
%\usepackage{mathtools}
%	\everymath{\displaystyle}

%\pagenumbering{gobble}

\usepackage{fontspec}
\setmainfont[Ligatures={Common,TeX}, BoldFont={* Bold}, ItalicFont={* Italic}, Numbers={Proportional}]{Linux Libertine O}
\setsansfont[Scale=MatchLowercase,Ligatures=TeX]{Linux Biolinum O}
\setmonofont[Scale=MatchLowercase]{Inconsolata}
\usepackage{microtype}

\usepackage{unicode-math}
\setmathfont[Scale=MatchLowercase]{Asana-Math.otf}
%\setmathfont{XITS Math}

% To define fonts for particular uses within a document. For example, 
% This sets the Libertine font to use tabular number format for tables.
%\newfontfamily{\tablenumbers}[Numbers={Monospaced}]{Linux Libertine O}
%\newfontfamily{\libertinedisplay}{Linux Libertine Display O}


\usepackage{booktabs}
%\usepackage{tabularx}
%\usepackage{longtable}
%\usepackage{siunitx}
%\usepackage[justification=raggedright, singlelinecheck=off]{caption}
%\captionsetup{labelsep=period} % Removes colon following figure / table number.
%\captionsetup{tablewithin=none}  % Sequential numbering of tables and figures instead of
%\captionsetup{figurewithin=none} % resetting numbers within each chapter (Intro, M&M, etc.)
%\captionsetup[table]{skip=0pt}

\usepackage{array}
\newcolumntype{L}[1]{>{\raggedright\let\newline\\\arraybackslash\hspace{0pt}}p{#1}}
\newcolumntype{C}[1]{>{\centering\let\newline\\\arraybackslash\hspace{0pt}}p{#1}}
\newcolumntype{R}[1]{>{\raggedleft\let\newline\\\arraybackslash\hspace{0pt}}p{#1}}

\usepackage{enumitem}
\usepackage{hyperref}
%\usepackage{placeins} %PRovides \FloatBarrier to flush all floats before a certain point.
\usepackage{hanging}

\usepackage{titling}
\setlength{\droptitle}{-60pt}
\posttitle{\par\end{center}}
\predate{}\postdate{}

\usepackage{fancyhdr}
\fancyhf{}
\pagestyle{fancy}
\lhead{}
\chead{}
\rhead{\footnotesize Presentation and Proposal Guidelines, pg. \thepage }
\renewcommand{\headrulewidth}{0.4pt}

\newcommand{\VSpace}{\vspace{\baselineskip}}
\newcommand{\BigVSpace}{\vspace{2\baselineskip}}


\title{Presentation and Proposal Guidelines}
\author{Marine Evolutionary Ecology}
\date{}                                           % Activate to display a given date or no date

\begin{document}
\maketitle
\thispagestyle{plain}

As a graduate or honors student, you must deliver a presentation to the class and submit the associated literature source material to me. You must also submit a written research proposal related to your presentation topic. The presentation is worth 60 points, the literature cited is worth 60 points, and the proposal is worth 50 points.  The presentations will be given in the few days prior to the final exam. The order of the presentations will be determined by random draw early in the semester. See the syllabus regarding the presentation attendance policy.  

\subsection*{Presentation}

You will give a 10--12 minute PowerPoint presentation on a broad conceptual issue related to evolution or ecology in the marine environment.  The topic is not about the biology of any particular organism or group of organisms, although you may use specific taxa to illustrate key points about the concept.  You may choose from one of the topics listed below or you may suggest a topic of your own.  No more than two students may present the same topic.  Your chosen topic must be approved by me before you begin.  You may not change your topic once we have agreed on your topic.  

The purpose of the presentation is for you to teach the class about your chosen topic or issue.  You should explain or define the conceptual issue, provide some background information on the issue, then describe a few studies that have attempted to address the issue.  A comparison of two or more studies that provide contrasting or conflicting results is especially good.  The audience should take notes because the information provided in any presentation is fair game for a short-answer exam question.  I will provide each student with a printout of each presentation for reference.

\subsection*{Literature Cited}

You must base your presentation on the primary scientific literature (i.e., journal articles and chapters from an edited book).  You may \emph{not} use any other type of reference.  You must turn in to me photocopies of the actual pages of the source material, with the specific passages marked or highlighted, that you used to develop your presentation.  I want to be as certain as possible that you are actually delving into the literature and not building your thoughts and presentations by skimming the abstracts.  Abstracts provide only superficial explanations of the background, the results, and the importance of the results.  You must dive headfirst into the paper to truly understand the context and implication of the results.  

You must use a minimum of six citations from the primary literature.  At least three must have been published in 2000 or later.  deally, most of your literature should be current but some issues are relatively old (e.g., Sanders 1968) but still relevant.  You can earn two extra credit points per citation (2010 or later) above six that is used effectively in your presentation.  No more than 6 extra credit points (10\% of total possible points) will be awarded.  You must also include a separate page listing your references.  See the grading rubric for details.

You must turn in a separate sheet of your literature cited from your presentation and proposal.  The citation format should follow the examples provided in the rubric. More importantly, however, your format must be consistent.  You do not need to include a slide of your cited literature but each slide must contain citations to the literature, as appropriate. You must include a Literature Cited section in your proposal but it must be in abbreviated form, as defined by the proposal rubric and as shown by the example proposal.

Start early so that you have plenty of time to gather the literature you need for your presentation.  An early homework assignment will show you an easy way to obtain the current literature.

\subsection*{Proposal}

Your scientific proposal must be a single-spaced 2--3 page document related to the presented issue.  Use the proposal guidelines outlined below.  Have fun with the budget because the funding agency (The MT Head Foundation) has an unlimited budget; however, be realistic in terms of actual costs.  You must justify the need for each item in your  budget justification. \VSpace

\noindent\emph{Proposal sections} (modified from the American Museum of Natural History guidelines)\vspace{0.5\baselineskip}

Describe your project using no less than two and no more than three separate pages, single-spaced.  Include the following information: 
\begin{enumerate}
	\item name of investigator 
	\item project title 
	\item project location and duration 
	\item project description
	\item methodology
	\item literature cited 
	\item proposed budget and justification 
\end{enumerate}
	
The project description must include at least one clearly defined research objective, a hypothesis and prediction, and an explanation of the significance of your proposed research.  I will provide you with an example proposal. \VSpace

\noindent\emph{Formatting}\vspace{0.5\baselineskip}

You may use no smaller than 10 point type in a serif font (e.g., Times New Roman) and no larger than 12 point type.  Use 1" margins on all sides.

\subsection*{Proposal Judging}

As the final homework assignment for this course, each student will be required to judge the proposals.  Prior to the start of the presentations, each student will receive a copy of each proposal.  You must read and judge each of the proposals based on the quality of the science and the clarity.  You must also write a brief summary of each proposal, indicating the strengths and weaknesses.  You will not judge the budget and justification sections of the proposal.  

You should consider several points when judging the proposals:
\begin{itemize}
	\item Is the objective clearly stated?  
	\item Does the objective seem reasonable and follow logically from the project description?  
	\item Does the objective and predicted outcome seem like it would be a significant contribution to the field?
	\item Did the project description provide enough detail to illustrate the issue in a broader biological (not necessarily just marine) context?
	\item Does the methodology and time frame seem suitable to accomplish the proposed research?
	\item Is the proposal written clearly and concisely? Was it easy to follow the thought process?
\end{itemize}

Your judging should be as objective as possible.  You should not elevate your friend’s proposal to the highest level unless merited by the quality of the proposal.  Each graduate student should try to be objective when evaluating her or his proposal against the other proposals.  The outcome of the student rankings will not affect your grade.

\subsection*{Due Dates}

\textit{Thursday, 19 February:} You must submit your chosen topic no later than this date but I encourage you to submit earlier. I will approve topics in the order that I receive the requests (first come, first serve). Remember that no more than two students may choose the same topic.\VSpace

\noindent\textit{Tuesday, 21 April:}  All presentations, bibliographies, and proposals must be complete and turned in by this date to ensure that all students have the same amount of time to finish the assigned tasks.  I will deduct points if you modify your presentation after the submission date.

\subsection*{Grading}

I will grade each assignment by following the attached rubrics.  The rubrics also serve as guidelines to help you prepare each assignment.

\subsection*{Potential Topics}

Below is a list of just a few of the many topics suitable for your presentation and proposal.  Each has plenty of literature necessary for you to develop a solid presentation.  You may also suggest a topic.  I encourage you to look through your text and journals for topics of interest to you.  Remember that I must approve your topic before you begin.  Choose early: no more than two students may present on the same topic.

\begin{itemize}
	\item Planktonic larval dispersal: does it happen?  Evidence for and against. 

	\item Recruitment of fishes into reef communities: choose one hypothesis:
	\begin{itemize}[label=$\circ$]
		\item recruitment limitation
		\item competition
		\item lottery hypothesis
		\item predation
	\end{itemize}
		
	\item Ecological diversity of deep-sea benthic communities.

	\item Hydrothermal vents and the origin of life (I’m honestly not sure of the literature base for this)

	\item Succession in hydrothermal vent communities

	\item Biogeographic barriers and organismal distribution.  The barrier should not be obvious physical barrier such as the Isthmus of Panama between the Atlantic and Pacific oceans.  Instead, it should be a potential barrier that is not readily apparent.  Suggestions include:
	\begin{itemize}[label=$\circ$]
		\item East Pacific Barrier
		\item Mona Passage
		\item The Gulfstream
		\item Marine Wallace’s Line
		\item Point Conception in California
	\end{itemize}
	\item Ocean acidification and global warming

	\item Coral bleaching: What is coral bleaching and what causes it?

	\item Coral decline: Why are coral reefs on the decline?

	\item Iron enrichment (fertilization) of the Southern Ocean and CO$_2$ sequestration.  Will dumping iron into the Southern Ocean reduce the amount of a major greenhouse gas in the atmosphere?

	\item Any other topic related to global warming and the marine environment

	\item Deep-Sea miniaturization and gigantism: Why are many deep-sea organisms relatively large?  (see page 160 of text)

	\item Sweepstakes effect and population genetic structure (Hedgecock). Does variance in reproductive success limit effective population sizes of marine organisms?  The original 1994 pub on this is rather obscure but I can provide you with a copy.

	\item Cleaning symbiosis:  do hosts or clients use signals to initiate the interaction?

	\item Cleaning symbiosis: why don’t the clients cheat?  (Cheating is the client eating the cleaner host).

	\item Effects of marine reserves on adjacent fisheries.

	\item Endemism in the marine environment.

	\item Ooooooodles of other cool and interesting topics.  Browse the literature and use your imagination but remember that I must approve your choice.

\end{itemize}

\end{document}  