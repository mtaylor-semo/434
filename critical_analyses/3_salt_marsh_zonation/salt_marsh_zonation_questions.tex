%!TEX TS-program = lualatex
%!TEX encoding = UTF-8 Unicode

\documentclass[11pt, addpoints]{exam}
\usepackage{graphicx}
	\graphicspath{{/Users/goby/Pictures/teach/300/}
	{img/}} % set of paths to search for images

\usepackage{geometry}
\geometry{letterpaper, bottom=1in}                   
%\geometry{landscape}                % Activate for for rotated page geometry
%\usepackage[parfill]{parskip}    % Activate to begin paragraphs with an empty line rather than an indent
\usepackage{amssymb, amsmath}
\usepackage{mathtools}
	\everymath{\displaystyle}

\usepackage{fontspec}
\setmainfont[Ligatures={TeX}, BoldFont={* Bold}, ItalicFont={* Italic}, BoldItalicFont={* BoldItalic}, Numbers={Proportional}]{Linux Libertine O}
\setsansfont[Scale=MatchLowercase,Ligatures=TeX]{Linux Biolinum O}
\setmonofont[Scale=MatchLowercase]{Inconsolata}
\usepackage{microtype}

\usepackage{unicode-math}
\setmathfont[Scale=MatchLowercase]{Asana Math}

\newfontfamily{\tablenumbers}[Numbers={Monospaced}]{Linux Libertine O}
\newfontfamily{\libertinedisplay}{Linux Libertine Display O}

\usepackage{booktabs}
%\usepackage{tabularx}
%\usepackage{longtable}
%\usepackage{siunitx}

\usepackage{hanging}

\usepackage{array}
\newcolumntype{L}[1]{>{\raggedright\let\newline\\\arraybackslash\hspace{0pt}}p{#1}}
\newcolumntype{C}[1]{>{\centering\let\newline\\\arraybackslash\hspace{0pt}}p{#1}}
\newcolumntype{R}[1]{>{\raggedleft\let\newline\\\arraybackslash\hspace{0pt}}p{#1}}

%\usepackage{enumitem}

%\usepackage{titling}
%\setlength{\droptitle}{-60pt}
%\posttitle{\par\end{center}}
%\predate{}\postdate{}

\renewcommand{\solutiontitle}{\noindent}
\unframedsolutions
\SolutionEmphasis{\bfseries}

\pagestyle{headandfoot}
\firstpageheader{BI 434: Marine Evolutionary Ecology}{}{Critical Analysis 2}
\runningheader{}{}{\footnotesize{pg. \thepage}}
\footer{}{}{}
\runningheadrule

%\printanswers

\begin{document}

\subsection*{Trophic Cascades (\numpoints\ points)}

Read the following two papers carefully, and then answer the following questions. Type the question number and then your answer. You do not need to retype the question. All questions must be answered in your own words. Do not copy and quote from the papers. The first several questions emphasize Bruno et al. because we discussed many of the important findings from Estes et al. in class.\vspace{\baselineskip}

\begin{hangparas}{1.5em}{1}
Bruno, J.F. and M.I. O’Conner. 2005. Cascading effects of predator diversity and omnivore in a marine food web. Ecology Letters 8: 1048--1056.\vspace{\baselineskip}

Estes, J.A., M.T. Tinker, T.M. Williams and D.F. Doak. 1998. Killer whale predation on sea otters linking ocean and nearshore ecosystem. Science 282: 473--476.
\end{hangparas}

\begin{questions}

\question[6]
Interpret Figure 1 of Bruno et al. Do not tell me what the figure legend says. Provide specific examples of positive and negative effects for direct and indirect interactions to demonstrate your understanding. 

\question[4]
Why is the pinfish unique compared to the other predators used in the Bruno et al. study? How is this shown in Figure 1?

\question[4]
Figure 4 of Bruno et al. shows that that mesocosms with the generalist carnivores had the greatest macroalgal diversity. Explain why. 

\question[5]
Figure 3b of Bruno et al. showed that 3 predator species had high abundance of \textit{Enteromorpha} but 5 predator species had no \textit{Enteromorpha} remaining in the mesocosm. Explain why. What result shown in the same figure  supports your answer?

\question[5]
The results of Estes et al. suggest that human overfishing may have greatly decreased overall diversity of invertebrates in the ecosystem, despite the increased abundance of sea urchins. Explain why. (Hint: remember how you answered question 7 on page 459 of your text.)

\question[8]
Estes et al. argued that abundance of pinnipeds declined, causing orca to begin feeding on otters. Assume that pinniped abundance makes a partial recovery such that orca begin to feed on pinnipeds and otters in roughly equal numbers. Predict how kelp abundance and diversity would change. Use conclusions from Bruno et al. to support your answer. 

\question[8]
Estes et al. showed a strong trophic cascade in a near-Arctic ecosystem off the coast of Alaska.  Bruno et al. demonstrated that the trophic cascade was often weaker in a temperate ecosystem (North Carolina estuary). Hypothesize how strong a trophic cascade is likely to be in a tropical ecosystem. Paraphrase relevant passages from Bruno et al. (include page number in your citations) to support your answer.

\end{questions}

\end{document}  