%!TEX TS-program = lualatex
%!TEX encoding = UTF-8 Unicode

\documentclass[12pt, addpoints]{exam}

%\printanswers

\usepackage{fontspec}
\setmainfont[Ligatures={Common,TeX}, BoldFont={* Bold}, ItalicFont={* Italic}, BoldItalicFont={* BoldItalic}, Numbers={OldStyle}]{Linux Libertine O}
\setsansfont[Scale=MatchLowercase,Ligatures=TeX, Numbers=OldStyle]{Linux Biolinum O}
\usepackage{microtype}


\usepackage{geometry}
\geometry{letterpaper, bottom=1in}                   
\usepackage[parfill]{parskip} 

\newfontfamily{\tablenumbers}[Numbers={Monospaced}]{Linux Libertine O}
\newfontfamily{\libertinedisplay}{Linux Libertine Display O}

\usepackage{hanging}

\usepackage[sc]{titlesec}

\usepackage{array}
\newcolumntype{L}[1]{>{\raggedright\let\newline\\\arraybackslash\hspace{0pt}}p{#1}}
\newcolumntype{C}[1]{>{\centering\let\newline\\\arraybackslash\hspace{0pt}}p{#1}}
\newcolumntype{R}[1]{>{\raggedleft\let\newline\\\arraybackslash\hspace{0pt}}p{#1}}

\usepackage{enumitem}
\setlist{leftmargin=*}
\setlist[1]{labelindent=\parindent}
\setlist[enumerate]{label=\textsc{\alph*}.}
\setlist[itemize]{label=\color{gray}\textbullet}

\renewcommand{\questionshook}{%
	\setlength{\leftmargin}{-\leftskip}%
}

\renewcommand{\solutiontitle}{\noindent}
\unframedsolutions
\SolutionEmphasis{\bfseries}

\pagestyle{headandfoot}
\firstpageheader{\textsc{bi} 434: Marine Ecology and Conservation}{}{\ifprintanswers\textbf{KEY}\else Critical Analysis\fi}
\runningheader{}{}{\footnotesize{pg. \thepage}}
\footer{}{}{}
\runningheadrule

\newcommand*\AnswerBox[2]{%
    \parbox[t][#1]{0.92\textwidth}{%
    \begin{solution}#2\end{solution}}
%    \vspace*{\stretch{1}}
}

\newenvironment{AnswerPage}[1]
    {\begin{minipage}[t][#1]{0.92\textwidth}%
    \begin{solution}}
    {\end{solution}\end{minipage}
    \vspace*{\stretch{1}}}

\newlength{\basespace}
\setlength{\basespace}{5\baselineskip}

\begin{document}

\subsection*{Climate change and ecosystem resilience (\numpoints\ points)}

Read carefully the paper listed below and then type your answers to the following questions.
Type the question number and then your answer. You do not need to retype the question. Hand-written
assignments will not be accepted. You must have your completed answers
with you in class or your assignment will be considered late. Assignments e-mailed to me after the \emph{start} of class will be
considered late. Be prepared to discuss your answers in small groups and as a class. Failure
to discuss these questions may result in a pop quiz with a point value
equal to this assignment.

\emph{Interpret} what you read. Do not copy answers directly from the text (that would
be plagiarism). Show me you understand the paper.

\textbf{Failure to attend} \textbf{class} \textbf{on the due date} will
result in an automatic 50\% deduction.

\begin{hangparas}{1.5em}{1}
Bernhardt, J.\,R.~and H.\,M.~Leslie.~2013. Resilience to climate change in 
coastal marine ecosystems. Annual Review of Marine Science 2013: 371--392.

\end{hangparas}


\begin{questions}

\question[5]
The authors discuss response diversity. Based on their explanation, how might 
high genetic diversity affect the degree of hysteresis between two alternate 
stable states?

\begin{solution}
Greater genetic variability may confer a greater variability of response in variable environments. Studies suggest that species with greater genetic variation have increased resistance to disturbance and return more quickly to original conditions. Thus, might change from a high to low state of hysteresis. (pg.~375)
\end{solution}

\question[5]
In class, we saw data that showed how climate change, specifically
warming temperatures, might affect larval dispersal. Describe how 
larval dispersal might be affected, and then how this could affect 
population response diversity and recovery?


\begin{solution}
	Larval dispersal decreased, increasing isolation of populations.
	This would reduce genetic diversity in the population, reducing
	the diversity of response (ability to respond to wide-ranging
	environmental conditions), and thus reduce ability to recover.
	
	This question attempts to tie together population connectivity (pg.~378, lecture) with response diversity (pg.~375).
\end{solution}

\question[5]
What is evolutionary rescue? (I put a supplementary paper online that may help you figure it out.) How does modularity affect the potential for evolutionary rescue in the face of climate change?


\begin{solution}
	Evolutionary rescue occurs when a few individuals are able 
	to adapt to rapidly changing conditions in the face of population 
	decline. These individuals reproduce and prevent the population 
	from going extinct. According to the authors (pg.~379), 
	modularity is the opposite of connectivity. Thus, there would be
	little to no gene flow among populations, decreasing gene flow and
	thus reducing genetic variation. Less genetic variation reduces the
	chance of rescue because less likely to have individuals with
	suitable mutations to confer adaptive flexibility.
\end{solution}


\question[5]
Assume you are managing a small, isolated population of an endangered 
species (high modularity, no potential for connectivity). What single 
management step do you think you could take to increase the chance
for evolutionary rescue in the face of climate change? (See page 383 and Table 1 for clues.)


\begin{solution}
Establish a marine reserve that emphasizes population growth to create a large population. Large populations tend to harbor more genetic diversity, lending to greater resilience and potential for evolutionary rescue.
\end{solution}

\question[5]
Explain how the strength and numbers of species interactions could affect functional diversity and resilience in the face of climate change.


\begin{solution}
More species present tends to increase functional diversity and to reduce the strength of species interactions. With more species and weaker interactions, abundance of each species tends to remain relatively stable. Also, because of higher functional diversity, the temporary reduction of one species does not affect ecosystem function very much. Overall, more species will increase the links in the food web, and maintain greater ecosystem resilience. (pg.~377)
\end{solution}

\end{questions}

%\ifprintanswers
%
%\newpage
%
%\subsubsection*{Other questions to ask:}
%
%\begin{enumerate}[label=\alph*.]
%
%	\item Anything?
%	
%\end{enumerate}
%
%\fi

\end{document}  